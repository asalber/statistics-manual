%% Input file name: distribucion_asimetrica_izquierda.fig
%% FIG version: 3.2
%% Orientation: Landscape
%% Justification: Flush Left
%% Units: Inches
%% Paper size: A4
%% Magnification: 100.0
%% Resolution: 1200ppi
%% Include the following in the preamble:
%% \usepackage{textcomp}
%% End

\begin{pspicture}(5.91cm,3.25cm)(17.07cm,13.10cm)
\psset{unit=0.8cm}
%%
%% Depth: 2147483647
%%
\newrgbcolor{mycolor0}{1.00 0.50 0.31}\definecolor{mycolor0}{rgb}{1.00,0.50,0.31}
%%
%% Depth: 100
%%
\rput[l](12.07,15.54){Distribución asimétrica a la izquierda \alert{$g_1<0$}}
\rput{90}(9.68,10.89){Frecuencia relativa}
\psset{linestyle=solid,linewidth=0.03175,linecolor=black,fillstyle=none}
\psline(11.04,7.07)(11.04,14.71)
\psline(11.04,7.07)(10.83,7.07)
\psline(11.04,8.34)(10.83,8.34)
\psline(11.04,9.61)(10.83,9.61)
\psline(11.04,10.89)(10.83,10.89)
\psline(11.04,12.16)(10.83,12.16)
\psline(11.04,13.44)(10.83,13.44)
\psline(11.04,14.71)(10.83,14.71)
\rput{90}(10.53,7.07){0.00}
\rput{90}(10.53,8.34){0.02}
\rput{90}(10.53,9.61){0.04}
\rput{90}(10.53,10.89){0.06}
\rput{90}(10.53,12.16){0.08}
\rput{90}(10.53,13.44){0.10}
\rput{90}(10.53,14.71){0.12}
\psset{fillstyle=solid,fillcolor=mycolor0}
\pspolygon(10.79,7.07)(10.79,7.07)(11.42,7.07)(11.42,7.07)(10.79,7.07)
\pspolygon(11.42,7.07)(11.42,7.08)(12.04,7.08)(12.04,7.07)(11.42,7.07)
\pspolygon(12.04,7.07)(12.04,7.08)(12.67,7.08)(12.67,7.07)(12.04,7.07)
\pspolygon(12.67,7.07)(12.67,7.11)(13.30,7.11)(13.30,7.07)(12.67,7.07)
\pspolygon(13.30,7.07)(13.30,7.15)(13.92,7.15)(13.92,7.07)(13.30,7.07)
\pspolygon(13.92,7.07)(13.92,7.24)(14.55,7.24)(14.55,7.07)(13.92,7.07)
\pspolygon(14.55,7.07)(14.55,7.42)(15.18,7.42)(15.18,7.07)(14.55,7.07)
\pspolygon(15.18,7.07)(15.18,7.75)(15.81,7.75)(15.81,7.07)(15.18,7.07)
\pspolygon(15.81,7.07)(15.81,8.32)(16.43,8.32)(16.43,7.07)(15.81,7.07)
\pspolygon(16.43,7.07)(16.43,9.29)(17.06,9.29)(17.06,7.07)(16.43,7.07)
\pspolygon(17.06,7.07)(17.06,10.70)(17.69,10.70)(17.69,7.07)(17.06,7.07)
\pspolygon(17.69,7.07)(17.69,12.43)(18.31,12.43)(18.31,7.07)(17.69,7.07)
\pspolygon(18.31,7.07)(18.31,13.85)(18.94,13.85)(18.94,7.07)(18.31,7.07)
\pspolygon(18.94,7.07)(18.94,13.75)(19.57,13.75)(19.57,7.07)(18.94,7.07)
\pspolygon(19.57,7.07)(19.57,11.01)(20.20,11.01)(20.20,7.07)(19.57,7.07)
\pspolygon(20.20,7.07)(20.20,7.67)(20.82,7.67)(20.82,7.07)(20.20,7.07)
\rput[l](18.22,6.5){$\bar x$}
\psset{linewidth=0.0635,fillstyle=none}
\psline(19.34,4.58)(19.34,5.76)
\psset{linestyle=dashed,linewidth=0.03175}
\psline(17.19,5.17)(18.76,5.17)
\psline(20.82,5.17)(19.80,5.17)
\psset{linestyle=solid}
\psline(17.19,4.88)(17.19,5.47)
\psline(20.82,4.88)(20.82,5.47)
\psline(18.76,4.58)(18.76,5.76)(19.80,5.76)(19.80,4.58)(18.76,4.58)
\end{pspicture}
%% End
