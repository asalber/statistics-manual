% !TEX program = xelatex
% Author: Alfredo Sánchez Alberca (asalber@ceu.es)
\documentclass[a4paper, 10pt]{article}
\usepackage[top=3cm, bottom=2.5cm, left=2cm, right=2cm, headsep=1cm]{geometry}
\usepackage[no-math]{fontspec}
\setmainfont[BoldFont={Fira Sans}]{Fira Sans Light}
\setmonofont{Fira Mono}
%%%%%%%%%%%%%%%%%%%%%%%%%%%%
\usepackage{multicol}
\usepackage{macros}
\usepackage{tikz}
\usepackage{lmodern}
\usepackage{graphicx}
\usepackage{fancybox}
\usepackage[most]{tcolorbox}
\usepackage{varwidth}
\usepackage{fancyhdr}
\pagestyle{fancy}
\usepackage[pdfauthor={Alfredo Sánchez Alberca}, pdftitle={Statistical formulas}, colorlinks=true]{hyperref}

% COLORS
\usepackage{xcolor}
\definecolor{color1}{RGB}{5,161,230} % Blue
\definecolor{color2}{RGB}{238,50,36} % Red
\definecolor{color3}{RGB}{0,205,0} % Green
\definecolor{color4}{RGB}{243,102,25} % Orange
\definecolor{ocre}{RGB}{243,102,25} % Define the orange color used for highlighting throughout the book
\definecolor{blueceu}{RGB}{5,161,230} % Blue color of CEU logo
\definecolor{greenceu}{RGB}{185,209,16} % Green color of CEU logo
\definecolor{redceu}{RGB}{238,50,36} % Red color of CEU logo
\definecolor{grayceu}{RGB}{111,107,83} % Gray color of CEU logo
\definecolor{coral}{rgb}{1,0.5,0.31} % Orange color for graphics
\definecolor{royalblue1}{rgb}{0.28,0.46,1} % Blue color for graphics
\definecolor{mygreen}{rgb}{0,0.8,0} % Green color for graphics
\definecolor{chaptergrey}{RGB}{5,161,230} % Blue color of CEU logo
\definecolor{DarkBrown}{HTML}{604c38} % Brown of Metropoly theme
\definecolor{DarkTeal}{HTML}{23373b} % Teal of Metropoly theme

% Lists
\usepackage[shortlabels]{enumitem} % Customize lists
\setlist{nolistsep} % Reduce spacing between bullet points and numbered lists
\setlist[description]{style=sameline,leftmargin=0cm}

\makeatletter
\let\savees@listquot\es@listquot
\def\es@listquot{\protect\savees@listquot}
\makeatletter


\lhead{\textsc{San Pablo CEU University}} \rhead{\url{http://aprendeconalf.es}}
\renewcommand{\headrulewidth}{0pt}
\renewcommand{\floatpagefraction}{.8}
\renewcommand{\textfraction}{.1}

\setlength{\columnsep}{1cm}


\newlength{\mylength}
\newenvironment{marco}{
	\setlength{\fboxsep}{5pt}
	\setlength{\mylength}{\textwidth}
	\addtolength{\mylength}{-2\fboxsep}
	\addtolength{\mylength}{-2\fboxrule}
	\noindent
	\begin{Sbox}
	\begin{minipage}{\mylength}
	\setlength{\abovedisplayskip}{3pt}
	\setlength{\belowdisplayskip}{3pt}
}
{
	\end{minipage}
	\end{Sbox}
	\fbox{\TheSbox}
}

\begin{document}
\sloppy
\section*{\sffamily\huge\color{color1} Statistical Formulas}

\footnotesize
\tcbset{enhanced, colback=color1!10, colframe=color1, fonttitle=\bfseries\large\sffamily, %lifted shadow={1mm}{-2mm}{3mm}{0.1mm}
}
\subsection*{\sffamily Descriptive Statistics}
\begin{multicols}{2}
\begin{tcolorbox}[hbox, title=Frequencies]
\begin{varwidth}{0.4\textwidth}
	\begin{description}
		\item [Sample size] $n$ num of individuals in the sample.
	\end{description}
	\begin{description}
		\item [Absolute frequency] $n_i$. (num of $x_i$ in the sample)
		\item [Relative frequency] $f_i=n_i/n$.
		\item [Cumulative absolute freq] $N_i=\sum_{k=0}^in_i$.
		\item [Cumulative relative freq] $F_i=N_i/n$.
	\end{description}
\end{varwidth}
\end{tcolorbox}

\medskip

\begin{tcolorbox}[hbox, title=Central tendency statistics]
\begin{varwidth}{0.4\textwidth}
	\begin{description}
		\item [Mean] $\bar{x}=\dfrac{\sum x_in_i}{n}$.
		\item [Median] $me$ The value with cum.rel.freq. $F_{me}=0.5$.
		\item [Mode] $mo$ The most frequent value.
	\end{description}
\end{varwidth}
\end{tcolorbox}

\medskip

\begin{tcolorbox}[hbox, title=Position statistics]
\begin{varwidth}{0.4\textwidth}
	\begin{description}
		\item [Quartiles] $Q_1,Q_2,Q_3$ divide the distribution into 4 equal parts.
		Their cum.rel.freqs. are
		$F_{Q_1}=0.25$, $F_{Q_2}=0.5$ and $F_{Q_3}=0.75$.
		\item [Percentiles] $P_1,P_2,\cdots,P_{99}$ divide the distribution into 100 equal parts.\\
		The cum.rel.freq. is $F_{P_i}=i/100$.
	\end{description}
\end{varwidth}
\end{tcolorbox}

\medskip

\begin{tcolorbox}[hbox, title=Dispersion statistics]
\begin{varwidth}{0.4\textwidth}
	\begin{description}
	\item [Interquartile range] $IQR=Q_3-Q_1$.
	\item [Variance] $s^2=\dfrac{\sum (x_i-\bar x)^2n_i}{n}=\dfrac{\sum x_i^2n_i}{n}-\bar x^2$
	\item [Standard deviation] $s=+\sqrt{s^2}$.
	\item [Coefficient of variation] $cv=\dfrac{s}{|\bar{x}|}$.
	\end{description}
\end{varwidth}
\end{tcolorbox}

\medskip

\begin{tcolorbox}[hbox, title=Shape statistics]
\begin{varwidth}{0.4\textwidth}
	\begin{description}
		\item [Coefficient of skewness] $g_1=\dfrac{\sum(x_i-\bar{x})^3f_i}{s^3}.$
		\item [Coefficient of kurtosis] $g_2=\dfrac{\sum(x_i-\bar{x})^4f_i}{s^4}-3.$
	\end{description}
\end{varwidth}
\end{tcolorbox}

\medskip

\begin{tcolorbox}[hbox, title=Standarization]
\begin{varwidth}{0.4\textwidth}
\rule{0.45\textwidth}{0pt}
\[z=\frac{x-\bar x}{s_x}\]
\end{varwidth}
\end{tcolorbox}
\end{multicols}

\subsection*{\sffamily Regression and correlation}
\begin{multicols}{2}
\begin{tcolorbox}[hbox, title=Linear regression]
\begin{varwidth}{0.4\textwidth}
	\rule{\textwidth}{0pt}
	\begin{description}
		\item [Covariance] $s_{xy}=\dfrac{\sum x_iy_jn_{ij}}{n}-\bar{x}\bar{y}$.
		\item [Regression lines]:
		\begin{description}
			\item [$y$ on $x$:] $y=\bar{y}+\dfrac{s_{xy}}{s_x^2}(x-\bar{x})$
			\item [$x$ on $y$:]	$x=\bar{x}+\dfrac{s_{xy}}{s_y^2}(y-\bar{y})$
		\end{description}
		\item [Regression coefficients]
		\[
		\mbox{($y$ on $x$) } b_{yx}=\dfrac{s_{xy}}{s_x^2}\quad \mbox{($x$ on
		$y$) } b_{xy}=\dfrac{s_{xy}}{s_y^2}
		\]
		\item[Coefficient of determination]
		\[r^2=\dfrac{s_{xy}^2}{s_x^2s_y^2} \qquad 0\leq r^2\leq 1\]

		\item[Correlation coefficient]
		\[r=\dfrac{s_{xy}}{s_xs_y}.\qquad -1\leq r\leq 1\]
	\end{description}
\end{varwidth}
\end{tcolorbox}

\medskip

\begin{tcolorbox}[hbox, title=Non-linear regression]
\begin{varwidth}{0.4\textwidth}
	\begin{description}
	  \item[Exponential model] $y=e^{a+bx}$\\
	  Apply the logarithm to the dependent variable and compute the line $\log y = a+bx$.
	  \item[Logarithmic model] $y=a+b\log x$\\
	  Apply the logarithm to the independent variable and compute the line $y=a+b\log x$.
	  \item[Potential model] $y=ax^b$\\
	  Apply the logarithm to both variables and compute the line $\log y = a+b\log x$.
	\end{description}
\end{varwidth}
\end{tcolorbox}
\end{multicols}

\end{document}
