\section{Regresión y Correlación}

\mode<presentation>{
%---------------------------------------------------------------------slide----
\begin{frame}
\frametitle{Regression and Correlation}
\tableofcontents[sectionstyle=show/hide,hideothersubsections]
\end{frame}
}


\subsection{Joint frequency distribution}

% ---------------------------------------------------------------------slide----
\begin{frame}
\frametitle{Relations among variables}
In the last chapter we saw how to describe the distribution of a single variable in a sample. 
However, in most cases, studies require to describe several variables that are often related.
For instance, a nutritional study should consider all the variables that could be related to the weight, as height, age,
gender, smoking, diet, physic exercise, etc.

To understand a phenomenon that involve several variables is not enough to study every variable by its own. 
We have to study all the variables together to describe how they interact and the type of relation among them. 

Usually in a \emph{dependency study} there is a \highlight{dependent variable} $Y$ that it is supposed to be influenced
by a set of variables $X_1,\ldots,X_n$ known as \highlight{independent variables}. 
The simpler case is a \emph{simple dependency study} when there is only one independent variable, that is the case
covered in this chapter. 
\end{frame}


% ---------------------------------------------------------------------slide----
\begin{frame}
\frametitle{Joint frequencies}
To study the relation between two variables $X$ and $Y$, we have to study the joint distribution of the
\highlight{two-dimensional variable} $(X,Y)$, whose values are pairs $(x_i,y_j)$ where the first element is a value of
$X$ and the second a value of $Y$.

\begin{definition}[Joint sample frequencies]
Given a sample of $n$ values and a two-dimensional variable $(X,Y)$, for every value of the variable $(x_i,y_j)$
is defined
\begin{itemize}
\item \highlight{Absolute frequency $n_{ij}$}: Is the number of times that the pair $(x_i,y_j)$ appears in the sample.
\item \highlight{Relative frequency $f_{ij}$}: Is the proportion of times that the pair $(x_i,y_j)$ appears in the
sample.
\[
f_{ij}=\frac{n_{ij}}{n}
\]
\end{itemize}
\end{definition}

\begin{center}
\alert{\emph{Watch out! For two-dimensional variables it make no sense cumulative frequencies.}}
\end{center}
\end{frame}


%---------------------------------------------------------------------slide----
\begin{frame}
\frametitle{Joint frequency distribution}
The values of the two-dimensional variable with their frequencies is known as \highlight{joint frequency distribution},
and is represented in a \hightlight{joint frequency table}.

\[
\begin{array}{|c|ccccc|}
\hline
X\backslash Y & y_1 & \cdots & y_j & \cdots & y_q\\
\hline
x_1 & n_{11} & \cdots & n_{1j} & \cdots & n_{1q}\\
\vdots & \vdots & \vdots & \vdots & \vdots & \vdots\\
x_i & n_{i1} & \cdots & n_{ij} & \cdots & n_{iq}\\
\vdots & \vdots & \vdots & \vdots & \vdots & \vdots\\
x_p & n_{p1} & \cdots & n_{pj} & \cdots & n_{pq}\\
\hline
\end{array}
\]
\end{frame}


%---------------------------------------------------------------------slide----
\begin{frame}
\frametitle{Joint frequency distribution}
\framesubtitle{Example with grouped data}
The height (in cm) and weight (in kg) of a sample of 30 students is:
\begin{center}
(179,85), (173,65), (181,71), (170,65), (158,51), (174,66),\\
(172,62), (166,60), (194,90), (185,75), (162,55), (187,78),\\
(198,109), (177,61), (178,70), (165,58), (154,50), (183,93),\\
(166,51), (171,65), (175,70), (182,60), (167,59), (169,62),\\
(172,70), (186,71), (172,54), (176,68),(168,67), (187,80).
\end{center}

The joint frequency table is 
\[
\begin{array}{|c||c|c|c|c|c|c|}
\hline
  X/Y & [50,60) & [60,70) & [70,80) & [80,90) & [90,100) & [100,110) \\
  \hline\hline
  (150,160] & 2 & 0 & 0 & 0 & 0 & 0 \\
  \hline
  (160,170] & 4 & 4 & 0 & 0 & 0 & 0 \\
  \hline
  (170,180] & 1 & 6 & 3 & 1 & 0 & 0 \\
  \hline
  (180,190] & 0 & 1 & 4 & 1 & 1 & 0 \\
  \hline
  (190,200] & 0 & 0 & 0 & 0 & 1 & 1 \\
  \hline
\end{array}
\]
\end{frame}


%---------------------------------------------------------------------slide----
\begin{frame}
\frametitle{Scatter plot}
The joint frequency distribution can be represented graphically with a \highlight{scatter plot}, where data is displayed
as a collections of points on a $XY$ coordinate system.

Usually the independent variable is represented in the $X$ axis and the dependent variable in the $Y$ axis.
For every data pair $(x_i,y_j)$ in the sample a dot is drawn on the plane with those coordinates.  
\begin{center}
\tikzsetnextfilename{regression/scatterplot}
\resizebox*{!}{0.45\textheight}{% Author: Alfredo Sánchez Alberca (asalber@ceu.es)

\pgfplotsset{
    standard/.style={
        axis x line=bottom,
        axis y line=left,
        %enlarge x limits=-0.15,
        %enlarge y limits=-0.15,
        every axis x label/.style={at={(current axis.right of origin)},anchor=north west},
        every axis y label/.style={at={(current axis.above origin)},anchor=east},
    }
}

\begin{tikzpicture}
\begin{axis}[standard,ymin=0,ymax=5,axis
equal,xlabel={$X$},ylabel={$Y$},xtick={4},xticklabels={$x_i$},ytick={3},yticklabels={$y_j$}]
\addplot[color=color1,mark=*,domain=0:7] coordinates {(4,3)}; \coordinate (A) at (axis cs:4,3); \node[anchor=south west] at (A) {$(x_i,y_j)$};
\draw[dashed,color=gray] (axis cs:0,3) -- (A) -- (axis cs:4,0);
\end{axis}

\end{tikzpicture}}
\end{center}

The result is a set of points that usually is known as a \emph{point cloud}.

%A scatter plot only represent the observed values in the sample but not their frequencies.
%To represent frequencies there are othe types of charts like \emph{buble charts} or \emph{three-dimensional
% histograms}.

%\begin{center}
%\alert{\emph{¡Watch out! It make no sense for qualitative variables.}}
%\end{center}
\end{frame}


%---------------------------------------------------------------------slide----
\begin{frame}
\frametitle{Scatter plot}
\framesubtitle{Example with heights and weights}
\begin{center}
\resizebox{0.9\textwidth}{!}{% Created by tikzDevice version 0.10.1 on 2016-02-23 14:28:12
% !TEX encoding = UTF-8 Unicode
\begin{tikzpicture}[x=1pt,y=1pt]
\definecolor{fillColor}{RGB}{255,255,255}
\path[use as bounding box,fill=fillColor,fill opacity=0.00] (0,0) rectangle (505.89,361.35);
\begin{scope}
\path[clip] ( 49.20, 61.20) rectangle (480.69,312.15);
\definecolor{fillColor}{RGB}{238,50,36}

\path[fill=fillColor] (296.91,208.34) circle (  2.25);

\path[fill=fillColor] (248.96,129.57) circle (  2.25);

\path[fill=fillColor] (312.89,153.20) circle (  2.25);

\path[fill=fillColor] (224.99,129.57) circle (  2.25);

\path[fill=fillColor] (129.11, 74.43) circle (  2.25);

\path[fill=fillColor] (256.95,133.51) circle (  2.25);

\path[fill=fillColor] (240.97,117.75) circle (  2.25);

\path[fill=fillColor] (193.03,109.88) circle (  2.25);

\path[fill=fillColor] (416.77,228.03) circle (  2.25);

\path[fill=fillColor] (344.85,168.95) circle (  2.25);

\path[fill=fillColor] (161.07, 90.19) circle (  2.25);

\path[fill=fillColor] (360.83,180.77) circle (  2.25);

\path[fill=fillColor] (448.73,302.86) circle (  2.25);

\path[fill=fillColor] (280.93,113.82) circle (  2.25);

\path[fill=fillColor] (288.92,149.26) circle (  2.25);

\path[fill=fillColor] (185.04,102.00) circle (  2.25);

\path[fill=fillColor] ( 97.14, 70.49) circle (  2.25);

\path[fill=fillColor] (328.87,239.84) circle (  2.25);

\path[fill=fillColor] (193.03, 74.43) circle (  2.25);

\path[fill=fillColor] (232.98,129.57) circle (  2.25);

\path[fill=fillColor] (264.95,149.26) circle (  2.25);

\path[fill=fillColor] (320.88,109.88) circle (  2.25);

\path[fill=fillColor] (201.02,105.94) circle (  2.25);

\path[fill=fillColor] (217.00,117.75) circle (  2.25);

\path[fill=fillColor] (240.97,149.26) circle (  2.25);

\path[fill=fillColor] (352.84,153.20) circle (  2.25);

\path[fill=fillColor] (240.97, 86.25) circle (  2.25);

\path[fill=fillColor] (272.94,141.38) circle (  2.25);

\path[fill=fillColor] (209.01,137.45) circle (  2.25);

\path[fill=fillColor] (360.83,188.64) circle (  2.25);
\end{scope}
\begin{scope}
\path[clip] (  0.00,  0.00) rectangle (505.89,361.35);
\definecolor{drawColor}{RGB}{0,0,0}

\path[draw=drawColor,line width= 0.4pt,line join=round,line cap=round] ( 65.18, 61.20) -- (464.71, 61.20);

\path[draw=drawColor,line width= 0.4pt,line join=round,line cap=round] ( 65.18, 61.20) -- ( 65.18, 55.20);

\path[draw=drawColor,line width= 0.4pt,line join=round,line cap=round] (145.09, 61.20) -- (145.09, 55.20);

\path[draw=drawColor,line width= 0.4pt,line join=round,line cap=round] (224.99, 61.20) -- (224.99, 55.20);

\path[draw=drawColor,line width= 0.4pt,line join=round,line cap=round] (304.90, 61.20) -- (304.90, 55.20);

\path[draw=drawColor,line width= 0.4pt,line join=round,line cap=round] (384.80, 61.20) -- (384.80, 55.20);

\path[draw=drawColor,line width= 0.4pt,line join=round,line cap=round] (464.71, 61.20) -- (464.71, 55.20);

\node[text=drawColor,anchor=base,inner sep=0pt, outer sep=0pt, scale=  1.00] at ( 65.18, 39.60) {150};

\node[text=drawColor,anchor=base,inner sep=0pt, outer sep=0pt, scale=  1.00] at (145.09, 39.60) {160};

\node[text=drawColor,anchor=base,inner sep=0pt, outer sep=0pt, scale=  1.00] at (224.99, 39.60) {170};

\node[text=drawColor,anchor=base,inner sep=0pt, outer sep=0pt, scale=  1.00] at (304.90, 39.60) {180};

\node[text=drawColor,anchor=base,inner sep=0pt, outer sep=0pt, scale=  1.00] at (384.80, 39.60) {190};

\node[text=drawColor,anchor=base,inner sep=0pt, outer sep=0pt, scale=  1.00] at (464.71, 39.60) {200};

\path[draw=drawColor,line width= 0.4pt,line join=round,line cap=round] ( 49.20, 70.49) -- ( 49.20,306.79);

\path[draw=drawColor,line width= 0.4pt,line join=round,line cap=round] ( 49.20, 70.49) -- ( 43.20, 70.49);

\path[draw=drawColor,line width= 0.4pt,line join=round,line cap=round] ( 49.20,109.88) -- ( 43.20,109.88);

\path[draw=drawColor,line width= 0.4pt,line join=round,line cap=round] ( 49.20,149.26) -- ( 43.20,149.26);

\path[draw=drawColor,line width= 0.4pt,line join=round,line cap=round] ( 49.20,188.64) -- ( 43.20,188.64);

\path[draw=drawColor,line width= 0.4pt,line join=round,line cap=round] ( 49.20,228.03) -- ( 43.20,228.03);

\path[draw=drawColor,line width= 0.4pt,line join=round,line cap=round] ( 49.20,267.41) -- ( 43.20,267.41);

\path[draw=drawColor,line width= 0.4pt,line join=round,line cap=round] ( 49.20,306.79) -- ( 43.20,306.79);

\node[text=drawColor,rotate= 90.00,anchor=base,inner sep=0pt, outer sep=0pt, scale=  1.00] at ( 34.80, 70.49) {50};

\node[text=drawColor,rotate= 90.00,anchor=base,inner sep=0pt, outer sep=0pt, scale=  1.00] at ( 34.80,109.88) {60};

\node[text=drawColor,rotate= 90.00,anchor=base,inner sep=0pt, outer sep=0pt, scale=  1.00] at ( 34.80,149.26) {70};

\node[text=drawColor,rotate= 90.00,anchor=base,inner sep=0pt, outer sep=0pt, scale=  1.00] at ( 34.80,188.64) {80};

\node[text=drawColor,rotate= 90.00,anchor=base,inner sep=0pt, outer sep=0pt, scale=  1.00] at ( 34.80,228.03) {90};

\node[text=drawColor,rotate= 90.00,anchor=base,inner sep=0pt, outer sep=0pt, scale=  1.00] at ( 34.80,267.41) {100};

\node[text=drawColor,rotate= 90.00,anchor=base,inner sep=0pt, outer sep=0pt, scale=  1.00] at ( 34.80,306.79) {110};

\path[draw=drawColor,line width= 0.4pt,line join=round,line cap=round] ( 49.20, 61.20) --
	(480.69, 61.20) --
	(480.69,312.15) --
	( 49.20,312.15) --
	( 49.20, 61.20);
\end{scope}
\begin{scope}
\path[clip] (  0.00,  0.00) rectangle (505.89,361.35);
\definecolor{drawColor}{RGB}{0,0,0}

\node[text=drawColor,anchor=base,inner sep=0pt, outer sep=0pt, scale=  1.20] at (264.95,332.61) {\bfseries Height and weight scatter plot};

\node[text=drawColor,anchor=base,inner sep=0pt, outer sep=0pt, scale=  1.20] at (264.95, 15.60) {Height (cm)};

\node[text=drawColor,rotate= 90.00,anchor=base,inner sep=0pt, outer sep=0pt, scale=  1.20] at ( 10.80,186.67) {Weight (Kg)};
\end{scope}
\begin{scope}
\path[clip] ( 49.20, 61.20) rectangle (480.69,312.15);
\definecolor{drawColor}{RGB}{190,190,190}

\path[draw=drawColor,line width= 0.4pt,dash pattern=on 4pt off 4pt ,line join=round,line cap=round] (296.91,  0.00) -- (296.91,208.34);

\path[draw=drawColor,line width= 0.4pt,dash pattern=on 4pt off 4pt ,line join=round,line cap=round] (  0.00,208.34) -- (296.91,208.34);
\definecolor{drawColor}{RGB}{0,0,0}

\node[text=drawColor,anchor=base,inner sep=0pt, outer sep=0pt, scale=  1.00] at (296.91,213.71) {(179,85)};
\end{scope}
\end{tikzpicture}
}
\end{center}
\end{frame}


%---------------------------------------------------------------------slide----
\begin{frame}
\frametitle{Scatter plot interpretation}
The shape of the point cloud in a scatter plot gives information about the type of relation between the variables.

\begin{center}
\resizebox{\textwidth}{!}{
\begin{tabular}{ccc}
% Created by tikzDevice version 0.9 on 2016-02-26 12:45:51
% !TEX encoding = UTF-8 Unicode
\begin{tikzpicture}[x=1pt,y=1pt]
\definecolor{fillColor}{RGB}{255,255,255}
\path[use as bounding box,fill=fillColor,fill opacity=0.00] (0,0) rectangle (505.89,361.35);
\begin{scope}
\path[clip] ( 36.13, 36.13) rectangle (469.75,325.21);
\definecolor{fillColor}{RGB}{5,161,230}

\path[fill=fillColor] ( 86.49,193.21) circle (  2.25);

\path[fill=fillColor] (235.58,136.44) circle (  2.25);

\path[fill=fillColor] ( 93.17,202.27) circle (  2.25);

\path[fill=fillColor] (304.04,120.24) circle (  2.25);

\path[fill=fillColor] (215.32,175.95) circle (  2.25);

\path[fill=fillColor] ( 89.82, 64.27) circle (  2.25);

\path[fill=fillColor] (147.05,104.47) circle (  2.25);

\path[fill=fillColor] (424.36, 63.93) circle (  2.25);

\path[fill=fillColor] ( 52.20,157.06) circle (  2.25);

\path[fill=fillColor] (394.16, 58.06) circle (  2.25);

\path[fill=fillColor] (235.83,288.31) circle (  2.25);

\path[fill=fillColor] (158.10,242.64) circle (  2.25);

\path[fill=fillColor] (371.50,232.48) circle (  2.25);

\path[fill=fillColor] (270.88,152.40) circle (  2.25);

\path[fill=fillColor] (282.49,178.10) circle (  2.25);

\path[fill=fillColor] (263.04,231.92) circle (  2.25);

\path[fill=fillColor] (269.39,140.00) circle (  2.25);

\path[fill=fillColor] (114.98,239.67) circle (  2.25);

\path[fill=fillColor] (422.11,144.22) circle (  2.25);

\path[fill=fillColor] (369.76,297.53) circle (  2.25);

\path[fill=fillColor] ( 76.61,165.93) circle (  2.25);

\path[fill=fillColor] ( 60.86, 46.84) circle (  2.25);

\path[fill=fillColor] (354.36,116.00) circle (  2.25);

\path[fill=fillColor] (350.78, 97.11) circle (  2.25);

\path[fill=fillColor] (379.65, 81.61) circle (  2.25);

\path[fill=fillColor] (399.41,158.38) circle (  2.25);

\path[fill=fillColor] (324.25,287.45) circle (  2.25);

\path[fill=fillColor] ( 89.27,226.92) circle (  2.25);

\path[fill=fillColor] (440.93,126.12) circle (  2.25);

\path[fill=fillColor] (175.80, 82.68) circle (  2.25);

\path[fill=fillColor] (293.21, 61.79) circle (  2.25);

\path[fill=fillColor] (395.95, 57.74) circle (  2.25);

\path[fill=fillColor] (123.84,143.64) circle (  2.25);

\path[fill=fillColor] (117.99, 98.47) circle (  2.25);

\path[fill=fillColor] (159.70,116.17) circle (  2.25);

\path[fill=fillColor] ( 71.80,194.69) circle (  2.25);

\path[fill=fillColor] ( 70.14,305.90) circle (  2.25);

\path[fill=fillColor] (360.51,259.22) circle (  2.25);

\path[fill=fillColor] (140.52, 91.40) circle (  2.25);

\path[fill=fillColor] (151.16, 54.18) circle (  2.25);

\path[fill=fillColor] (447.46,241.50) circle (  2.25);

\path[fill=fillColor] (136.69,254.20) circle (  2.25);

\path[fill=fillColor] (106.22,182.55) circle (  2.25);

\path[fill=fillColor] (106.06,294.18) circle (  2.25);

\path[fill=fillColor] (121.05,183.11) circle (  2.25);

\path[fill=fillColor] (446.82,285.83) circle (  2.25);

\path[fill=fillColor] (228.89, 76.04) circle (  2.25);

\path[fill=fillColor] (321.27, 99.41) circle (  2.25);

\path[fill=fillColor] (197.11,213.49) circle (  2.25);

\path[fill=fillColor] (218.93,189.81) circle (  2.25);

\path[fill=fillColor] (307.53,159.68) circle (  2.25);

\path[fill=fillColor] (140.62,226.88) circle (  2.25);

\path[fill=fillColor] (342.01,232.47) circle (  2.25);

\path[fill=fillColor] (404.14,245.51) circle (  2.25);

\path[fill=fillColor] (101.45,179.22) circle (  2.25);

\path[fill=fillColor] (253.90,295.23) circle (  2.25);

\path[fill=fillColor] ( 97.34,246.82) circle (  2.25);

\path[fill=fillColor] ( 93.95,137.60) circle (  2.25);

\path[fill=fillColor] (453.70,309.05) circle (  2.25);

\path[fill=fillColor] (336.58,188.49) circle (  2.25);

\path[fill=fillColor] (223.40,295.74) circle (  2.25);

\path[fill=fillColor] ( 80.63,172.82) circle (  2.25);

\path[fill=fillColor] (116.71,291.05) circle (  2.25);

\path[fill=fillColor] (311.83,255.02) circle (  2.25);

\path[fill=fillColor] ( 59.38,181.87) circle (  2.25);

\path[fill=fillColor] (172.09,155.72) circle (  2.25);

\path[fill=fillColor] (194.99,142.09) circle (  2.25);

\path[fill=fillColor] ( 71.87,121.47) circle (  2.25);

\path[fill=fillColor] (213.03,301.14) circle (  2.25);

\path[fill=fillColor] ( 72.36,181.91) circle (  2.25);

\path[fill=fillColor] ( 57.10,217.10) circle (  2.25);

\path[fill=fillColor] ( 95.86,224.66) circle (  2.25);

\path[fill=fillColor] (320.75,225.87) circle (  2.25);

\path[fill=fillColor] (292.82, 79.52) circle (  2.25);

\path[fill=fillColor] (296.84,310.23) circle (  2.25);

\path[fill=fillColor] (197.74,274.87) circle (  2.25);

\path[fill=fillColor] (325.08, 52.39) circle (  2.25);

\path[fill=fillColor] (187.54,237.22) circle (  2.25);

\path[fill=fillColor] ( 65.66, 96.71) circle (  2.25);

\path[fill=fillColor] (233.98,222.05) circle (  2.25);

\path[fill=fillColor] (100.08,286.21) circle (  2.25);

\path[fill=fillColor] (172.58,184.00) circle (  2.25);

\path[fill=fillColor] ( 65.45,278.76) circle (  2.25);

\path[fill=fillColor] (311.76,130.20) circle (  2.25);

\path[fill=fillColor] (430.49,243.50) circle (  2.25);

\path[fill=fillColor] ( 72.43,289.99) circle (  2.25);

\path[fill=fillColor] (220.14,216.00) circle (  2.25);

\path[fill=fillColor] (318.77,314.51) circle (  2.25);

\path[fill=fillColor] (396.79,306.57) circle (  2.25);

\path[fill=fillColor] ( 55.06,287.93) circle (  2.25);

\path[fill=fillColor] (364.51,152.39) circle (  2.25);

\path[fill=fillColor] (383.07,304.84) circle (  2.25);

\path[fill=fillColor] (214.81,282.85) circle (  2.25);

\path[fill=fillColor] (449.54,257.30) circle (  2.25);

\path[fill=fillColor] (426.48,224.40) circle (  2.25);

\path[fill=fillColor] (211.04,272.24) circle (  2.25);

\path[fill=fillColor] (317.69,200.08) circle (  2.25);

\path[fill=fillColor] (392.55,196.82) circle (  2.25);

\path[fill=fillColor] (366.29,256.37) circle (  2.25);

\path[fill=fillColor] ( 97.99,210.22) circle (  2.25);
\end{scope}
\begin{scope}
\path[clip] (  0.00,  0.00) rectangle (505.89,361.35);
\definecolor{drawColor}{RGB}{0,0,0}

\node[text=drawColor,anchor=base,inner sep=0pt, outer sep=0pt, scale=  1.20] at (252.94,339.14) {\bfseries No relation};

\node[text=drawColor,anchor=base west,inner sep=0pt, outer sep=0pt, scale=  1.20] at (248.49, 14.53) {\itshape X};

\node[text=drawColor,rotate= 90.00,anchor=base west,inner sep=0pt, outer sep=0pt, scale=  1.20] at ( 21.73,176.22) {\itshape Y};
\end{scope}
\begin{scope}
\path[clip] (  0.00,  0.00) rectangle (505.89,361.35);
\definecolor{drawColor}{RGB}{0,0,0}

\path[draw=drawColor,line width= 0.4pt,line join=round,line cap=round] ( 36.13, 36.13) --
	(469.75, 36.13) --
	(469.75,325.21) --
	( 36.13,325.21) --
	( 36.13, 36.13);
\end{scope}
\end{tikzpicture}
 &
% Created by tikzDevice version 0.9 on 2016-02-26 12:45:52
% !TEX encoding = UTF-8 Unicode
\begin{tikzpicture}[x=1pt,y=1pt]
\definecolor{fillColor}{RGB}{255,255,255}
\path[use as bounding box,fill=fillColor,fill opacity=0.00] (0,0) rectangle (505.89,361.35);
\begin{scope}
\path[clip] ( 36.13, 36.13) rectangle (469.75,325.21);
\definecolor{fillColor}{RGB}{5,161,230}

\path[fill=fillColor] ( 86.49, 46.84) circle (  2.25);

\path[fill=fillColor] (235.58,185.35) circle (  2.25);

\path[fill=fillColor] ( 93.17, 62.53) circle (  2.25);

\path[fill=fillColor] (304.04,227.88) circle (  2.25);

\path[fill=fillColor] (215.32,131.55) circle (  2.25);

\path[fill=fillColor] ( 89.82,121.23) circle (  2.25);

\path[fill=fillColor] (147.05, 95.62) circle (  2.25);

\path[fill=fillColor] (424.36,279.10) circle (  2.25);

\path[fill=fillColor] ( 52.20, 58.31) circle (  2.25);

\path[fill=fillColor] (394.16,260.58) circle (  2.25);

\path[fill=fillColor] (235.83,181.72) circle (  2.25);

\path[fill=fillColor] (158.10,107.23) circle (  2.25);

\path[fill=fillColor] (371.50,242.60) circle (  2.25);

\path[fill=fillColor] (270.88,181.47) circle (  2.25);

\path[fill=fillColor] (282.49,196.65) circle (  2.25);

\path[fill=fillColor] (263.04,141.41) circle (  2.25);

\path[fill=fillColor] (269.39,213.99) circle (  2.25);

\path[fill=fillColor] (114.98, 64.12) circle (  2.25);

\path[fill=fillColor] (422.11,285.77) circle (  2.25);

\path[fill=fillColor] (369.76,247.46) circle (  2.25);

\path[fill=fillColor] ( 76.61, 91.06) circle (  2.25);

\path[fill=fillColor] ( 60.86, 66.42) circle (  2.25);

\path[fill=fillColor] (354.36,235.74) circle (  2.25);

\path[fill=fillColor] (350.78,188.80) circle (  2.25);

\path[fill=fillColor] (379.65,226.79) circle (  2.25);

\path[fill=fillColor] (399.41,253.36) circle (  2.25);

\path[fill=fillColor] (324.25,206.06) circle (  2.25);

\path[fill=fillColor] ( 89.27, 91.59) circle (  2.25);

\path[fill=fillColor] (440.93,314.51) circle (  2.25);

\path[fill=fillColor] (175.80,140.72) circle (  2.25);

\path[fill=fillColor] (293.21,196.97) circle (  2.25);

\path[fill=fillColor] (395.95,202.87) circle (  2.25);

\path[fill=fillColor] (123.84, 81.12) circle (  2.25);

\path[fill=fillColor] (117.99,135.22) circle (  2.25);

\path[fill=fillColor] (159.70,139.19) circle (  2.25);

\path[fill=fillColor] ( 71.80, 87.62) circle (  2.25);

\path[fill=fillColor] ( 70.14, 75.41) circle (  2.25);

\path[fill=fillColor] (360.51,252.97) circle (  2.25);

\path[fill=fillColor] (140.52, 73.79) circle (  2.25);

\path[fill=fillColor] (151.16,120.65) circle (  2.25);

\path[fill=fillColor] (447.46,276.48) circle (  2.25);

\path[fill=fillColor] (136.69,145.75) circle (  2.25);

\path[fill=fillColor] (106.22, 63.81) circle (  2.25);

\path[fill=fillColor] (106.06, 99.18) circle (  2.25);

\path[fill=fillColor] (121.05, 91.75) circle (  2.25);

\path[fill=fillColor] (446.82,251.90) circle (  2.25);

\path[fill=fillColor] (228.89,211.71) circle (  2.25);

\path[fill=fillColor] (321.27,197.86) circle (  2.25);

\path[fill=fillColor] (197.11,156.78) circle (  2.25);

\path[fill=fillColor] (218.93,175.40) circle (  2.25);

\path[fill=fillColor] (307.53,166.45) circle (  2.25);

\path[fill=fillColor] (140.62,124.66) circle (  2.25);

\path[fill=fillColor] (342.01,179.43) circle (  2.25);

\path[fill=fillColor] (404.14,288.22) circle (  2.25);

\path[fill=fillColor] (101.45, 87.38) circle (  2.25);

\path[fill=fillColor] (253.90,167.32) circle (  2.25);

\path[fill=fillColor] ( 97.34,105.66) circle (  2.25);

\path[fill=fillColor] ( 93.95,125.18) circle (  2.25);

\path[fill=fillColor] (453.70,261.46) circle (  2.25);

\path[fill=fillColor] (336.58,178.53) circle (  2.25);

\path[fill=fillColor] (223.40,168.13) circle (  2.25);

\path[fill=fillColor] ( 80.63, 56.05) circle (  2.25);

\path[fill=fillColor] (116.71, 79.15) circle (  2.25);

\path[fill=fillColor] (311.83,165.99) circle (  2.25);

\path[fill=fillColor] ( 59.38, 55.35) circle (  2.25);

\path[fill=fillColor] (172.09, 56.29) circle (  2.25);

\path[fill=fillColor] (194.99,118.94) circle (  2.25);

\path[fill=fillColor] ( 71.87, 72.94) circle (  2.25);

\path[fill=fillColor] (213.03,112.81) circle (  2.25);

\path[fill=fillColor] ( 72.36, 58.33) circle (  2.25);

\path[fill=fillColor] ( 57.10, 79.40) circle (  2.25);

\path[fill=fillColor] ( 95.86, 64.52) circle (  2.25);

\path[fill=fillColor] (320.75,216.90) circle (  2.25);

\path[fill=fillColor] (292.82,169.05) circle (  2.25);

\path[fill=fillColor] (296.84,196.41) circle (  2.25);

\path[fill=fillColor] (197.74,144.19) circle (  2.25);

\path[fill=fillColor] (325.08,226.22) circle (  2.25);

\path[fill=fillColor] (187.54,131.24) circle (  2.25);

\path[fill=fillColor] ( 65.66, 97.17) circle (  2.25);

\path[fill=fillColor] (233.98,135.15) circle (  2.25);

\path[fill=fillColor] (100.08, 49.83) circle (  2.25);

\path[fill=fillColor] (172.58, 81.62) circle (  2.25);

\path[fill=fillColor] ( 65.45, 71.70) circle (  2.25);

\path[fill=fillColor] (311.76,213.49) circle (  2.25);

\path[fill=fillColor] (430.49,293.39) circle (  2.25);

\path[fill=fillColor] ( 72.43, 77.03) circle (  2.25);

\path[fill=fillColor] (220.14,165.64) circle (  2.25);

\path[fill=fillColor] (318.77,172.85) circle (  2.25);

\path[fill=fillColor] (396.79,259.35) circle (  2.25);

\path[fill=fillColor] ( 55.06, 67.05) circle (  2.25);

\path[fill=fillColor] (364.51,231.15) circle (  2.25);

\path[fill=fillColor] (383.07,219.19) circle (  2.25);

\path[fill=fillColor] (214.81,133.84) circle (  2.25);

\path[fill=fillColor] (449.54,298.85) circle (  2.25);

\path[fill=fillColor] (426.48,276.62) circle (  2.25);

\path[fill=fillColor] (211.04,166.12) circle (  2.25);

\path[fill=fillColor] (317.69,179.21) circle (  2.25);

\path[fill=fillColor] (392.55,207.14) circle (  2.25);

\path[fill=fillColor] (366.29,208.14) circle (  2.25);

\path[fill=fillColor] ( 97.99, 76.20) circle (  2.25);
\end{scope}
\begin{scope}
\path[clip] (  0.00,  0.00) rectangle (505.89,361.35);
\definecolor{drawColor}{RGB}{0,0,0}

\node[text=drawColor,anchor=base,inner sep=0pt, outer sep=0pt, scale=  1.20] at (252.94,339.14) {\bfseries Linear relation};

\node[text=drawColor,anchor=base west,inner sep=0pt, outer sep=0pt, scale=  1.20] at (248.49, 14.53) {\itshape X};

\node[text=drawColor,rotate= 90.00,anchor=base west,inner sep=0pt, outer sep=0pt, scale=  1.20] at ( 21.73,176.22) {\itshape Y};
\end{scope}
\begin{scope}
\path[clip] (  0.00,  0.00) rectangle (505.89,361.35);
\definecolor{drawColor}{RGB}{0,0,0}

\path[draw=drawColor,line width= 0.4pt,line join=round,line cap=round] ( 36.13, 36.13) --
	(469.75, 36.13) --
	(469.75,325.21) --
	( 36.13,325.21) --
	( 36.13, 36.13);
\end{scope}
\end{tikzpicture}
 &
\input{img/regression/cuadratic_scatterplot} \\
% Created by tikzDevice version 0.9 on 2016-02-26 12:45:52
% !TEX encoding = UTF-8 Unicode
\begin{tikzpicture}[x=1pt,y=1pt]
\definecolor{fillColor}{RGB}{255,255,255}
\path[use as bounding box,fill=fillColor,fill opacity=0.00] (0,0) rectangle (505.89,361.35);
\begin{scope}
\path[clip] ( 36.13, 36.13) rectangle (469.75,325.21);
\definecolor{fillColor}{RGB}{5,161,230}

\path[fill=fillColor] ( 86.49, 77.57) circle (  2.25);

\path[fill=fillColor] (235.58, 75.69) circle (  2.25);

\path[fill=fillColor] ( 93.17, 74.14) circle (  2.25);

\path[fill=fillColor] (304.04, 73.27) circle (  2.25);

\path[fill=fillColor] (215.32, 94.01) circle (  2.25);

\path[fill=fillColor] ( 89.82, 63.05) circle (  2.25);

\path[fill=fillColor] (147.05, 77.97) circle (  2.25);

\path[fill=fillColor] (424.36,198.39) circle (  2.25);

\path[fill=fillColor] ( 52.20, 91.96) circle (  2.25);

\path[fill=fillColor] (394.16,148.14) circle (  2.25);

\path[fill=fillColor] (235.83, 94.17) circle (  2.25);

\path[fill=fillColor] (158.10, 70.24) circle (  2.25);

\path[fill=fillColor] (371.50,150.10) circle (  2.25);

\path[fill=fillColor] (270.88, 98.23) circle (  2.25);

\path[fill=fillColor] (282.49,101.36) circle (  2.25);

\path[fill=fillColor] (263.04, 89.90) circle (  2.25);

\path[fill=fillColor] (269.39, 71.57) circle (  2.25);

\path[fill=fillColor] (114.98, 87.75) circle (  2.25);

\path[fill=fillColor] (422.11,214.35) circle (  2.25);

\path[fill=fillColor] (369.76,150.97) circle (  2.25);

\path[fill=fillColor] ( 76.61, 54.55) circle (  2.25);

\path[fill=fillColor] ( 60.86, 61.72) circle (  2.25);

\path[fill=fillColor] (354.36,134.38) circle (  2.25);

\path[fill=fillColor] (350.78,135.57) circle (  2.25);

\path[fill=fillColor] (379.65,139.19) circle (  2.25);

\path[fill=fillColor] (399.41,177.13) circle (  2.25);

\path[fill=fillColor] (324.25,104.69) circle (  2.25);

\path[fill=fillColor] ( 89.27, 61.99) circle (  2.25);

\path[fill=fillColor] (440.93,223.94) circle (  2.25);

\path[fill=fillColor] (175.80, 72.63) circle (  2.25);

\path[fill=fillColor] (293.21, 80.70) circle (  2.25);

\path[fill=fillColor] (395.95,156.73) circle (  2.25);

\path[fill=fillColor] (123.84, 81.49) circle (  2.25);

\path[fill=fillColor] (117.99, 66.73) circle (  2.25);

\path[fill=fillColor] (159.70, 77.20) circle (  2.25);

\path[fill=fillColor] ( 71.80, 73.37) circle (  2.25);

\path[fill=fillColor] ( 70.14, 82.13) circle (  2.25);

\path[fill=fillColor] (360.51,128.43) circle (  2.25);

\path[fill=fillColor] (140.52, 81.28) circle (  2.25);

\path[fill=fillColor] (151.16, 70.82) circle (  2.25);

\path[fill=fillColor] (447.46,249.83) circle (  2.25);

\path[fill=fillColor] (136.69, 71.22) circle (  2.25);

\path[fill=fillColor] (106.22, 64.49) circle (  2.25);

\path[fill=fillColor] (106.06, 66.43) circle (  2.25);

\path[fill=fillColor] (121.05, 72.43) circle (  2.25);

\path[fill=fillColor] (446.82,267.73) circle (  2.25);

\path[fill=fillColor] (228.89, 67.22) circle (  2.25);

\path[fill=fillColor] (321.27,110.09) circle (  2.25);

\path[fill=fillColor] (197.11, 85.84) circle (  2.25);

\path[fill=fillColor] (218.93,108.92) circle (  2.25);

\path[fill=fillColor] (307.53, 92.22) circle (  2.25);

\path[fill=fillColor] (140.62, 71.38) circle (  2.25);

\path[fill=fillColor] (342.01,111.85) circle (  2.25);

\path[fill=fillColor] (404.14,192.62) circle (  2.25);

\path[fill=fillColor] (101.45, 56.16) circle (  2.25);

\path[fill=fillColor] (253.90, 94.16) circle (  2.25);

\path[fill=fillColor] ( 97.34, 46.84) circle (  2.25);

\path[fill=fillColor] ( 93.95, 64.06) circle (  2.25);

\path[fill=fillColor] (453.70,314.51) circle (  2.25);

\path[fill=fillColor] (336.58,105.46) circle (  2.25);

\path[fill=fillColor] (223.40, 82.75) circle (  2.25);

\path[fill=fillColor] ( 80.63, 81.31) circle (  2.25);

\path[fill=fillColor] (116.71, 81.90) circle (  2.25);

\path[fill=fillColor] (311.83,108.98) circle (  2.25);

\path[fill=fillColor] ( 59.38, 60.16) circle (  2.25);

\path[fill=fillColor] (172.09, 66.70) circle (  2.25);

\path[fill=fillColor] (194.99, 90.72) circle (  2.25);

\path[fill=fillColor] ( 71.87, 62.15) circle (  2.25);

\path[fill=fillColor] (213.03, 74.20) circle (  2.25);

\path[fill=fillColor] ( 72.36, 56.29) circle (  2.25);

\path[fill=fillColor] ( 57.10, 91.26) circle (  2.25);

\path[fill=fillColor] ( 95.86, 76.86) circle (  2.25);

\path[fill=fillColor] (320.75, 76.83) circle (  2.25);

\path[fill=fillColor] (292.82, 81.11) circle (  2.25);

\path[fill=fillColor] (296.84, 90.23) circle (  2.25);

\path[fill=fillColor] (197.74, 74.57) circle (  2.25);

\path[fill=fillColor] (325.08,107.10) circle (  2.25);

\path[fill=fillColor] (187.54, 64.32) circle (  2.25);

\path[fill=fillColor] ( 65.66, 75.04) circle (  2.25);

\path[fill=fillColor] (233.98, 60.57) circle (  2.25);

\path[fill=fillColor] (100.08, 89.95) circle (  2.25);

\path[fill=fillColor] (172.58, 75.15) circle (  2.25);

\path[fill=fillColor] ( 65.45, 79.00) circle (  2.25);

\path[fill=fillColor] (311.76, 96.42) circle (  2.25);

\path[fill=fillColor] (430.49,202.87) circle (  2.25);

\path[fill=fillColor] ( 72.43, 79.03) circle (  2.25);

\path[fill=fillColor] (220.14, 68.28) circle (  2.25);

\path[fill=fillColor] (318.77,101.53) circle (  2.25);

\path[fill=fillColor] (396.79,150.46) circle (  2.25);

\path[fill=fillColor] ( 55.06, 72.41) circle (  2.25);

\path[fill=fillColor] (364.51,138.99) circle (  2.25);

\path[fill=fillColor] (383.07,177.91) circle (  2.25);

\path[fill=fillColor] (214.81, 85.91) circle (  2.25);

\path[fill=fillColor] (449.54,289.95) circle (  2.25);

\path[fill=fillColor] (426.48,191.78) circle (  2.25);

\path[fill=fillColor] (211.04, 78.28) circle (  2.25);

\path[fill=fillColor] (317.69, 97.55) circle (  2.25);

\path[fill=fillColor] (392.55,149.63) circle (  2.25);

\path[fill=fillColor] (366.29,131.20) circle (  2.25);

\path[fill=fillColor] ( 97.99, 54.40) circle (  2.25);
\end{scope}
\begin{scope}
\path[clip] (  0.00,  0.00) rectangle (505.89,361.35);
\definecolor{drawColor}{RGB}{0,0,0}

\node[text=drawColor,anchor=base,inner sep=0pt, outer sep=0pt, scale=  1.20] at (252.94,339.14) {\bfseries Exponential relation};

\node[text=drawColor,anchor=base west,inner sep=0pt, outer sep=0pt, scale=  1.20] at (248.49, 14.53) {\itshape X};

\node[text=drawColor,rotate= 90.00,anchor=base west,inner sep=0pt, outer sep=0pt, scale=  1.20] at ( 21.73,176.22) {\itshape Y};
\end{scope}
\begin{scope}
\path[clip] (  0.00,  0.00) rectangle (505.89,361.35);
\definecolor{drawColor}{RGB}{0,0,0}

\path[draw=drawColor,line width= 0.4pt,line join=round,line cap=round] ( 36.13, 36.13) --
	(469.75, 36.13) --
	(469.75,325.21) --
	( 36.13,325.21) --
	( 36.13, 36.13);
\end{scope}
\end{tikzpicture}
 &
% Created by tikzDevice version 0.9 on 2016-02-26 12:45:52
% !TEX encoding = UTF-8 Unicode
\begin{tikzpicture}[x=1pt,y=1pt]
\definecolor{fillColor}{RGB}{255,255,255}
\path[use as bounding box,fill=fillColor,fill opacity=0.00] (0,0) rectangle (505.89,361.35);
\begin{scope}
\path[clip] ( 36.13, 36.13) rectangle (469.75,325.21);
\definecolor{fillColor}{RGB}{5,161,230}

\path[fill=fillColor] (103.07,156.18) circle (  2.25);

\path[fill=fillColor] ( 77.81,148.50) circle (  2.25);

\path[fill=fillColor] ( 97.12,132.22) circle (  2.25);

\path[fill=fillColor] (236.95,259.09) circle (  2.25);

\path[fill=fillColor] ( 65.50,125.56) circle (  2.25);

\path[fill=fillColor] (163.28,211.03) circle (  2.25);

\path[fill=fillColor] (279.91,272.75) circle (  2.25);

\path[fill=fillColor] (319.44,273.04) circle (  2.25);

\path[fill=fillColor] (198.47,233.87) circle (  2.25);

\path[fill=fillColor] (400.08,297.07) circle (  2.25);

\path[fill=fillColor] (238.93,253.05) circle (  2.25);

\path[fill=fillColor] (135.96,193.58) circle (  2.25);

\path[fill=fillColor] (241.67,244.69) circle (  2.25);

\path[fill=fillColor] (440.28,314.51) circle (  2.25);

\path[fill=fillColor] (300.70,274.40) circle (  2.25);

\path[fill=fillColor] (159.70,205.81) circle (  2.25);

\path[fill=fillColor] ( 78.52,123.33) circle (  2.25);

\path[fill=fillColor] (201.78,233.37) circle (  2.25);

\path[fill=fillColor] ( 71.08,133.17) circle (  2.25);

\path[fill=fillColor] (403.87,303.14) circle (  2.25);

\path[fill=fillColor] (293.83,263.12) circle (  2.25);

\path[fill=fillColor] ( 82.22,125.22) circle (  2.25);

\path[fill=fillColor] ( 57.01,106.09) circle (  2.25);

\path[fill=fillColor] (351.67,296.29) circle (  2.25);

\path[fill=fillColor] (380.18,296.41) circle (  2.25);

\path[fill=fillColor] (285.69,271.53) circle (  2.25);

\path[fill=fillColor] ( 76.60,141.12) circle (  2.25);

\path[fill=fillColor] (140.06,211.02) circle (  2.25);

\path[fill=fillColor] (213.18,236.85) circle (  2.25);

\path[fill=fillColor] (234.39,253.68) circle (  2.25);

\path[fill=fillColor] (373.76,294.19) circle (  2.25);

\path[fill=fillColor] (172.42,217.93) circle (  2.25);

\path[fill=fillColor] ( 68.93,110.48) circle (  2.25);

\path[fill=fillColor] (405.02,301.86) circle (  2.25);

\path[fill=fillColor] (230.92,246.20) circle (  2.25);

\path[fill=fillColor] (162.65,215.49) circle (  2.25);

\path[fill=fillColor] (228.99,240.56) circle (  2.25);

\path[fill=fillColor] (234.18,260.72) circle (  2.25);

\path[fill=fillColor] (125.09,192.00) circle (  2.25);

\path[fill=fillColor] (201.45,227.87) circle (  2.25);

\path[fill=fillColor] (368.18,286.12) circle (  2.25);

\path[fill=fillColor] ( 53.91, 46.84) circle (  2.25);

\path[fill=fillColor] (389.62,287.33) circle (  2.25);

\path[fill=fillColor] (134.66,200.38) circle (  2.25);

\path[fill=fillColor] (116.16,193.07) circle (  2.25);

\path[fill=fillColor] (339.59,294.21) circle (  2.25);

\path[fill=fillColor] (284.26,277.86) circle (  2.25);

\path[fill=fillColor] ( 73.21,116.82) circle (  2.25);

\path[fill=fillColor] (375.85,294.54) circle (  2.25);

\path[fill=fillColor] (394.94,310.50) circle (  2.25);

\path[fill=fillColor] (377.09,291.77) circle (  2.25);

\path[fill=fillColor] (110.68,188.50) circle (  2.25);

\path[fill=fillColor] (109.51,167.49) circle (  2.25);

\path[fill=fillColor] ( 72.35,123.78) circle (  2.25);

\path[fill=fillColor] (167.76,204.69) circle (  2.25);

\path[fill=fillColor] (399.58,292.73) circle (  2.25);

\path[fill=fillColor] (279.41,278.36) circle (  2.25);

\path[fill=fillColor] (327.97,275.98) circle (  2.25);

\path[fill=fillColor] (357.87,287.31) circle (  2.25);

\path[fill=fillColor] (424.36,314.46) circle (  2.25);

\path[fill=fillColor] ( 98.08,182.13) circle (  2.25);

\path[fill=fillColor] ( 85.04,136.92) circle (  2.25);

\path[fill=fillColor] (453.70,301.35) circle (  2.25);

\path[fill=fillColor] (267.40,264.04) circle (  2.25);

\path[fill=fillColor] (136.22,193.04) circle (  2.25);

\path[fill=fillColor] (175.08,209.40) circle (  2.25);

\path[fill=fillColor] (285.08,283.12) circle (  2.25);

\path[fill=fillColor] (380.38,301.41) circle (  2.25);

\path[fill=fillColor] (135.04,189.53) circle (  2.25);

\path[fill=fillColor] (352.79,299.10) circle (  2.25);

\path[fill=fillColor] (288.58,272.12) circle (  2.25);

\path[fill=fillColor] (412.38,301.58) circle (  2.25);

\path[fill=fillColor] (436.67,313.81) circle (  2.25);

\path[fill=fillColor] (326.18,280.63) circle (  2.25);

\path[fill=fillColor] (433.69,305.65) circle (  2.25);

\path[fill=fillColor] ( 52.20,128.41) circle (  2.25);

\path[fill=fillColor] (102.04,165.85) circle (  2.25);

\path[fill=fillColor] (228.13,244.99) circle (  2.25);

\path[fill=fillColor] ( 68.68,108.65) circle (  2.25);

\path[fill=fillColor] (319.85,280.01) circle (  2.25);

\path[fill=fillColor] (197.51,241.80) circle (  2.25);

\path[fill=fillColor] (249.57,254.83) circle (  2.25);

\path[fill=fillColor] (402.39,305.31) circle (  2.25);

\path[fill=fillColor] (309.95,265.15) circle (  2.25);

\path[fill=fillColor] (334.73,279.41) circle (  2.25);

\path[fill=fillColor] (431.07,303.37) circle (  2.25);

\path[fill=fillColor] (402.61,310.08) circle (  2.25);

\path[fill=fillColor] (302.07,288.25) circle (  2.25);

\path[fill=fillColor] (119.41,167.20) circle (  2.25);

\path[fill=fillColor] (346.69,295.67) circle (  2.25);

\path[fill=fillColor] (119.96,189.58) circle (  2.25);

\path[fill=fillColor] (445.14,309.44) circle (  2.25);

\path[fill=fillColor] (165.67,223.47) circle (  2.25);

\path[fill=fillColor] ( 79.87,140.87) circle (  2.25);

\path[fill=fillColor] (287.17,264.86) circle (  2.25);

\path[fill=fillColor] (195.97,224.57) circle (  2.25);

\path[fill=fillColor] (306.50,282.99) circle (  2.25);

\path[fill=fillColor] (181.85,216.49) circle (  2.25);

\path[fill=fillColor] (246.46,260.62) circle (  2.25);

\path[fill=fillColor] (167.89,210.52) circle (  2.25);
\end{scope}
\begin{scope}
\path[clip] (  0.00,  0.00) rectangle (505.89,361.35);
\definecolor{drawColor}{RGB}{0,0,0}

\node[text=drawColor,anchor=base,inner sep=0pt, outer sep=0pt, scale=  1.20] at (252.94,339.14) {\bfseries Logarithmic relation};

\node[text=drawColor,anchor=base west,inner sep=0pt, outer sep=0pt, scale=  1.20] at (248.49, 14.53) {\itshape X};

\node[text=drawColor,rotate= 90.00,anchor=base west,inner sep=0pt, outer sep=0pt, scale=  1.20] at ( 21.73,176.22) {\itshape Y};
\end{scope}
\begin{scope}
\path[clip] (  0.00,  0.00) rectangle (505.89,361.35);
\definecolor{drawColor}{RGB}{0,0,0}

\path[draw=drawColor,line width= 0.4pt,line join=round,line cap=round] ( 36.13, 36.13) --
	(469.75, 36.13) --
	(469.75,325.21) --
	( 36.13,325.21) --
	( 36.13, 36.13);
\end{scope}
\end{tikzpicture}
 &
% Created by tikzDevice version 0.10.1 on 2016-02-26 19:01:20
% !TEX encoding = UTF-8 Unicode
\begin{tikzpicture}[x=1pt,y=1pt]
\definecolor{fillColor}{RGB}{255,255,255}
\path[use as bounding box,fill=fillColor,fill opacity=0.00] (0,0) rectangle (505.89,361.35);
\begin{scope}
\path[clip] ( 36.13, 36.13) rectangle (469.75,325.21);
\definecolor{fillColor}{RGB}{5,161,230}

\path[fill=fillColor] (358.48, 70.03) circle (  2.25);

\path[fill=fillColor] (427.24, 78.35) circle (  2.25);

\path[fill=fillColor] ( 91.32,177.39) circle (  2.25);

\path[fill=fillColor] (237.36,117.28) circle (  2.25);

\path[fill=fillColor] (152.14,135.47) circle (  2.25);

\path[fill=fillColor] (219.23, 81.40) circle (  2.25);

\path[fill=fillColor] (306.82, 93.39) circle (  2.25);

\path[fill=fillColor] (155.21,136.62) circle (  2.25);

\path[fill=fillColor] ( 56.22,272.58) circle (  2.25);

\path[fill=fillColor] (180.84,107.36) circle (  2.25);

\path[fill=fillColor] ( 52.19,244.55) circle (  2.25);

\path[fill=fillColor] (379.82, 85.53) circle (  2.25);

\path[fill=fillColor] (256.90, 87.77) circle (  2.25);

\path[fill=fillColor] (355.48,129.56) circle (  2.25);

\path[fill=fillColor] (427.18, 86.05) circle (  2.25);

\path[fill=fillColor] (252.38,110.96) circle (  2.25);

\path[fill=fillColor] ( 91.11,192.07) circle (  2.25);

\path[fill=fillColor] ( 75.21,270.49) circle (  2.25);

\path[fill=fillColor] (329.04, 95.99) circle (  2.25);

\path[fill=fillColor] (406.76, 87.06) circle (  2.25);

\path[fill=fillColor] (171.97,123.31) circle (  2.25);

\path[fill=fillColor] (258.94, 87.83) circle (  2.25);

\path[fill=fillColor] ( 88.58,162.57) circle (  2.25);

\path[fill=fillColor] (320.06, 68.64) circle (  2.25);

\path[fill=fillColor] (311.86, 87.25) circle (  2.25);

\path[fill=fillColor] (343.35,128.96) circle (  2.25);

\path[fill=fillColor] ( 74.23,230.09) circle (  2.25);

\path[fill=fillColor] (426.00, 68.60) circle (  2.25);

\path[fill=fillColor] (453.69, 76.78) circle (  2.25);

\path[fill=fillColor] (391.20, 90.91) circle (  2.25);

\path[fill=fillColor] (127.54,136.77) circle (  2.25);

\path[fill=fillColor] (163.49,128.55) circle (  2.25);

\path[fill=fillColor] ( 99.24,206.98) circle (  2.25);

\path[fill=fillColor] ( 72.00,256.57) circle (  2.25);

\path[fill=fillColor] (129.85,177.67) circle (  2.25);

\path[fill=fillColor] (212.74, 99.02) circle (  2.25);

\path[fill=fillColor] (319.02, 88.66) circle (  2.25);

\path[fill=fillColor] (372.14, 77.92) circle (  2.25);

\path[fill=fillColor] (166.60,128.96) circle (  2.25);

\path[fill=fillColor] (400.07, 94.20) circle (  2.25);

\path[fill=fillColor] (380.14, 82.16) circle (  2.25);

\path[fill=fillColor] ( 68.71,314.51) circle (  2.25);

\path[fill=fillColor] (378.99, 46.84) circle (  2.25);

\path[fill=fillColor] (323.72, 82.18) circle (  2.25);

\path[fill=fillColor] (108.61,173.94) circle (  2.25);

\path[fill=fillColor] (127.16,145.33) circle (  2.25);

\path[fill=fillColor] (164.00,113.42) circle (  2.25);

\path[fill=fillColor] ( 94.19,209.39) circle (  2.25);

\path[fill=fillColor] (199.31,116.97) circle (  2.25);

\path[fill=fillColor] (147.49,129.05) circle (  2.25);

\path[fill=fillColor] (272.62,111.14) circle (  2.25);

\path[fill=fillColor] (115.97,188.03) circle (  2.25);

\path[fill=fillColor] (404.00, 60.91) circle (  2.25);

\path[fill=fillColor] ( 81.44,207.99) circle (  2.25);

\path[fill=fillColor] (424.94, 88.22) circle (  2.25);

\path[fill=fillColor] ( 76.40,213.05) circle (  2.25);

\path[fill=fillColor] (404.08, 88.89) circle (  2.25);

\path[fill=fillColor] ( 81.24,222.23) circle (  2.25);

\path[fill=fillColor] ( 57.79,311.61) circle (  2.25);

\path[fill=fillColor] (351.48,104.93) circle (  2.25);

\path[fill=fillColor] (395.98, 48.96) circle (  2.25);

\path[fill=fillColor] (107.92,208.94) circle (  2.25);

\path[fill=fillColor] (168.66,119.60) circle (  2.25);

\path[fill=fillColor] ( 84.85,203.74) circle (  2.25);

\path[fill=fillColor] ( 98.07,172.22) circle (  2.25);

\path[fill=fillColor] (224.23,100.70) circle (  2.25);

\path[fill=fillColor] (130.25,140.21) circle (  2.25);

\path[fill=fillColor] ( 62.23,272.53) circle (  2.25);

\path[fill=fillColor] (412.96, 88.40) circle (  2.25);

\path[fill=fillColor] (167.16,119.89) circle (  2.25);

\path[fill=fillColor] (181.55,135.38) circle (  2.25);

\path[fill=fillColor] (289.49, 69.27) circle (  2.25);

\path[fill=fillColor] (431.78, 60.73) circle (  2.25);

\path[fill=fillColor] (303.02, 71.68) circle (  2.25);

\path[fill=fillColor] (351.28, 67.39) circle (  2.25);

\path[fill=fillColor] (333.54,100.98) circle (  2.25);

\path[fill=fillColor] (216.86,110.84) circle (  2.25);

\path[fill=fillColor] (157.77,137.90) circle (  2.25);

\path[fill=fillColor] (345.32, 75.94) circle (  2.25);

\path[fill=fillColor] ( 97.81,222.27) circle (  2.25);

\path[fill=fillColor] ( 74.81,186.06) circle (  2.25);

\path[fill=fillColor] ( 85.08,194.74) circle (  2.25);

\path[fill=fillColor] (196.97,123.76) circle (  2.25);

\path[fill=fillColor] (160.40,125.75) circle (  2.25);

\path[fill=fillColor] (243.58, 93.34) circle (  2.25);

\path[fill=fillColor] (141.23,135.28) circle (  2.25);

\path[fill=fillColor] (211.38,119.87) circle (  2.25);

\path[fill=fillColor] (364.14, 87.52) circle (  2.25);

\path[fill=fillColor] (239.18, 77.35) circle (  2.25);

\path[fill=fillColor] (322.45, 67.19) circle (  2.25);

\path[fill=fillColor] (384.24, 87.90) circle (  2.25);

\path[fill=fillColor] (368.12, 78.31) circle (  2.25);

\path[fill=fillColor] ( 84.37,216.82) circle (  2.25);

\path[fill=fillColor] (446.99, 74.40) circle (  2.25);

\path[fill=fillColor] ( 65.10,269.43) circle (  2.25);

\path[fill=fillColor] (340.68, 81.98) circle (  2.25);

\path[fill=fillColor] (446.02, 90.03) circle (  2.25);

\path[fill=fillColor] (204.02,126.97) circle (  2.25);

\path[fill=fillColor] (172.74,128.12) circle (  2.25);

\path[fill=fillColor] (388.75,107.99) circle (  2.25);
\end{scope}
\begin{scope}
\path[clip] (  0.00,  0.00) rectangle (505.89,361.35);
\definecolor{drawColor}{RGB}{0,0,0}

\node[text=drawColor,anchor=base,inner sep=0pt, outer sep=0pt, scale=  1.20] at (252.94,339.14) {\bfseries Inverse relation};

\node[text=drawColor,anchor=base west,inner sep=0pt, outer sep=0pt, scale=  1.20] at (248.49, 14.53) {\itshape X};

\node[text=drawColor,rotate= 90.00,anchor=base west,inner sep=0pt, outer sep=0pt, scale=  1.20] at ( 21.73,176.22) {\itshape Y};
\end{scope}
\begin{scope}
\path[clip] (  0.00,  0.00) rectangle (505.89,361.35);
\definecolor{drawColor}{RGB}{0,0,0}

\path[draw=drawColor,line width= 0.4pt,line join=round,line cap=round] ( 36.13, 36.13) --
	(469.75, 36.13) --
	(469.75,325.21) --
	( 36.13,325.21) --
	( 36.13, 36.13);
\end{scope}
\end{tikzpicture}

\end{tabular}
}
\end{center}
\end{frame}


%---------------------------------------------------------------------slide----
\begin{frame}
\frametitle{Marginal frequency distributions}
The frequency distributions of each variable of the two-dimensional variable are known as \highlight{marginal frequency
distributions}.

We can get the marginal frequency distributions from the joint frequency table summing frequencies by rows and columns. 

\begin{center}
\[
\begin{array}{|c|ccccc|>{\columncolor{color1!50}}c|}
\hline
X\backslash Y & y_1 & \cdots & y_j & \cdots & y_q & n_x\\
\hline
x_1 & n_{11} & \cdots & n_{1j} & \cdots & n_{1q} & n_{x_1}\\
\vdots & \vdots & \vdots & \tikz[baseline=-0.5ex]{\node at (0,0) [fill=color2!50,single arrow,shape border
rotate=270, single arrow head extend=1mm]{$+$\phantom{}}; } & \vdots &
\vdots & \vdots \\
x_i & n_{i1} & \tikz[baseline=-0.5ex]{\node at (0,0) [fill=color1!50,single arrow,shape border rotate=0,
single arrow head extend=1mm]{$+$\phantom{}}; }  & n_{ij} & \tikz[baseline=-0.5ex]{\node
at (0,0) [fill=color1!50,single arrow,shape border rotate=0, single
arrow head extend=1mm]{$+$\phantom{}}; } & n_{iq} & n_{x_i}\\
\vdots & \vdots & \vdots & \tikz[baseline=-0.5ex]{\node at (0,0) [fill=color2!50,single arrow,shape border
rotate=270, single arrow head extend=1mm]{$+$\phantom{}}; } & \vdots & \vdots & \vdots\\
x_p & n_{p1} & \cdots & n_{pj} & \cdots & n_{pq} & n_{x_p} \\
\hline
\rowcolor{color2!50}
n_y & n_{y_1} & \cdots & n_{y_j} & \cdots & n_{y_q} & \cellcolor{white} n\\
\hline
\end{array}
\]
\end{center}
\end{frame}


%---------------------------------------------------------------------slide----
\begin{frame}
\frametitle{Marginal frequency distributions}
\framesubtitle{Example with heights and weights}
The marginal frequency distributions for the previous sample of heights and weights are
\[
\begin{array}{|c||c|c|c|c|c|c|>{\columncolor{color1!50}}c|}
\hline
  X/Y & [50,60) & [60,70) & [70,80) & [80,90) & [90,100) & [100,110) & n_x\\
  \hline\hline
  (150,160] & 2 & 0 & 0 & 0 & 0 & 0 & 2\\
  \hline
  (160,170] & 4 & 4 & 0 & 0 & 0 & 0 & 8\\
  \hline
  (170,180] & 1 & 6 & 3 & 1 & 0 & 0 & 11 \\
  \hline
  (180,190] & 0 & 1 & 4 & 1 & 1 & 0 & 7 \\
  \hline
  (190,200] & 0 & 0 & 0 & 0 & 1 & 1 & 2\\
  \hline
  \rowcolor{color2!50}
  n_y & 7 & 11 & 7 & 2 & 2 & 1 & \cellcolor{white} 30\\
  \hline
\end{array}
\]
and the corresponding statistics are 
\[
\begin{array}{lllll}
\bar x = 174.67 \mbox{ cm} & \quad & s^2_x = 102.06 \mbox{ cm}^2 & \quad & s_x = 10.1 \mbox{ cm}\\
\bar y = 69.67 \mbox{ Kg} & & s^2_y = 164.42 \mbox{ Kg}^2 & & s_y = 12.82 \mbox{ Kg}
\end{array}
\]
\end{frame}


\subsection{Covariance}
%---------------------------------------------------------------------slide----
\begin{frame}
\frametitle{Deviations from the means}
To study the relation between two variables, we have to analyze the joint variation of them. 

\centering
\resizebox{0.8\textwidth}{!}{% Created by tikzDevice version 0.10.1 on 2016-02-27 12:43:39
% !TEX encoding = UTF-8 Unicode
\begin{tikzpicture}[x=1pt,y=1pt]
\definecolor{fillColor}{RGB}{255,255,255}
\path[use as bounding box,fill=fillColor,fill opacity=0.00] (0,0) rectangle (325.21,238.49);
\begin{scope}
\path[clip] ( 34.80, 34.80) rectangle (313.21,236.09);
\definecolor{fillColor}{RGB}{5,161,230}

\path[fill=fillColor] (194.63,152.82) circle (  2.25);

\path[fill=fillColor] (163.70, 89.64) circle (  2.25);

\path[fill=fillColor] (204.94,108.59) circle (  2.25);

\path[fill=fillColor] (148.23, 89.64) circle (  2.25);

\path[fill=fillColor] ( 86.36, 45.41) circle (  2.25);

\path[fill=fillColor] (168.85, 92.80) circle (  2.25);

\path[fill=fillColor] (158.54, 80.16) circle (  2.25);

\path[fill=fillColor] (127.61, 73.85) circle (  2.25);

\path[fill=fillColor] (271.97,168.61) circle (  2.25);

\path[fill=fillColor] (225.57,121.23) circle (  2.25);

\path[fill=fillColor] (106.98, 58.05) circle (  2.25);

\path[fill=fillColor] (235.88,130.71) circle (  2.25);

\path[fill=fillColor] (292.59,228.64) circle (  2.25);

\path[fill=fillColor] (184.32, 77.00) circle (  2.25);

\path[fill=fillColor] (189.48,105.44) circle (  2.25);

\path[fill=fillColor] (122.45, 67.53) circle (  2.25);

\path[fill=fillColor] ( 65.73, 42.26) circle (  2.25);

\path[fill=fillColor] (215.25,178.09) circle (  2.25);

\path[fill=fillColor] (127.61, 45.41) circle (  2.25);

\path[fill=fillColor] (153.38, 89.64) circle (  2.25);

\path[fill=fillColor] (174.01,105.44) circle (  2.25);

\path[fill=fillColor] (210.10, 73.85) circle (  2.25);

\path[fill=fillColor] (132.76, 70.69) circle (  2.25);

\path[fill=fillColor] (143.07, 80.16) circle (  2.25);

\path[fill=fillColor] (158.54,105.44) circle (  2.25);

\path[fill=fillColor] (230.72,108.59) circle (  2.25);

\path[fill=fillColor] (158.54, 54.89) circle (  2.25);

\path[fill=fillColor] (179.16, 99.12) circle (  2.25);

\path[fill=fillColor] (137.92, 95.96) circle (  2.25);

\path[fill=fillColor] (235.88,137.02) circle (  2.25);
\end{scope}
\begin{scope}
\path[clip] (  0.00,  0.00) rectangle (325.21,238.49);
\definecolor{drawColor}{RGB}{0,0,0}

\node[text=drawColor,anchor=base,inner sep=0pt, outer sep=0pt, scale=  1.00] at (174.01,  3.60) {$X$};

\node[text=drawColor,rotate= 90.00,anchor=base,inner sep=0pt, outer sep=0pt, scale=  1.00] at ( 10.80,135.45) {$Y$};
\end{scope}
\begin{scope}
\path[clip] (  0.00,  0.00) rectangle (325.21,238.49);
\definecolor{drawColor}{RGB}{0,0,0}

\path[draw=drawColor,line width= 0.4pt,line join=round,line cap=round] ( 34.80, 34.80) --
	(313.21, 34.80) --
	(313.21,236.09) --
	( 34.80,236.09) --
	( 34.80, 34.80);

\path[draw=drawColor,line width= 0.4pt,line join=round,line cap=round] (172.31, 34.80) -- (172.31, 34.80);

\path[draw=drawColor,line width= 0.4pt,line join=round,line cap=round] (172.31, 34.80) -- (172.31, 30.77);

\node[text=drawColor,anchor=base,inner sep=0pt, outer sep=0pt, scale=  1.00] at (172.31, 20.40) {$\bar x$};

\path[draw=drawColor,line width= 0.4pt,line join=round,line cap=round] ( 34.80,104.39) -- ( 34.80,104.39);

\path[draw=drawColor,line width= 0.4pt,line join=round,line cap=round] ( 34.80,104.39) -- ( 30.77,104.39);

\node[text=drawColor,anchor=base east,inner sep=0pt, outer sep=0pt, scale=  1.00] at ( 30.00,100.95) {$\bar y$};
\end{scope}
\begin{scope}
\path[clip] ( 34.80, 34.80) rectangle (313.21,236.09);
\definecolor{drawColor}{RGB}{190,190,190}

\path[draw=drawColor,line width= 0.4pt,dash pattern=on 4pt off 4pt ,line join=round,line cap=round] ( 34.80,104.39) -- (313.21,104.39);

\path[draw=drawColor,line width= 0.4pt,dash pattern=on 4pt off 4pt ,line join=round,line cap=round] (172.31, 34.80) -- (172.31,236.09);
\definecolor{fillColor}{RGB}{238,50,36}

\path[fill=fillColor] (172.31,104.39) circle (  2.25);
\definecolor{drawColor}{RGB}{238,50,36}

\node[text=drawColor,anchor=base,inner sep=0pt, outer sep=0pt, scale=  1.00] at (158.54,112.41) {$(\bar x, \bar y)$};
\definecolor{drawColor}{RGB}{0,0,0}

\node[text=drawColor,anchor=base,inner sep=0pt, outer sep=0pt, scale=  1.00] at (287.44,175.59) {$(x_i,y_j)$};
\definecolor{drawColor}{RGB}{238,50,36}

\path[draw=drawColor,line width= 0.4pt,line join=round,line cap=round] (172.31,168.61) -- (271.97,168.61);

\path[draw=drawColor,line width= 0.4pt,line join=round,line cap=round,fill=fillColor] (264.89,165.96) --
	(271.97,168.61) --
	(264.89,171.27) --
	cycle;

\path[draw=drawColor,line width= 0.4pt,line join=round,line cap=round,fill=fillColor] (179.39,171.27) --
	(172.31,168.61) --
	(179.39,165.96) --
	cycle;

\node[text=drawColor,anchor=base,inner sep=0pt, outer sep=0pt, scale=  1.00] at (230.72,172.43) {$x_i-\bar x$};

\path[draw=drawColor,line width= 0.4pt,line join=round,line cap=round] (271.97,104.39) -- (271.97,168.61);

\path[draw=drawColor,line width= 0.4pt,line join=round,line cap=round,fill=fillColor] (274.62,161.54) --
	(271.97,168.61) --
	(269.31,161.54) --
	cycle;

\path[draw=drawColor,line width= 0.4pt,line join=round,line cap=round,fill=fillColor] (269.31,111.47) --
	(271.97,104.39) --
	(274.62,111.47) --
	cycle;

\node[text=drawColor,anchor=base,inner sep=0pt, outer sep=0pt, scale=  1.00] at (292.59,134.52) {$y_j-\bar y$};
\end{scope}
\end{tikzpicture}
}
\end{frame}

% 
% %---------------------------------------------------------------------slide----
% \begin{frame}
% \frametitle{Estudio de las desviaciones respecto de las medias}
% Si dividimos la nube de puntos del diagrama de dispersión en 4 cuadrantes centrados en el punto de medias $(\bar x, \bar y)$, el signo de las desviaciones será:
% \begin{center}
% \mode<presentation>{\rowcolors{1}{red!20}{red!10}}
% \begin{tabular}{cccc}
% Cuadrante & $(x_i-\bar x)$ & $(y_j-\bar y)$ & $(x_i-\bar x)(y_j-\bar y)$\\
% \hline
% 1 & $+$ & $+$ & \alert{$+$}\\
% 2 & $-$ & $+$ & \alert{$-$}\\
% 3 & $-$ & $-$ & \alert{$+$}\\
% 4 & $+$ & $-$ & \alert{$-$}\\
% \hline
% \end{tabular}
% 
% \scalebox{0.5}{\input{img/regresion/cuadrantes_nube_puntos}}
% \end{center}
% 
% \note{Si dividimos la nube de puntos del diagrama de dispersión en 4 cuadrantes centrados en el punto de medias $(\bar x, \bar y)$, el signo
% dependiendo del cuadrante donde caiga cada punto de la muestra, tendremos desviaciones con distinto signo:
% \begin{enumerate}
% \item Los puntos que caigan en el primer cuadrante tendrán desviación en $X$ y en $Y$ positivas ya que $x_i$ estará por encima de $\bar x$ e
% $y_j$ estará por encima de $\bar y$.
% \item Los puntos que caigan en el segundo cuadrante tendrán desviación en $X$ negativa, porque estarán por debajo de $\bar x$, pero
% desviación en $Y$ positiva puesto que estarán por encima de $\bar y$.
% \item Los puntos que caigan en el tercer cuadrante tendrán desviación en $X$ y en $Y$ negativas, por que estarán por debajo de la $\bar x$ y
% $\bar y$.
% \item Y los puntos que caigan en el cuarto cuadrante tendrán desviación en $X$ positiva, ya que estarán por encima de $\bar x$ pero
% desviación en $Y$ negativa ya que estarán por debajo de $\bar y$.
% \end{enumerate} 
% 
% Para ver la variabilidad conjunta en ambas varibles, se puede tomar el producto de las desviaciones, que para los puntos del primer
% cuadrante tendrá signo positivo ya que tanto las desviaciones en $X$ como las desviaciones en $Y$ son positivas. Para los puntos del segundo
% cuadrante tendremos, menos por más, menos, para los del tercero menos por menos, más, y para los del cuarto más por menos, menos.
% 
% Así pues, en los puntos del primer y tercer cuadrantes, los productos de las desviaciones a las medias tendrán signo positivo, mientras que
% en el segundo y cuarto cuadrantes tendrán signo negativo.}
% \end{frame}
% 
% 
% %---------------------------------------------------------------------slide----
% \begin{frame}
% \frametitle{Estudio de las desviaciones respecto de las medias}
% \begin{columns}
% \begin{column}{0.5\textwidth}
% Si la relación entre las variables es \emph{lineal y creciente}, entonces la
% mayor parte de los puntos estarán en los cuadrantes 1 y 3 y la suma de los productos de desviaciones será positiva.
% \begin{center}
% \scalebox{0.43}{\input{img/regresion/diagrama_dispersion_relacion_lineal_creciente}}
% \end{center} 
% \[
% \sum(x_i-\bar x)(y_j-\bar y) = +
% \]
% \end{column}
% \begin{column}{0.5\textwidth}
% Si la relación entre las variables es \emph{lineal y decreciente}, entonces la
% mayor parte de los puntos estarán en los cuadrantes 2 y 4 y la suma de los productos de desviaciones será negativa.
% \begin{center}
% \scalebox{0.43}{\input{img/regresion/diagrama_dispersion_relacion_lineal_decreciente}}
% \end{center} 
% \[
% \sum(x_i-\bar x)(y_j-\bar y) = -
% \]
% \end{column}
% \end{columns}
% 
% \note{Si la relación entre las variables es lineal y creciente, como refleja el primer diagrama de dispersión, entonces la mayor parte de
% los puntos estarán en los cuadrantes 1 y 3, y por tanto habrá muchos más puntos cuyo producto de desviaciones sea positivo, que puntos
% cuyo producto de desviaciones sea negativo, de manera que la suma de los productos de desviaciones de todos los puntos será positiva.
% 
% Por contra, cuando la relación entre las variables sea lineal y decreciente, como ocurre en el segundo diagrama, entonces la la mayor parte de
% los puntos estarán en los cuadrantes 2 y 4, y por tanto habrá muchos más puntos cuyo producto de desviaciones sea negativo, que puntos
% cuyo producto de desviaciones sea positivo, de manera que la suma de los productos de desviaciones de todos los puntos será negativa.
% }
% \end{frame}
% 
% 
% %---------------------------------------------------------------------slide----
% \begin{frame}
% \frametitle{Covarianza}
% Del estudio conjunto de las desviaciones respecto de la media surge el siguiente estadístico de relación lineal:
% \begin{definicion}[Covarianza muestral]
% La \emph{covarianza muestral} de una variable aleatoria bidimensional $(X,Y)$ se define como el promedio de los
% productos de las respectivas desviaciones respecto de las medias de $X$ e $Y$.
% \[
% s_{xy}=\frac{\sum (x_i-\bar x)(y_j-\bar y)n_{ij}}{n}
% \]
% \end{definicion}
% También puede calcularse de manera más sencilla mediante la fórmula
% \[
% s_{xy}=\frac{\sum x_iy_jn_{ij}}{n}-\bar x\bar y.
% \]
% La covarianza sirve para estudiar la relación lineal entre dos variables:
% \begin{itemize}
% \item Si $s_{xy}>0$ existe una relación lineal creciente entre las variables.
% \item Si $s_{xy}<0$ existe una relación lineal decreciente entre las variables.
% \item Si $s_{xy}=0$ no existe relación lineal entre las variables. 
% \end{itemize} 
% 
% \note{Del razonamiento anterior surge un estadístico conocido como covarianza para estudiar la relación lineal entre variables.
% 
% La covarianza, que se representa $s_{xy}$ se define como la suma de los productos de las desviaciones a las respectivas medias,
% multiplicadas por la frecuencia absoluta relativa de cada par y dividido por el tamaño de la muestra.
% 
% Si se desarrolla el producto de las desviaciones y se simplifica, se llega a una fórmula equivalente para el cálculo de la covarianza que
% consiste en sumar los productos del valor de $X$ por el valor de $Y$ en cada individuo, por la frecuencia absoluta del par, dividido por el
% tamaño de la muestra y restarle al cociente el producto de las medias de $X$ e $Y$. Esta fórmula es un poco más fácil de aplicar y por
% tanto será la que utilizaremos a lo largo del tema.
% 
% A la hora de interpretar la covarianza como medida de relación lineal, lo más importante es su signo, ya que como vimos antes, si existe
% relación lineal creciente entre las variables, la suma de los productos de las desviaciones será positiva y también lo será la
% covarianza; si la relación es lineal decreciente, la suma de los productos de las desviaciones será negativa y también lo será la
% covarianza; mientras que si no hay relación lineal, los puntos se distribuirán más o menos por igual en los cuatro cuadrantes y se
% compensarán los productos de desviaciones positivos con los negativos, con lo que la covarianza valdrá aproximadamente 0.
% 
% Hay que tener en cuenta que la covarianza tiene unidades, que son el producto de las unidades de $X$ e $Y$, lo cual dificulta su
% interpretación a la hora de valorar el grado de dependencia lineal entre las variables, aunque, como veremos más adelante, a partir de ella
% es posible obtener una medida adimensional que si permitirá estudiar el grado de dependencia lineal.}
% \end{frame}
% 
% 
% %---------------------------------------------------------------------slide----
% \begin{frame}
% \frametitle{Cálculo de la covarianza}
% \framesubtitle{Ejemplo con estaturas y pesos}
% En el ejemplo de las estaturas y pesos, teniendo en cuenta que 
% \[
% \begin{array}{|c||c|c|c|c|c|c|c|}
% \hline
%   X/Y & [50,60) & [60,70) & [70,80) & [80,90) & [90,100) & [100,110) & n_x\\
%   \hline\hline
%   (150,160] & 2 & 0 & 0 & 0 & 0 & 0 & 2\\
%   \hline
%   (160,170] & 4 & 4 & 0 & 0 & 0 & 0 & 8\\
%   \hline
%   (170,180] & 1 & 6 & 3 & 1 & 0 & 0 & 11 \\
%   \hline
%   (180,190] & 0 & 1 & 4 & 1 & 1 & 0 & 7 \\
%   \hline
%   (190,200] & 0 & 0 & 0 & 0 & 1 & 1 & 2\\
%   \hline
%   n_y & 7 & 11 & 7 & 2 & 2 & 1 & 30\\
%   \hline
% \end{array}
% \]
% \[
% \bar x = 174.67 \mbox{ cm} \qquad \bar y = 69.67 \mbox{ Kg} 
% \]
% la covarianza vale
% \begin{align*}
% s_{xy} &=\frac{\sum x_iy_jn_{ij}}{n}-\bar x\bar y =  \frac{155\cdot 55\cdot 2 + 165\cdot 55\cdot 4 + \cdots + 195\cdot 105\cdot 1}{30}-174.67\cdot 69.67 =\\
% & = \frac{368200}{30}-12169.26 = 104.07 \mbox{ cm$\cdot$ Kg},
% \end{align*}
% lo que indica que existe una relación lineal creciente entre la estatura y el peso. 
% 
% \note{En el ejemplo de las estaturas y los pesos, para calcular la covarianza a partir de la tabla de frecuencias bidimensional, iremos
% sumando el producto de la componente $X$ por la componente $Y$ por su frecuencia absoluta para cada posible par. Como en este caso, además,
% los datos se habían agrupado en intervalos, a la hora de hacer estos productos tomaremos las marcas de cada clase, es decir, para el primer
% par, correspondiente al intervalo $(150,160]$ de estatura y $[50,60)$ de peso, haremos el producto de 155 que es la marca de clase del
% intervalo de la estatura, por 55 que es la marca de clase del intervalo del peso, por su frecuencia absoluta que es 2. A continuación vamos
% calculando el resto de productos de igual modo. 155 por 65 y por su frecuencia que es 0, lo que nos da 0, etc. Una vez calculados los
% productos se suman, lo que da 368200, esto se divide por el tamaño de la muestra que era 30 y al resultado del cociente se le resta el
% producto de las medias que eran $174.67$ cm y $69.67$ kg, lo que nos da una covarianza de $104.07$ cm por kg.
% 
% Esto indica que existe una relación lineal creciente entre el peso y la estatura, es decir, que a mayor estatura le corresponde mayor peso
% y a mayor peso le corresponde mayor estatura, aunque resulta complicado establecer si esa relación es fuerte o débil ya que no es fácil
% interpretar las unidades de cm por kg.}
% \end{frame}
% 
% 
% \subsection{Regresión}
% %---------------------------------------------------------------------slide----
% \begin{frame}
% \frametitle{Regresión}
% En muchos casos el objetivo de un estudio no es solo detectar una relación entre variables, sino explicarla mediante
% alguna función matemática.
% 
% La \highlight{regresión}} es la parte de la estadística que trata de determinar la posible relación entre una variable numérica
% dependiente $Y$, y otro conjunto de variables numéricas independientes, $X_1,X_2,\ldots,X_n$, de una misma población. Dicha relación se
% refleja mediante un modelo funcional 
% \[ y=f(x_1,\ldots,x_n). \]
% 
% El objetivo es determinar una ecuación mediante la que pueda estimarse el valor de la variable dependiente en función de los valores de las independientes.
% 
% El caso más sencillo se da cuando sólo hay una variable independiente $X$, entonces se habla de \emph{regresión
% simple}. En este caso el modelo que explica la relación de $Y$ como función de $X$ es una función de una variable
% $y=f(x)$ que se conoce como \highlight{función de regresión}}.
% 
% \note{En muchos casos el objetivo de un estudio no es solo detectar si existe cierta relación entre variables, sino explicarla mediante
% alguna función matemática.
% 
% De esto se encarga la regresión que trata de buscar un modelo funcional que explique lo mejor posible la relación entre una variable
% dependiente $Y$ y un conjunto de variables independientes $X_1,\ldots,X_n$. 
% 
% El caso más simple, y el que estudiaremos aquí, es cuando sólo hay una variable independiente $X$ y entonces el modelo es una función de una
% sóla variable $y=f(x)$ que se conoce como función de regresión simple.
% }
% \end{frame}
% 
% 
% %---------------------------------------------------------------------slide----
% \begin{frame}
% \frametitle{Modelos de regresión simple}
% Dependiendo de la forma de función de regresión, existen muchos tipos de regresión simple. Los más habituales son los que aparecen en la
% siguiente tabla:
% \begin{center}
% \mode<presentation>{\rowcolors{1}{red!20}{red!10}}
% \begin{tabular}{|l|c|}
% \hline
%  Familia de curvas       &     Ecuación genérica      \\
% \hline\hline
%  Lineal                  &          $y=a+bx$          \\
% \hline
%  Parabólica              &       $y=a+bx+cx^2$        \\
% \hline
%  Polinómica de grado $n$ & $y=a_0+a_1x+\cdots+a_nx^n$ \\
% \hline
%  Potencial               &       $y=a\cdot x^b$       \\
% \hline
%  Exponencial             &     $y=a\cdot e^{bx}$      \\
% \hline
%  Logarítmica             &       $y=a+b\log x$        \\
% \hline
%  Inverso                 &       $y = a+\frac{b}{x}$  \\
% \hline
%  Curva S                 &      $y= e^{a+\frac{b}{x}}$ \\
% \hline
% \end{tabular}
% \end{center}
% 
% La elección de un tipo u otro depende de la forma que tenga la nube de puntos del diagrama de dispersión. 
% 
% \note{Dependiendo de la forma de función de regresión, existen muchos tipos de regresión simple. Los más habituales son:
% \begin{enumerate}
% \item El modelo lineal, en el que la función de regresión es una recta.
% \item El modelo parabólico, en el que la función de regresión es una parabola, es decir, un polinomio de grado 2.
% \item El modelo polinómico de grado $n$, en el que la función de regresión es un polinomio de grado $n$.
% \item El modelo potencial, en el que la función de regresión es una función potencial donde el exponente puede ser un númro no entero.
% \item El modelo exponencia, en el que la función de regresión es una función exponencial, habitualmente de base $e$, la constante de Euler,
% y donde la variable independiente aparece en el exponente.
% \item El modelo logarítmico, en el que la función de regresión es una función logarítmica.
% \item El modelo inverso, en el que en la función de regresión la variable independiente aparece dividiendo a una constante.
% \item Y el modelo de curva S o sigmoidal, que es una combinación del modelo exponencial e inverso.
% \end{enumerate}
% 
% A la hora de decidir qué modelo construir habrá que tener en cuenta la forma que tenga la nube de puntos del diagrama de dispersión.
% }
% \end{frame}
% 
% 
% %---------------------------------------------------------------------slide----
% \begin{frame}
% \frametitle{Residuos o errores predictivos}
% Una vez elegida la familia de curvas que mejor se adapta a la nube de puntos, se determina, dentro de dicha familia, la curva que mejor se
% ajusta a la distribución.
% 
% El objetivo es encontrar la función de regresión que haga mínimas las distancias entre los valores de la variable dependiente observados en
% la muestra, y los predichos por la función de regresión. Estas distancias se conocen como \emph{residuos} o \emph{errores predictivos}.
% 
% \begin{definicion}[Residuos o Errores predictivos]
% Dado el modelo de regresión $y=f(x)$ para una variable bidimensional $(X,Y)$, el \emph{residuo} o \emph{error
% predictivo} de un valor $(x_i,y_j)$ observado en la muestra, es la diferencia entre el valor observado de la variable dependiente $y_j$ y el predicho por la función de regresión para $x_i$:
% \[
% e_{ij} = y_j-f(x_i) .
% \]
% \end{definicion}
% 
% \note{Una vez elegida la familia de curvas que mejor se adapta a la nube de puntos, se determina, dentro de dicha familia, la curva que mejor se
% ajusta a la distribución.
% 
% El objetivo es encontrar la función de regresión que haga mínimas las distancias entre los valores de la variable dependiente observados en
% la muestra, y los predichos por la función de regresión. Estas distancias se conocen como \emph{residuos} o \emph{errores predictivos} en
% $Y$ y se calculan para cada punto $(x_i,y_j)$ restando al valor observado en la variable dependiente $y_j$ lo que vale la función de regresión para
% el valor observado en la variable dependiente $f(x_i)$..
% }
% \end{frame}
% 
% 
% %---------------------------------------------------------------------slide----
% \begin{frame}
% \frametitle{Residuos o errores predictivos en $Y$}
% \begin{center}
% \scalebox{0.8}{\input{img/regresion/residuos_y}}
% \end{center}
% \note{Es decir, en el caso de representar la variable dependiente en el eje $Y$ los residuos de cada punto serían las distancias verticales
% desde el punto hasta la curva de la función de regresión.
% 
% Está claro que cuanto más pequeños sean los residuos, máx próxima estará la curva de regresión a los puntos de la nube de puntos y
% mejor explicará la relación entre las variables. 
% }
% \end{frame}
% 
% 
% %---------------------------------------------------------------------slide----
% \begin{frame}
% \frametitle{Método de mínimos cuadrados}
% Una forma posible de obtener la función de regresión es mediante el método de \emph{mínimos cuadrados} que consiste en calcular la función
% que haga mínima la suma de los cuadrados de los residuos
% \[
% \sum e_{ij}^2.
% \]
% 
% En el caso de un modelo de regresión lineal $f(x) = a + bx$, como la recta depende de dos parámetros (el término independiente $a$ y la
% pendiente $b$), la suma también dependerá de estos parámetros
% \[
% \theta(a,b) = \sum e_{ij}^2 =\sum (y_j - f(x_i))^2 =\sum (y_j-a-bx_i)^2.
% \]
% 
% Así pues, todo se reduce a buscar los valores $a$ y $b$ que hacen mínima esta suma. 
% 
% \note{Según este criterio, para buscar la curva de regresión mejor se ajuste a la nube de puntos se suele utilizar un método conocido como
% método de los mínimos cuadrados, que consiste en calcular la curva que haga mínima la suma de los residuos al cuadrado. Se toman los
% cuadrados porque como los residuos pueden ser positivos y negativos, al sumarlos sin más podrían compensarse. 
% 
% Por ejemplo, en el caso del ajuste de una recta de ecuación $f(x) = a + bx$, hay que determinar los dos parámetros $a$ y $b$ que definen la
% recta. En este caso, los residuos de cada punto $e_{ij}=y_j-f(x_i)$ se convierten en $e_{ij}=y_j-a-bx_i$ ya que $f(x_i)$ es el valor de la
% recta en $x_i$, y por tanto los residuos dependen de los parámetros de la recta $a$ y $b$. 
% En consecuencia, el problema se reduce a calcular los valores de $a$ y $b$ que hagan mínima la suma de los residuos al cuadrado. Es decir,
% se trata de un problema de minimización.}
% \end{frame}
% 
% 
% \subsection{Recta de regresión}
% %---------------------------------------------------------------------slide----
% \begin{frame}
% \frametitle{Cálculo de la recta de regresión}
% \framesubtitle{Método de mínimos cuadrados}
% Considerando la suma de los cuadrados de los residuos como una función de dos variables $\theta(a,b)$, se pueden calcular los valores de los
% parámetros del modelo que hacen mínima esta suma derivando e igualando a 0 las derivadas:
% \begin{align*}
% \frac{\partial \theta(a,b)}{\partial a} &=  \frac{\partial \sum (y_j-a-bx_i)^2 }{\partial a} =0\\
% \frac{\partial \theta(a,b)}{\partial b} &=  \frac{\partial \sum (y_j-a-bx_i)^2 }{\partial b} =0
% \end{align*}
% 
% Tras resolver el sistema se obtienen los valores
% \[
% a= \bar y - \frac{s_{xy}}{s_x^2}\bar x \qquad b=\frac{s_{xy}}{s_x^2}
% \]
% Estos valores hacen mínimos los residuos en $Y$ y por tanto dan la recta de regresión.
% 
% \note{Aunque no insistiremos en los detalles matemáticos del cálculo de mínimos, hay que recordar que el cálculo del mínimo de una función
% se hacía entre los puntos que anulaban su derivada. En nuestro caso, puesto que la suma de los residuos al cuadrado depende de dos
% parámetros $a$ y $b$, habría que tomar las derivadas parciales con respecto a $a$ y $b$, igualarlas a cero y resolver el sistema de
% ecuaciones que forman.
% 
% Tras resolver el sistema se obtiene que los valores de $a$ y $b$ que hacen mínima la suma de los residuos al cuadrado son $a= \bar y -
% \frac{s_{xy}}{s_x^2}\bar x$ y $b=\frac{s_{xy}}{s_x^2}$.
% }
% \end{frame}
% 
% 
% %---------------------------------------------------------------------slide----
% \begin{frame}
% \frametitle{Recta de regresión}
% \begin{definicion}[Recta de regresión]
% Dada una variable bidimensional $(X,Y)$, la \emph{recta de regresión} de $Y$ sobre $X$ es
% \[
% y = \bar y +\frac{s_{xy}}{s_x^2}(x-\bar x).
% \] 
% \end{definicion}
% 
% La recta de regresión de $Y$ sobre $X$ es la recta que hace mínimos los errores predictivos en $Y$, y por tanto es la recta que hará mejores
% predicciones de $Y$ para cualquier valor de $X$.
% 
% \note{Al sustituir estos valores en la ecuación genérica de la recta $f(x)=a+bx$, se obtiene la recta de ecuación $y = \bar y
% +\frac{s_{xy}}{s_x^2}(x-\bar x)$ que se conoce como recta de regresión de $Y$ sobre $X$ y es la recta que mejor explica la relación lineal
% entre $Y$ y $X$ ya que hace mínimos los residuos o errores predictivos en $Y$, y por tanto es la recta que hará mejores predicciones de $Y$
% en función de $X$.
% }
% \end{frame}
% 
% 
% %---------------------------------------------------------------------slide----
% \begin{frame}
% \frametitle{Cálculo de la recta de regresión}
% \framesubtitle{Ejemplo con estaturas y pesos}
% Siguiendo con el ejemplo de las estaturas ($X$) y los pesos ($Y$) con los siguientes estadísticos:
% \[
% \begin{array}{lllll}
% \bar x = 174.67 \mbox{ cm} & \quad & s^2_x = 102.06 \mbox{ cm}^2 & \quad & s_x = 10.1 \mbox{ cm}\\
% \bar y = 69.67 \mbox{ Kg} & & s^2_y = 164.42 \mbox{ Kg}^2 & & s_y = 12.82 \mbox{ Kg}\\
% & & s_{xy} = 104.07 \mbox{ cm$\cdot$ Kg} & &
% \end{array}
% \]
% Entonces, la recta de regresión del peso sobre la estatura es:
% \[
% y = \bar y +\frac{s_{xy}}{s_x^2}(x-\bar x) = 69.67+\frac{104.07}{102.06}(x-174.67) = 1.02 x - 108.49.
% \]
% De igual modo, si en lugar de considerar el peso como variable dependiente, tomamos la estatura, entonces la recta de regresión de la estatura sobre el peso es:
% \[
% x = \bar x +\frac{s_{xy}}{s_y^2}(y-\bar y) = 174.67+\frac{104.07}{164.42}(y-69.67) = 0.63 y +130.78.
% \]
% 
% \note{Siguiendo con el ejemplo de las estaturas y los pesos, si tomamos como variable independiente $X$ la estatura, y como variable
% dependiente $Y$ los pesos, la recta de regresión del peso sobre la estarura tendrá como ecuación 
% \[
% y = \bar y +\frac{s_{xy}}{s_x^2}(x-\bar x) = 69.67+\frac{104.07}{102.06}(x-174.67) = 1.02 x - 108.49.
% \]
% 
% De igual modo, si en lugar de considerar el peso como variable dependiente, tomamos la estatura, entonces la recta de regresión de la estatura sobre el peso es:
% \[
% x = \bar x +\frac{s_{xy}}{s_y^2}(y-\bar y) = 174.67+\frac{104.07}{164.42}(y-69.67) = 0.63 y +130.78.
% \]
% }
% \end{frame}
% 
% 
% %---------------------------------------------------------------------slide----
% \begin{frame}
% \frametitle{Rectas de regresión}
% \framesubtitle{Ejemplo de estaturas y pesos}
% \begin{center}
% \scalebox{0.78}{\input{img/regresion/rectas_regresion}}
% \end{center}
% 
% \note{Si dibujamos ambas rectas, la del peso sobre la estatura y la de la estatura sobre el peso, podemos comprobar que se trata de rectas
% distintas, ya que la primera hace mínimos los residuos del peso, y por tanto es la que debe utilizarse para predecir el peso en función de
% la estatura, mientras que la segunda es la que hace mínimos los residuos de la estatura, y por tanto debe utilizarse para predecir la
% estatura en función del peso.
% 
% Además puede comprobarse que las rectas de regresión se cortan en el punto de medias, lo cual se deduce fácilmente de sus respectivas
% ecuaciones.
% }
% \end{frame}
% 
% 
% %---------------------------------------------------------------------slide----
% \begin{frame}
% \frametitle{Posición relativa de las rectas de regresión}
% Las rectas de regresión siempre se cortan en el punto de medias $(\bar x,\bar y)$.
% 
% \begin{columns}
% \begin{column}{0.48\textwidth}
% Si entre las variables la relación lineal es perfecta, entonces ambas rectas coinciden ya que sus residuos son nulos.
% \begin{center}
% \scalebox{0.5}{\input{img/regresion/rectas_dependencia_lineal_perfecta}}
% \end{center}
% \end{column}
% \begin{column}{0.48\textwidth}
% Si no hay relación lineal, entonces las ecuaciones de las rectas son\\
% \qquad \qquad $y = \bar y$, \qquad $x = \bar x,$\\
% y se cortan perpendicularmente  
% \begin{center}
% \scalebox{0.5}{\input{img/regresion/rectas_independencia_lineal}}
% \end{center}
% \end{column}
% \end{columns}
% 
% \note{La posición relativa de las rectas de regresión depende del grado de relación lineal entre las variables. 
% 
% Si la relación lineal entre las variables es perfecta, entonces los puntos de la nube de puntos estarán perfectamente alineados y la
% recta de regresión de $Y$ sobre $X$ y de $X$ sobre $Y$ coincidirán en la recta que pasa por todos los puntos ya que para esta recta tanto
% los residuos en $Y$ como los residuos en $X$ serán nulos.
% 
% Mientras que si no hay relación lineal entre las variables, entonces la covarianza será nula y la ecuación de las rectas será constante, $y
% = \bar y$, para la recta de $Y$ sobre $X$, y $x = \bar x$ para la de $X$ sobre $Y$, de manera que las rectas se cortarán perpendicularmente en el punto
% de medias.
% 
% Así pues, cuanto menor sea el ángulo con que se cortan las rectas, mayor será la dependencia lineal entre las variables.
% }
% \end{frame}
% 
% 
% %---------------------------------------------------------------------slide----
% \begin{frame}
% \frametitle{Coeficiente de regresión}
% \begin{definicion}[Coeficiente de regresión $b_{yx}$]
% Dada una variable bidimensional $(X,Y)$, el \emph{coeficiente de regresión} de la recta de regresión de $Y$ sobre $X$ es su pendiente,
% \[
% b_{yx} = \frac{s_{xy}}{s_x^2} 
% \]
% \end{definicion}
% 
% El coeficiente de regresión siempre tiene el mismo signo que la covarianza y refleja el crecimiento de la recta de regresión, ya que da el
% número de unidades que aumenta o disminuye la variable dependiente por cada unidad que aumenta la variable independiente, según la recta de
% regresión.
% 
% En el ejemplo de las estaturas y los pesos, el coeficiente de regresión del peso sobre la estatura es $b_{yx}= 1.02$ Kg/cm, lo que indica
% que, según la recta de regresión del peso sobre la estatura, por cada cm más de estatura, la persona pesará $1.02$ Kg más.
% 
% \note{En la ecuación de la recta de regresión de $Y$ sobre $X$, la parte más importante es la pendiente que se conoce como coeficiente de
% regresión de $Y$ sobre $X$ y vale $b_{yx} = \frac{s_{xy}}{s_x^2}$.
% 
% El coeficiente de regresión siempre tiene el mismo signo que la covarianza y refleja el crecimiento de la recta de regresión, ya que da el
% número de unidades que aumenta o disminuye la variable dependiente por cada unidad que aumenta la variable independiente, según la recta de
% regresión.
% 
% En el ejemplo de las estaturas y los pesos, el coeficiente de regresión del peso sobre la estatura es $b_{yx}= 1.02$ Kg/cm, lo que indica
% que, según la recta de regresión del peso sobre la estatura, por cada cm más de estatura, la persona pesará $1.02$ Kg más.
% }
% \end{frame}
% 
% 
% %---------------------------------------------------------------------slide----
% \begin{frame}
% \frametitle{Predicciones con las rectas de regresión}
% \framesubtitle{Ejemplo con estaturas y pesos}
% Las rectas de regresión, y en general cualquier modelo de regresión, suele utilizarse con fines predictivos. 
% 
% \begin{center}
% \alert{\emph{¡Ojo! Para predecir una variable, esta siempre debe considerarse como dependiente en el modelo de regresión que se utilice.}}
% \end{center}
% 
% Así, en el ejemplo de las estaturas y los pesos, si se quiere predecir el peso de una persona que mide 180 cm, se debe utilizar la recta de
% regresión del peso sobre la estatura:
% \[
% y = 1.02 \cdot 180 -108.49 = 75.11 \mbox{ Kg}.
% \] 
% Y si se quiere predecir la estatura de una persona que pesa 79 Kg, se debe utilizar la recta de regresión de la estatura sobre el peso:
% \[
% x = 0.63\cdot 79+ 130.78 = 180.55 \mbox{ cm}. 
% \]
% \begin{center}
% \emph{Ahora bien, ¿qué fiabilidad tienen estas predicciones?}
% \end{center}
% 
% \note{Las rectas de regresión, y en general cualquier modelo de regresión, suele utilizarse con fines predictivos. 
% 
% Como hemos visto, dependiendo de si se toma $X$ o $Y$ como la variable dependiente, tendremos modelos de regresión, el de $Y$ sobre $X$ y
% el de $X$ sobre $Y$, pero a la hora de hacer predicciones de una variable, siempre debe utilizarse el modelo en el que dicha variable sea la
% variable dependiente, ya que ese será el modelo que haga mínimos los errores predictivos en dicha variable.
% 
% Por ejemplo, en el caso de las estaturas y los pesos,  si se quiere predecir el peso de una persona que mide 180 cm, se debe utilizar la recta de
% regresión del peso sobre la estatura. Sutituyendo en esta recta $X$ por 180 cm se obtiene una predicción del peso de $75.11$ kg.
% 
% Y si se quiere predecir la estatura de una persona que pesa 79 Kg, se debe utilizar la recta de regresión de la estatura sobre el peso.
% Sustituyendo en esta otra recta $Y$ por 79 kg, se obtiene una predicción de la estatura de $180.55$ cm.
% }
% \end{frame}
% 
% 
% \subsection{Correlación}
% % ---------------------------------------------------------------------slide----
% \begin{frame}
% \frametitle{Correlación}
% Una vez construido un modelo de regresión, para saber si se trata de un buen modelo predictivo, se tiene que analizar
% el grado de dependencia entre las variables según el tipo de dependencia planteada en el modelo. De ello se encarga la
% parte de la estadística conocida como \highlight{correlación}}.
% 
% Para cada tipo de modelo existe el correspondiente tipo de correlación.
% 
% La correlación se basa en el estudio de los residuos. Cuanto menores sean éstos, más se ajustará la curva de regresión a los puntos, y más
% intensa será la correlación.
% 
% \note{Una vez construido un modelo de regresión, para saber si se trata de un buen modelo predictivo, se tiene que analizar
% el grado de dependencia entre las variables según el tipo de dependencia planteada en el modelo. De ello se encarga la
% parte de la estadística conocida como \highlight{correlación}}.
% 
% Para cada tipo de modelo existe el correspondiente tipo de correlación. Así por ejemplo, si se ha construido la recta de regresión
% hablaremos de correlación lineal, si se ha construído un modelo parabólico hablaremos de correlación parabólica, y lo mismo para cualquier
% tipo de modelo de regresión.
% 
% La correlación se basa de nuevo en el estudio de los residuos para el modelo de regresión construido. Cuanto menores sean éstos, más se
% ajustará la curva de regresión a los puntos, mejor explicará dicho modelo la relación entre las variables.
% }
% \end{frame}
% 
% 
% %---------------------------------------------------------------------slide----
% \begin{frame}
% \frametitle{Varianza residual muestral}
% Una medida de la bondad del ajuste del modelo de regresión es la \emph{varianza residual}.
% \begin{definicion}[Varianza residual $s_{ry}^2$]
% Dado un modelo de regresión simple $y=f(x)$ de una variable bidimensional $(X,Y)$, su \emph{varianza residual muestral} es el promedio de los cuadrados de los residuos para los valores de la muestra,
% \[
% s_{ry}^2 = \frac{\sum e_{ij}^2n_{ij}}{n} = \frac{\sum (y_j - f(x_i))^2n_{ij}}{n}.
% \]
% \end{definicion}
% Cuanto más alejados estén los puntos de la curva de regresión, mayor será la varianza residual y menor la dependencia.
% 
% \note{A partir de los residuos se puede calcular una primera medida de la bondad del ajuste del modelo de regresión de $Y$ sobre $X$,
% conocida como varianza residual. La varianza residual se representa $s_{ry}$ y es la suma de los residuos al cuadrado, dividida por el
% tamaño de la muestra.
% 
% Resulta evidente que cuanto más alejados estén los puntos del modelo de regresión, mayores serán los residuos y mayor será la varianza
% residual, lo que indicará una menor dependencia del tipo que plantea el modelo de regresión.
% 
% Ahora bien, la varianza residual tiene el problema de que sus unidades son las unidades de la variable dependiente al cuadrado, y por tanto
% resulta difícil de interpretar.
% }
% \end{frame}
% 
% 
% %---------------------------------------------------------------------slide----
% \begin{frame}
% \frametitle{Descomposición de la variabilidad total: \\Variabilidad explicada y no explicada}
% \begin{center}
% \scalebox{0.8}{\input{img/regresion/descomposicion_variabilidad}}
% \end{center}
% 
% \note{Para interpretar mejor la varianza residual, si dibujamos sobre el diagrama de dispersión, podemos comprobar que la variabilidad de
% la variable dependiente depende de las desviaciones de sus valores a su media que en el gráfico aparece representada por esta línea
% horizontal. La suma de estas desviaciones la cuadrado, divididas por el tamaño de la muestra nos daba precisamente la varianza de $Y$.
% 
% Si ahora dibujamos sobre el diagrama de dispersión el modelo de regresión de $Y$ sobre $X$, en este caso la recta de regresión, podemos
% comprobar que la recta divide en dos la desviación de cada valor de $Y$ a su media. Una parte es precisamente el residuo en $Y$, ya que es
% la distancia vertical del punto a la recta, y se dice que es la variablidad no explicada por el modelo de regresión, mientras que la otra
% parte, la distancia vertical desde la recta a la media de $Y$ se dice que es la variabilidad explicada por el modelo de regresión.
% 
% Cuanto mayor sea la variabilidad explicada, menor será la no explicada, es decir menores serán los residuos y mejor explicará el modelo de
% regresión la dependencia entre las variables. Cuando el ajuste sea perfecto, la variabilidad no explicada será nula y diremos que el modelo
% de regresión de $Y$ sobre $X$ explica toda la variabilidad de $Y$, mientras que cuando no haya relación entre las variables del tipo que
% plantee el modelo de regresión, entonces la variablidad explicada será nula y los residuos coincidirán con las desviaciones a la media de
% $Y$, de manera que la varianza residual coincidirá con la varianza de $Y$.
% }
% \end{frame}
% 
% 
% \subsection{Coeficientes de determinación y correlación}
% %---------------------------------------------------------------------slide----
% \begin{frame}
% \frametitle{Coeficiente de determinación}
% A partir de la varianza residual se puede definir otro estadístico más sencillo de interpretar.
% \begin{definicion}[Coeficiente de determinación muestral]
% Dado un modelo de regresión simple $y=f(x)$ de una variable bidimensional $(X,Y)$, su \emph{coeficiente de determinación muestral} es
% \[
% r^2 = 1- \frac{s_{ry}^2}{s_y^2} 
% \]
% \end{definicion}
% El coeficiente de determinación mide la proporción de variabilidad de la variable dependiente explicada por el modelo de regresión, y por
% tanto, 
% \[
% \alert{0\leq r^2\leq 1}
% \]
% Cuanto mayor sea $r^2$, mejor explicará el modelo de regresión la relación entre las variables, en particular:
% \begin{itemize}
% \item Si $r^2 =0$ entonces no existe relación del tipo planteado por el modelo.
% \item Si $r^2=1$ entonces la relación que plantea el modelo es perfecta.
% \end{itemize}
% 
% \note{Como hemos visto, la varianza residual tiene el problema de que es difícil de interpretar porque tiene como unidades las de la
% variable dependiente al cuadrado. Teniendo en cuenta, como hemos visto antes, que la varianza residual puede valer entre 0, cuando el
% ajueste del modelo de regresión es perfecto, hasta como máximo la varianza de $Y$ cuando entre las variables no hay relación del tipo que
% plantea el modelo de regresión, es posible calcular una nueva medida que no tenga unidades y sea más fácil de interpretar. Dicho
% estadísticos se llama coeficiente de regresión, se representa por $r^2$ y se define como 1 menos el cociente de la varianza residual entre
% la varianza de $Y$.
% 
% Como la varianza residual tiene las mismas unidades que la varianza de $Y$, al hacer el cociente las unidades se cancelarán por lo que el
% coeficiente de determinación es adimensinal. 
% 
% Cuando el ajuste del modelo sea perfecto, la varianza residual valdrá 0 y el cociente también, de manera que el coeficiente de determinación
% valdrá 1, mientras que en el otro extremo, cuano no haya relación entre las variables del tipo del modelo de regresión, la varianza residual
% coincidirá con la varianza de $Y$ y el cociente valdrá 1, con lo que el coeficiente de determinación valdrá 0.
% 
% Así pues, el coeficiente de determinación siempre oscila entre 0 y 1 y será mayor cuanto mejor explique el modelo de regresión la relación
% entre las variables. 
% 
% Si se multiplica por 100, el coeficiente de determinación nos dará el porcentaje de la variabilidad de la variable dependiente explicado
% por el modelo de regresión.}
% \end{frame}
% 
% 
% %---------------------------------------------------------------------slide----
% \begin{frame}
% \frametitle{Coeficiente de determinación lineal}
% En el caso de las rectas de regresión, la varianza residual vale
% \begin{align*}
% s_{ry}^2 & = \sum e_{ij}^2f_{ij} = \sum (y_j - f(x_i))^2f_{ij} = \sum \left(y_j - \bar y -\frac{s_{xy}}{s_x^2}(x_i-\bar x) \right)^2f_{ij}=\\
% & = \sum \left((y_j - \bar y)^2 +\frac{s_{xy}^2}{s_x^4}(x_i-\bar x)^2 - 2\frac{s_{xy}}{s_x^2}(x_i-\bar x)(y_j -\bar y)\right)f_{ij} =\\ 
% & = \sum (y_j - \bar y)^2f_{ij} +\frac{s_{xy}^2}{s_x^4}\sum (x_i-\bar x)^2f_{ij}- 2\frac{s_{xy}}{s_x^2}\sum (x_i-\bar x)(y_j -\bar y)f_{ij}=\\
% & = s_y^2 + \frac{s_{xy}^2}{s_x^4}s_x^2 - 2 \frac{s_{xy}}{s_x^2}s_{xy} = s_y^2 - \frac{s_{xy}^2}{s_x^2}.
% \end{align*}
% y, por tanto, el coeficiente de determinación lineal vale
% \begin{align*}
% r^2 &= 1- \frac{s_{ry}^2}{s_y^2} = 1- \frac{s_y^2 - \frac{s_{xy}^2}{s_x^2}}{s_y^2} = 1 - 1 + \frac{s_{xy}^2}{s_x^2s_y^2} = \frac{s_{xy}^2}{s_x^2s_y^2}.
% \end{align*}
% 
% \note{En el caso del ajuste de un modelo lineal, teniendo en cuenta la ecuación de la recta de regresión de $Y$ sobre $X$, los residuos
% serán $y_j$ menos el valor de la recta de regresión para $x_i$, es decir $y_j - \bar y -\frac{s_{xy}}{s_x^2}(x_i-\bar x)$. Si en la fórmula
% de la varianza residual desarrollamos el cuadrado de estos residuos, y simplificamos, llegamos a que la varianza residual para el modelo
% lineal se puede calcular como la varianza de $Y$ menos la covarianza al cuadrado entre la varianza de $X$. Y si sustituimos en la fórmula
% del coeficiente de determinación, obtenemos que el coeficiente de determinación lineal puede calcularse como la covarianza al cuadrado entre
% el producto de las varianzas de $X$ e $Y$.
% }
% \end{frame}
% 
% 
% %---------------------------------------------------------------------slide----
% \begin{frame}
% \frametitle{Cálculo del coeficiente de determinación lineal}
% \framesubtitle{Ejemplo de estaturas y pesos}
% En el ejemplo de las estaturas y pesos se tenía
% \[
% \begin{array}{lll}
% \bar x = 174.67 \mbox{ cm} & \quad & s^2_x = 102.06 \mbox{ cm}^2\\
% \bar y = 69.67 \mbox{ Kg} & & s^2_y = 164.42 \mbox{ Kg}^2\\
% s_{xy} = 104.07 \mbox{ cm$\cdot$ Kg}
% \end{array}
% \]
% De modo que el coeficiente de determinación lineal vale
% \[
% r^2 = \frac{s_{xy}^2}{s_x^2s_y^2} = \frac{(104.07 \mbox{ cm\cdot Kg})^2}{102.06 \mbox{ cm}^2 \cdot 164.42 \mbox{ Kg}^2} = 0.65.
% \]
% Esto indica que la recta de regresión del peso sobre la estatura explica el 65\% de la variabilidad del peso, y de igual modo, la recta de regresión de la estatura sobre el peso explica el  65\% de la variabilidad de la estatura.
% \note{En el ejemplo de las estaturas y los pesos, como la covarianza valía $104.7$ cm por kg, la varianza de la estatura era $102.06$ kg$^2$
% y la varianza del peso era $164.42$ kg$^2$, el coeficiente de determinación lineal valdrá 
% \[
% r^2 = \frac{(104.07 \mbox{ cm\cdot Kg})^2}{102.06 \mbox{ cm}^2 \cdot 164.42 \mbox{ Kg}^2} = 0.65,
% \]
% lo que indica que la recta de regresión del peso sobre la estatura explica el 65\% de la variabilidad del peso, y de igual modo, la recta de
% regresión de la estatura sobre el peso explica el  65\% de la variabilidad de la estatura.
% }
% \end{frame}
% 
% 
% %---------------------------------------------------------------------slide----
% \begin{frame}
% \frametitle{Coeficiente de correlación lineal}
% \begin{definicion}[Coeficiente de correlación lineal]
% Dada una variable bidimensional $(X,Y)$, el \emph{coeficiente de correlación lineal muestral} es
% la raíz cuadrada de su coeficiente de determinación lineal, con signo el de la covarianza
% \[
% r = \sqrt{r^2} = \dfrac{s_{xy}}{s_xs_y} $. 
% \]
% \end{definicion}
% Como $r^2$ toma valores entre 0 y 1, el coeficiente de correlación lineal tomará valores entre -1 y 1:
% \[
% \alert{-1\leq r\leq 1}
% \]
% El coeficiente de correlación lineal también mide el grado de dependencia lineal:
% \begin{itemize}
% \item Si $r =0$ entonces no existe relación lineal.
% \item Si $r=1$ entonces existe una relación lineal creciente perfecta.
% \item Si $r=-1$ entonces existe una relación lineal decrececiente perfecta.
% \end{itemize}
% 
% \note{Otra media habitual del grado de relación lineal entre dos variables es el coeficiente de correlación lineal, es la raíz cuadrada del
% coeficiente de determinación lineal y por tanto se calcula dividiendo la covarianza entre el producto de las desviaciones típicas de $X$ e
% $Y$.
% 
% Como el coeficiente de determinación podía tomar valores entre 0 y 1, el coeficiente de correlación lineal tomará valores entre -1 y 1, de
% manera que un cuando no haya relación lineal entre las variables valdrá 0, cuando haya una relación lineal creciente perfecta valdrá 1
% y cuado haya una relación lineal decreciente perfecta valdrá -1. 
% }
% \end{frame}
% 
% 
% %---------------------------------------------------------------------slide----
% \begin{frame}
% \frametitle{Coeficiente de correlación lineal}
% \framesubtitle{Ejemplo}
% En el ejemplo de las estaturas y los pesos, el coeficiente de correlación lineal vale
% \[
% r = \frac{s_{xy}}{s_xs_y} = \frac{104.07 \mbox{ cm\cdot Kg}}{10.1 \mbox{ cm} \cdot 12.82 \mbox{ Kg}} = +0.8.
% \]
% lo que indica que la relación lineal entre el peso y la estatura es fuerte, y además creciente.
% 
% \note{
% Por ejemplo, en el caso de las estaturas y los pesos, el coeficiente de correlación lineal vale 
% \[
% r = \frac{104.07 \mbox{ cm\cdot Kg}}{10.1 \mbox{ cm} \cdot 12.82 \mbox{ Kg}} = +0.8.
% \]
% lo que indica que la relación lineal entre el peso y la estatura es fuerte, y además creciente.
% }
% \end{frame}
% 
% 
% %---------------------------------------------------------------------slide----
% \begin{frame}
% \frametitle{Fiabilidad de las predicciones de un modelo de regresión}
% Aunque el coeficiente de determinación o el de correlación hablan de la bondad de un modelo de regresión, no es lo único que hay que
% tener en cuenta a la hora de hacer predicciones.
% 
% La fiabilidad de las predicciones que hagamos con un modelo de regresión depende de varias cosas:
% \begin{itemize}
% \item El coeficiente de determinación: Cuanto mayor sea, menores serán los errores predictivos y mayor la fiabilidad de las predicciones.
% \item La variabilidad de la población: Cuanto más variable es una población, más difícil es predecir y por tanto menos fiables serán las
% predicciones.
% \item El tamaño muestral: Cuanto mayor sea, más información tendremos y, en consecuencia, más fiables serán las predicciones. 
% \end{itemize} 
% 
% Además, hay que tener en cuenta que un modelo de regresión es válido únicamente para el rango de valores observados en la muestra.
% Fuera de ese rango no hay información del tipo de relación entre las variables, por lo que no deben hacerse predicciones para valores lejos
% de los observados en la muestra.
% 
% \note{Aunque el coeficiente de determinación o el de correlación hablan de la bondad de un modelo de regresión, no es el único dato que hay que
% tener en cuenta a la hora de hacer predicciones.
% 
% La fiabilidad de las predicciones que hagamos con un modelo de regresión depende de varios factores como son:
% \begin{itemize}
% \item El coeficiente de determinación: Cuanto mayor sea, menores serán los errores predictivos y mayor la fiabilidad de las predicciones.
% \item La variabilidad de la población: Cuanto más variable es una población, más difícil es predecir y por tanto menos fiables serán las predicciones del modelo.
% \item El tamaño muestral: Cuanto mayor sea, más información tendremos y, en consecuencia, más fiables serán las predicciones. 
% \end{itemize} 
% 
% Además, hay que tener en cuenta que un modelo de regresión es válido únicamente para el rango de valores observados en la muestra.
% Fuera de ese rango, al no disponer de información del tipo de relación entre las variables, no deben hacerse predicciones.
% }
% \end{frame}
% 
% 
% \subsection{Regresión no lineal}
% %---------------------------------------------------------------------slide----
% \begin{frame}
% \frametitle{Regresión no lineal}
% El ajuste de un modelo de regresión no lineal es similar al del modelo lineal y también puede realizarse mediante la técnica de mínimos
% cuadrados.
%  
% No obstante, en determinados casos un ajuste no lineal puede convertirse en un ajuste lineal mediante una sencilla transformación de alguna
% de las variables del modelo.
% 
% \note{Ya hemos visto cómo se construyen las rectas de regresión. Para la construcción de los modelos de regresión no lineales el
% prodecidimiento es similar mediante el método de los mínimos cuadrados, aunque en determinados, transformando alguna de las variables del
% modelo, es posible convertir el modelo no lineal en uno lineal que ya sabemos construir.
% }
% \end{frame}
% 
% 
% %---------------------------------------------------------------------slide----
% \begin{frame}
% \frametitle{Transformación de modelos de regresión no lineales}
% \begin{itemize}
% \item \highlight{Modelo logarítmico}}: Un modelo logarítmico $y = a+b \log x$ se convierte en un modelo lineal haciendo el cambio
% $t=\log x$:
% \[y=a+b\log x = a+bt.\]
% \item \highlight{Modelo exponencial}}: Un modelo exponencial $y = ae^{bx}$ se convierte en un modelo lineal haciendo el cambio $z = \log y$:
% \[z = \log y = \log(ae^{bx}) =  \log a + \log e^{bx} = a^\prime +bx. \] 
% \item \highlight{Modelo potencial}}: Un modelo potencial $y = ax^b$ se convierte en un modelo lineal haciendo los cambios
% $t=\log x$ y $z=\log y$:
% \[z = \log y = \log(ax^b) = \log a + b \log x = a^\prime+bt.\]
% \item \highlight{Modelo inverso}}: Un modelo inverso $y = a+b/x$ se convierte en un modelo lineal haciendo el cambio $t=1/x$:
% \[y = a + b(1/x) = a+bt.\]
% \item \highlight{Modelo curva S}}: Un modelo curva S $y = e^{a+b/x}$ se convierte en un modelo lineal haciendo los cambios $t=1/x$
% y $z=\log y$:
% \[z = \log y = \log (e^{a+b/x}) = a+b(1/x) = a+bt.\]
% \end{itemize}
% 
% \note{
% \begin{itemize}
% \item El modelo logarítmico, $y = a+b \log x$, que se transforma en un modelo lineal aplicando el logaritmo a la variable independiente
% $t=\log x$.
% \item El modelo exponencial, $y = ae^{bx}$, que se convierte en lineal aplicando el logaritmo a la variable dependiente $z=\log y$.
% \item El modelo potencial $y = ax^b$ se convierte en un modelo lineal aplicando el logaritmo tanto a la variable independiente
% $t=\log x$ como a la variable dependiente $z=\log y$.
% \item El modelo inverso $y = a+b/x$ se convierte en un modelo lineal tomando el inverso de la variable independiente $t=1/x$.
% \item El modelo curva S $y = e^{a+b/x}$ se convierte en un modelo lineal tomando el inverso de la variable independiente $t=1/x$ y el
% logaritmo de la variable dependiente $z=\log y$.
% \end{itemize}
% }
% \end{frame}
% 
% 
% %---------------------------------------------------------------------slide----
% \begin{frame}
% \frametitle{Ejemplo de ajuste de un modelo exponencial}
% \framesubtitle{Evolución del número de bacterias de un cultivo}
% El número de bacterias de un cultivo evoluciona con el tiempo según la siguiente tabla:
% \begin{columns}
% \begin{column}{0.3\textwidth}
% \[
% \begin{array}{c|c}
% \mbox{Horas} & \mbox{Bacterias}\\
% \hline
% 0 &  25 \\
% 1 & 28 \\
% 2 &  47\\
% 3 & 65 \\
% 4 & 86\\
% 5 & 121\\
% 6 & 190\\
% 7 & 290\\
% 8 & 362
% \end{array}
% \]
% \end{column}
% \begin{column}{0.6\textwidth}
% El diagrama de dispersión asociado es
% \begin{center}
% \scalebox{0.6}{\input{img/regresion/diagrama_dispersion_horas_bacterias}}
% \end{center}
% \end{column}
% \end{columns}
% 
% \note{Veamos un ejemplo de construcción de un modelo exponencial. En esta muestra se presenta la evolución del número de bacterias en un
% cultivo, en función del tiempo. 
% 
% Si dibujamos el diagrama de dispersión asociado podemos observar que hay una clara relación exponencial entre el número de bacterias y el
% tiempo transcurrido.
% }
% \end{frame}
% 
% 
% %---------------------------------------------------------------------slide----
% \begin{frame}
% \frametitle{Ejemplo de ajuste de un modelo exponencial}
% \framesubtitle{Evolución del número de bacterias de un cultivo}
% Si realizamos un ajuste lineal, obtenemos la siguiente recta de regresión
% \begin{columns}
% \begin{column}{0.3\textwidth}
% \[
% \begin{array}{c|c}
% \mbox{Horas} & \mbox{Bacterias}\\
% \hline
% 0 &  25 \\
% 1 & 28 \\
% 2 &  47\\
% 3 & 65 \\
% 4 & 86\\
% 5 & 121\\
% 6 & 190\\
% 7 & 290\\
% 8 & 362
% \end{array}
% \]
% \[
% \mbox{Bacterias} = -30.18+41,27\,\mbox{Horas}
% \]
% \end{column}
% \begin{column}{0.6\textwidth}
% \begin{center}
% \scalebox{0.6}{\input{img/regresion/regresion_lineal_horas_bacterias}}
% \end{center}
% \end{column}
% \end{columns}
% \begin{center}
% \emph{¿Es un buen modelo?}
% \end{center}
% 
% \note{Si se calcula la recta de regresión del número de bacterias sobre las horas transcurridas, se observa que no es un mal modelo ya que
% su coefiente de determinación lineal vale $0.85$. Ahora bien, que sea un buen modelo, no quiere decir que sea el mejor.
% }
% \end{frame}
% 
% 
% %---------------------------------------------------------------------slide----
% \begin{frame}
% \frametitle{Ejemplo de ajuste de un modelo exponencial}
% \framesubtitle{Evolución del número de bacterias de un cultivo}
% Aunque el modelo lineal no es malo, de acuerdo al diagrama de dispersión es más lógico construir un modelo exponencial o cuadrático.
% 
% 
% Para construir el modelo exponencial $y = ae^{bx}$ hay que realizar la transformación $z=\log y$, es decir, aplicar el logaritmo a la
% variable dependiente.
% 
% \begin{columns}
% \begin{column}{0.45\textwidth}
% \[
% \begin{array}{c|c|c}
% \mbox{Horas} & \mbox{Bacterias} & \mbox{Log Bacterias}\\
% \hline
% 0 &  25 & 3.22\\
% 1 & 28 & 3.33\\
% 2 &  47 & 3.85\\
% 3 & 65  & 4.17\\
% 4 & 86 & 4.45\\
% 5 & 121 & 4.80\\
% 6 & 190 & 5.25\\
% 7 & 290 & 5.67\\
% 8 & 362 & 5.89
% \end{array}
% \]
% \end{column}
% \begin{column}{0.55\textwidth}
% \begin{center}
% \scalebox{0.58}{\input{img/regresion/diagrama_dispersion_horas_logbacterias}}
% \end{center}
% \end{column}
% \end{columns}
% 
% \note{Para construir el modelo de regresión exponencial hay que aplicar el logaritmo neperiano a la variable dependiente, que en este caso
% es el número de bacterias. El logaritmo neperiano de 25 nos da 3.22, el de 28 nos da 3.33 y así sucesivamente, de manera que obtenemos una
% nueva variable que es el logarítmo de las bacterias. 
% 
% Si representamos el diagrama de dispersion del logaritmo de las bacterias y las horas transcurridas, podemos comprobar cómo ahora la nube de
% puntos tiene forma lineal, es decir, mediante la transformación logarítmica de la variable dependiente hemos convertido una relación
% exponencial en una relación lineal.
% }
% \end{frame}
% 
% 
% %---------------------------------------------------------------------slide----
% \begin{frame}
% \frametitle{Ejemplo de ajuste de un modelo exponencial}
% \framesubtitle{Evolución del número de bacterias de un cultivo}
% Ahora sólo queda calcular la recta de regresión del logaritmo de Bacterias sobre Horas
% \begin{columns}
% \begin{column}{0.45\textwidth}
% \[
% \mbox{Log Bacterias} = 3.107 + 0.352\, \mbox{Horas}.
% \]
% Y deshaciendo el cambio de variable, se obtiene el modelo exponencial
% \[
% \mbox{Bacterias} = e^{3.107+0.352\,\textrm{Horas}},
% \]
% que, a la vista del coeficiente de determinación, es mucho mejor modelo que el lineal.
% \end{column}
% \begin{column}{0.55\textwidth}
% \begin{center}
% \scalebox{0.6}{\input{img/regresion/regresion_exponencial_horas_bacterias}}
% \end{center}
% \end{column}
% \end{columns}
% 
% \note{Si ahora calculamos la recta de regresión del logaritmo de las bacterias sobre las horas, se tiene $\mbox{Log Bacterias} = 3.107 +
% 0.352\, \mbox{Horas}$, cuyo coeficiente de determinación lineal vale $0.99$, lo cual indica un ajuste casi perfecto de la recta. 
% 
% Finalmente, para obtener el modelo exponencial, basta con desacer la transformación logarítmica del número de bacterias, aplicando la
% función exponencial, lo que nos da $\mbox{Bacterias} = e^{3.107+0.352\,\textrm{Horas}}$, y cuyo coeficiente de determinación exponencial
% coincide con el de la recta anterior, de manera que se puede concluir que el modelo exponencial explica muyo mejor la relación entre el
% número de bacterias y las horas que el modelo lineal. }
% \end{frame}
% 
% 
% %---------------------------------------------------------------------slide----
% \begin{frame}
% \frametitle{Interpretación de un coeficiente de determinación pequeño}
% Tanto el coeficiente de determinación como el de correlación hacen referencia a un modelo concreto, de manera que un coeficiente $r^2=0$
% significa que no existe relación entre las variables del tipo planteado por el modelo, pero \emph{eso no quiere decir que las variables sean
% independientes}, ya que puede existir relación de otro tipo.
% \begin{center}
% \scalebox{0.55}{\input{img/regresion/recta_regresion_relacion_parabolica}}
% \qquad
% \scalebox{0.55}{\input{img/regresion/regresion_parabolica}}
% \end{center}
% 
% \note{Hay que recordar que tanto el coeficiente de determinación hace referencia a un modelo concreto, de manera que
% un coeficiente $r^2=0$ significa que no existe relación entre las variables del tipo planteado por el modelo, pero \emph{eso no quiere decir
% que las variables sean independientes}, ya que puede existir relación de otro tipo.
% 
% En las gráficas puede apreciarse cómo al ajustar una recta sobre esta nube de puntos se obtiene un coeficiente de determinación
% prácticamente nulo pero sin embargo hay una clara relación parabólica entre las variables, por lo que no podemos decir que las variables
% son independientes, tan sólo que no tienen relación lineal.}
% \end{frame}
% 
% 
% %---------------------------------------------------------------------slide----
% \begin{frame}
% \frametitle{Datos atípicos en regresión}
% En un estudio de regresión es posible que aparezca algún individuo que se aleja notablemente de la tendencia del resto de individuos en la
% nube de puntos.
% 
% Aunque el individuo podría no ser un \emph{dato atípico} al considerar las variables de manera separada, sí lo sería al
% considerarlas de manera conjunta.
% 
% \begin{center}
% \scalebox{0.6}{\input{img/regresion/diagrama_dispersion_con_datos_atipicos}}
% \end{center}
% 
% \note{Otro de los problemas que pueden plantearse en la construcción de un modelo de regresión es la presencia de datos atípicos. En un
% estudio de regresión, los datos atípicos, a diferencia de lo que vimos en el tema anterior no son individuos que presentan un valor en
% alguna de las variables muy distinto del resto, sino que son puntos que se alejan notablemente de la tendencia de la nube de puntos.
% 
% Como se aprecia en el gráfico, aunque un individuo podría no ser un \emph{dato atípico} al considerar las variables de manera separada, sí
% lo sería al considerarlas de manera conjunta.}
% \end{frame}
% 
% 
% %---------------------------------------------------------------------slide----
% \begin{frame}
% \frametitle{Influencia de los datos atípicos en los modelos de regresión}
% Los datos atípicos en regresión suelen provocar cambios drásticos en el ajuste de los modelos de regresión, y por tanto, habrá que tener
% mucho cuidado con ellos.
% 
% \begin{center}
% \scalebox{0.55}{\input{img/regresion/recta_regresion_con_datos_atipicos}}
% \qquad
% \scalebox{0.55}{\input{img/regresion/recta_regresion_sin_datos_atipicos}}
% \end{center}
% 
% \note{Los datos atípicos en un estudio de regresión son un problema porque suelen provocar cambios drásticos en el modelo de regresión.
% 
% Como puede apreciarse en estos dos diagramas de dispersión, hay una clara tendencia lineal de la nube de puntos, pero en el primer caso hay
% un dato atípico. Si se calcula la recta de regresión con el dato atípico, la recta resultante aparece bastante desviada de la tendencia de
% la nube de puntos ya que el dato atípico está ejerciendo una gran influencia sobre ella. Como consecuencia el modelo resultante tiene
% coeficiente de determinación $0.08$ y lo que indica que esta recta no explica bien la relación entre las variables. Sin embargo, cuando se
% calcula la recta sin tener en cuenta el dato atípico, la recta resultante tiene un coeficiente de determinación de $0.98$ y explica casi a
% la perfección la relación que hay entre las variables. 
% 
% Así pues, cuando haya datos atípicos en un estudio de regresión habrá que tener mucho cuidado con ellos y muchas ocasiones se eliminarán
% para que no distorsionen el modelo de regresión.}
% \end{frame}
% 
% 
% \subsection{Medidas de relación entre atributos}
% %---------------------------------------------------------------------slide----
% \begin{frame}
% \frametitle{Relaciones entre atributos}
% Los modelos de regresión vistos sólo pueden aplicarse cuando las variables estudiadas son cuantitativas.
% 
% Cuando se desea estudiar la relación entre atributos, tanto ordinales como nominales, es necesario recurrir a otro tipo
% de medidas de relación o de asociación. En este tema veremos tres de ellas:
% \begin{itemize}
% \item Coeficiente de correlación de Spearman.
% \item Coeficiente chi-cuadrado.
% \item Coeficiente de contingencia.
% \end{itemize}
% 
% \note{La regresión sólo tiene sentido cuando cuando las variables estudiadas son cuantitativas.
% 
% Cuando se desea estudiar la relación entre atributos, tanto ordinales como nominales, es necesario recurrir a otro tipo
% de medidas de relación o de asociación. Veremos tres de las más importantes que son
% \begin{itemize}
% \item Coeficiente de correlación de Spearman.
% \item Coeficiente chi-cuadrado.
% \item Coeficiente de contingencia.
% \end{itemize}
% }
% \end{frame}
% 
% 
% %---------------------------------------------------------------------slide----
% \begin{frame}
% \frametitle{Coeficiente de correlación de Spearman}
% Cuando se tengan atributos ordinales es posible ordenar sus categorías y asignarles valores ordinales, de manera que se
% puede calcular el coeficiente de correlación lineal entre estos valores ordinales. 
% 
% Esta medida de relación entre el orden que ocupan las categorías de dos atributos ordinales se conoce como coeficiente
% ce correlación de Spearman, y puede demostrarse fácilmente que puede calcularse a partir de la siguiente fórmula
% 
% \begin{definicion}[Coeficiente de correlación de Spearman]
% Dada una muestra de $n$ individuos en los que se han medido dos atributos ordinales $X$ e $Y$, el coeficiente de
% correlación de Spearman se define como: 
% \[
% r_s = 1-\frac{6\sum d_i^2}{n(n^2-1)}
% \]
% donde $d_i$ es la diferencia entre el valor ordinal de $X$ y el valor ordinal de $Y$ del individuo $i$.
% \end{definicion}
% 
% \note{Si queremos estudiar la relación entre dos atributos ordinales, como sus categorías pueden ordenarse, una posibilidad tomar sus
% valores de orden, que serían numéricos, y calcular el coeficiente de correlación lineal entre los órdenes.
% 
% Esta medida se conoce como coeficiente de correlación de Spearman, que se representa como $r_s$ y también puede calcularse haciendo la suma
% de los cuadrados de las diferencias entre los números de orden del valor de $X$ e $Y$ en cada indiviudo, multiplicada por 6, dividida por el
% tamaño de la muestra multiplicado por el cuadrado del tamaño de la muestra menos 1 y restando el resultado del cociente a 1.
% }
% \end{frame}
% 
% 
% %---------------------------------------------------------------------slide----
% \begin{frame}
% \frametitle{Interpretación del coeficiente de correlación de Spearman}
% Como el coeficiente de correlación de Spearman es en el fondo el coeficiente de correlación lineal aplicado a los
% órdenes, se tiene:
% \[
% -1\leq r_s\leq 1,
% \]
% de manera que:
% \begin{itemize}
% \item Si $r_s=0$ entonces no existe relación entre los atributos ordinales.
% \item Si $r_s=1$ entonces los órdenes de los atributos coinciden y existe una relación directa perfercta.
% \item Si $r_s=-1$ entonces los órdenes de los atributos están invertidos y existe una relación inversa perfecta.
% \end{itemize}
% En general, cuanto más cerca de $1$ o $-1$ esté $r_s$, mayor será la relación entre los atributos, y cuanto más cerca
% de $0$, menor será la relación.
% 
% \note{Como el coeficiente de correlación de Spearman es en el fondo el coeficiente de correlación lineal aplicado a los
% órdenes, su valor estará entre -1 y 1 y se interpreta de manera similar al coeficiente de correlación lineal, es decir,
% \begin{itemize}
% \item Si $r_s=0$ entonces no existe relación entre los atributos ordinales.
% \item Si $r_s=1$ entonces los órdenes de los atributos coinciden y existe una relación directa perfercta.
% \item Si $r_s=-1$ entonces los órdenes de los atributos están invertidos y existe una relación inversa perfecta.
% \end{itemize}
% }
% \end{frame}
% 
% 
% %---------------------------------------------------------------------slide----
% \begin{frame}
% \frametitle{Cálculo del coeficiente de correlación de Spearman}
% \framesubtitle{Ejemplo}
% Una muestra de 5 alumnos realizaron dos tareas diferentes $X$ e $Y$, y se ordenaron de acuerdo a la destreza que
% manifestaron en cada tarea:
% \[
% \begin{array}{lrrrr}
% \hline
% \text{Alumnos} & X & Y & d_i & d_i^2\\
% \hline
% \text{Alumno 1} & 2 & 3 & -1 & 1\\
% \text{Alumno 2} & 5 & 4 & 1 & 1 \\
% \text{Alumno 3} & 1 & 2 & -1 & 1\\
% \text{Alumno 4} & 3 & 1 & 2 & 4\\
% \text{Alumno 5} & 4 & 5 & -1 & 1\\
% \hline
% \sum &  &  & 0 & 8 \\
% \hline
% \end{array}
% \]
% El coeficiente de correlación de Spearman para esta muestra es 
% \[
% r_s = 1-\frac{6\sum d_i^2}{n(n^2-1)} = 1- \frac{6\cdot 8}{5(5^2-1)} = 0.6,
% \]
% lo que indica que existe bastante relación directa entre las destrezas manifestadas en ambas tareas.
% 
% \note{Para ilustrar el cálculo del coeficiente de correlación de Spearman, supongamos una muestra de 5 alumnos que realizaron dos tareas
% diferentes $X$ e $Y$, y se ordenaron de acuerdo a la destreza que manifestaron en cada tarea.
% 
% El primer alumno fue el segundo con más destreza en la tarea $X$ y el tercero en la tarea $Y$ de manera que la diferencia entre los órdenes
% en $X$ e $Y$ es $-1$ y su cuadrado vale 1. El segundo alumno fue el peor en la tarea $X$ y el penúltimo en la tarea $Y$ de manera que la
% diferencia entre sus órdenes en $X$ e $Y$ vale también $-1$ y su cuadrado 1, etc.  
% 
% Una vez calculados los cuadrados de las diferencias entre los órdenes, se suman, se multiplican por 6 y se dividen por el cuadrado del
% tamaño de la muestra, que vale 5,  por el cuadrado del tamaño de la muestra menos, que es $5^2-1$ y finalmente el resultado del cociente se
% resta a 1. Esto nos da un coeficiente de correlación de Spearman de 0.6, lo cual indica que existe bastante relación directa entre las
% destrezas manifestadas en ambas tareas.
% }
% \end{frame}
% 
% 
% %---------------------------------------------------------------------slide----
% \begin{frame}
% \frametitle{Cálculo del coeficiente de correlación de Spearman}
% \framesubtitle{Ejemplo con empates}
% Cuando hay empates en el orden de las categorías se atribuye a cada valor empatado la media aritmética de los valores
% ordinales que hubieran ocupado esos individuos en caso de no haber estado empatados.
% 
% Si en el ejemplo anterior los alumnos 4 y 5 se hubiesen comportado igual en la primera tarea y los alumnos 3 y 4 se
% hubiesen comportado igual en la segunda tarea, entonces se tendría
% \[
% \begin{array}{lrrrr}
% \hline
% \text{Alumnos} & X & Y & d_i & d_i^2\\
% \hline
% \text{Alumno 1} & 2 & 3 & -1 & 1\\
% \text{Alumno 2} & 5 & 4 & 1 & 1 \\
% \text{Alumno 3} & 1 & 1.5 & -0.5 & 0.25\\
% \text{Alumno 4} & 3.5 & 1.5 & 2 & 4\\
% \text{Alumno 5} & 3.5 & 5 & -1.5 & 2.25\\
% \hline
% \sum &  &  & 0 & 8.5 \\
% \hline
% \end{array}
% \]
% El coeficiente de correlación de Spearman para esta muestra es 
% \[
% r_s = 1-\frac{6\sum d_i^2}{n(n^2-1)} = 1- \frac{6\cdot 8.5}{5(5^2-1)} = 0.58.
% \]
% 
% \note{Cuando haya individuos con la misma categoría del atributo, entonces dichos individuos recibirán el mismo número de orden y de manera
% que habrá empates en el orden. En tal caso se atribuye a cada valor empatado la media aritmética de los valores ordinales que hubieran
% ocupado esos individuos en caso de no haber estado empatados.
% 
% Si en ejemplo anterior los alumnos 4 y 5 se hubiesen comportado igual en la primera tarea, entonces como ambos ocupan la posición 3 y 4
% en la ordenación, se toma la media $3.5$ y se le asigna dicho valor a ambos. Del mismo modo, si los alumnos 3 y 4 se hubiesen comportado
% igual en la segunda tarea, como el orden que ocupan es el primero y el segundo, entonces a ambos se le asigna la media $1.5$.
% 
% El resto de los cálculos para el coeficiente de Spearman se haría igual que en el caso en que no había empates.
% }
% \end{frame}
% 
% 
% %---------------------------------------------------------------------slide----
% \begin{frame}
% \frametitle{Relación entre atributos nominales}
% Cuando se quiere estudiar la relación entre atributos nominales no tiene sentido calcular el coeficiente de correlación
% de Spearman ya que las categorías no pueden ordenarse.
% 
% Para estudiar la relación entre atributos nominales se utilizan medidas basadas en las frecuencias de la tabla de
% frecuencias bidimensional, que para atributos se suele llamar \emph{tabla de contingencia}.
% 
% \highlight{Ejemplo}} En un estudio para ver si existe relación entre el sexo y el hábito de fumar se ha tomado
% una muestra de 100 personas.
% La tabla de contingencia resultante es
% \[
% \begin{array}{|l|rr|r|} 
% \hline
% \text{Sexo}\backslash\text{Fuma} & \text{Si} & \text{No} & n_i\\
% \hline
% \text{Mujer} & 12 & 28 & 40 \\
% \text{Hombre} & 26 & 34 & 60 \\
% \hline
% n_j & 38 & 62 & 100\\
% \hline
% \end{array}
% \]
% Si el hábito de fumar fuese independiente del sexo, la proporción de fumadores en mujeres y hombres sería la misma.
% 
% \note{Cuando se quiere estudiar la relación entre atributos nominales no tiene sentido calcular el coeficiente de correlación
% de Spearman ya que las categorías no pueden ordenarse. En este caso se utilizan medidas basadas en las frecuencias de la tabla de
% frecuencias bidimensional, que para atributos se suele llamar \emph{tabla de contingencia}.
% 
% En este ejemplo tenemos una tabla de contingencia para los atributos sexo y hábito de fumar, ambos nominales. En la muestra hay 12 mujeres
% fumadoras, 28 no fumadoras, 26 hombres fumadores y 34 no fumadores. En total tenemos una muestra de tamaño 100. 
% 
% La forma de estudiar la relación entre el sexo y el hábito de fumar es por medio de las frecuencias relativas de cada par de categorías.
% Así, si el hábito de fumar fuese independiente del sexo, la proporción de fumadores en mujeres y hombres sería la misma, mientras que si
% hubiese dependencia, serían significativamente diferentes.
% }
% \end{frame}
% 
% 
% %---------------------------------------------------------------------slide----
% \begin{frame}
% \frametitle{Frecuencias teóricas o esperadas}
% En general, dada una tabla de contingencia para dos atributos $X$ e $Y$, 
% \[
% \begin{array}{|c|ccccc|c|}
% \hline
% X\backslash Y & y_1 & \cdots & y_j & \cdots & y_q & n_x\\
% \hline
% x_1 & n_{11} & \cdots & n_{1j} & \cdots & n_{1q} & n_{x_1}\\
% \vdots & \vdots & \ddots & \vdots & \ddots & \vdots & \vdots \\
% x_i & n_{i1} & \cdots & n_{ij} & \cdots & n_{iq} & n_{x_i}\\
% \vdots & \vdots & \ddots & \vdots & \ddots & \vdots & \vdots\\
% x_p & n_{p1} & \cdots & n_{pj} & \cdots & n_{pq} & n_{x_p} \\
% \hline
% n_y & n_{y_1} & \cdots & n_{y_j} & \cdots & n_{y_q} & n\\
% \hline
% \end{array}
% \]
% si $X$ e $Y$ fuesen independientes, para cualquier valor $y_j$ se tendría
% \[
% \frac{n_{1j}}{n_{x_1}} = \frac{n_{2j}}{n_{x_2}} = \cdots = \frac{n_{pj}}{n_{x_p}} = \frac{n_{1j}+\cdots
% +n_{pj}}{n_{x_1}+\cdots+n_{x_p}} = \frac{n_{y_j}}{n},
% \]
% de donde se deduce que
% \[
% n_{ij} = \frac{n_{x_i}n_{y_j}}{n}.
% \]
% 
% A esta última expresión se le llama \emph{frecuencia teórica} o \emph{frecuencia esperada} del par $(x_i,y_j)$. 
% 
% \note{En general, dada una tabla de contingencia pra dos atributos $X$ e $Y$, si las variables fuesen independientes, entonces la proporción
% de individuos que presentan el valor $y_j$ en la variable $Y$ entre los que presentan el valor $x_1$ en la variable $X$ sería la frecuencia
% absoluta del par $(x_i,y_j)$ $n_{ij}$ divida entre la frecuencia absoluta de $x_1$ y esta frecuencia sería la misma que la proporción de
% individuos con el valor $y_j$ en $Y$ entre los que presentan el valor $x_2$ en $X$, y en general, al ser independiente de $X$ sería la misma
% para cualquier otra categoría de $X$, de manera que todas serían iguales a la proporción total de individuos en la muestra con valor $y_j$
% en la variable $Y$, que se obtendría dividiendo la frecuencia absoluta de $y_j$ entre el tamaño de la muestra.
% 
% De esta igualdad se puede deducir que la frecuencia absoluta de cualquier par $(x_i,y_j)$ se podría calcular multiplicando las frecuencias
% absolutas de $x_i$, $n_i$ y $y_j$ que es $n_j$ y dividendo por el tamaño de la muestra.
% 
% A estas frecuencias calculadas bajo la hipótesis de independencia, se les llama \emph{frecuencia teóricas} o \emph{frecuencia esperadas}.
% }
% \end{frame}
% 
% 
% %---------------------------------------------------------------------slide----
% \begin{frame}
% \frametitle{Coeficiente chi-cuadrado $\chi^2$}
% Es posible estudiar la relación entre dos atributos $X$ e $Y$ comparando las frecuencias reales con las esperadas:
% \begin{definicion}[Coeficiente chi-cuadrado $\chi^2$]
% Dada una muestra de tamaño $n$ en la que se han medido dos atributos $X$ e $Y$, se define el coeficiente $\chi^2$ como
% \[
% \chi^2 = \sum_{i=1}^p\sum_{j=1}^q \frac{\left(n_{ij}-\frac{n_{x_i}n_{y_j}}{n}\right)^2}{\frac{n_{x_i}n_{y_j}}{n}},
% \]
% donde $p$ es el número de categorías de $X$ y $q$ el número de categorías de $Y$.
% \end{definicion}
% 
% Por ser suma de cuadrados, se cumple que 
% \[
% \chi^2 \geq 0,
% \]
% de manera que $\chi^2=0$ cuando los atributos son independientes, y crece a medida que aumenta la dependencia entre las variables.
% 
% \note{Una posible forma de estudiar la relación entre dos atributos nominales es comparar sus frecuencias absolutas reales con las esperadas
% bajo la hipótesis de indepencia. Esto es precisamente lo que hace un estadístico conocido como coeficiente chi-cuadrado, que se denota con
% la letra griega $\chi^2$ y se calcula sumando para cada par de la variable bidimensional los cuadrados de las diferencias entre las
% frecuencias reales y las frecuencias esperadas, dividos por las propias frecuencias esperadas.
% 
% Como se trata de una suma de cuadrados, el coeficiente chi-cuadrado siempre es positivo. Cuando los atributos sean independientes, las
% frecuencias reales coincidirán con las esperadas y todos los sumandos se anularán, de manera que el coeficiente chi-cuadrado valdrá cero. Y
% en la medida que cada vez haya más dependencia, las diferencias entre las frecuencias reales y las esperadas serán cada vez mayores de forma
% que el coeficiente chi-cuadrado será también mayor.
% }
% \end{frame}
% 
% 
% %---------------------------------------------------------------------slide----
% \begin{frame}
% \frametitle{Cálculo del coeficiente chi-cuadrado $\chi^2$}
% \framesubtitle{Ejemplo}
% Siguiendo con el ejemplo anterior, a partir de la tabla de contingencia 
% \[
% \begin{array}{|l|rr|r|} 
% \hline
% \text{Sexo}\backslash\text{Fuma} & \text{Si} & \text{No} & n_i\\
% \hline
% \text{Mujer} & 12 & 28 & 40 \\
% \text{Hombre} & 26 & 34 & 60 \\
% \hline
% n_j & 38 & 62 & 100\\
% \hline
% \end{array}
% \]
% se obtienen las siguientes frecuencias esperadas:
% \[
% \renewcommand{\arraystretch}{1.5}
% \begin{array}{|l|rr|r|} 
% \hline
% \text{Sexo\Fuma} & \text{Si} & \text{No} & n_i\\
% \hline
% \text{Mujer} & \frac{40\cdot 38}{100}=15.2 & \frac{40\cdot 62}{100}=24.8 & 40 \\
% \text{Hombre} & \frac{60\cdot 38}{100}=22.8 & \frac{60\cdot 62}{100}=37.2 & 60 \\
% \hline
% n_j & 38 & 62 & 100\\
% \hline
% \end{array}
% \]
% y el coeficiente $\chi^2$ vale
% \[
% \chi^2 = \frac{(12-15.2)^2}{15.2}+\frac{(28-24.8)^2}{24.8}+\frac{(26-22.8)^2}{22.8}+\frac{(34-37.2)^2}{37.2} = 1.81,
% \]
% lo que indica que no existe gran relación entre el sexo y el hábito de fumar.
% 
% \note{Siguiendo con el ejemplo del sexo y el hábito de fumar, a partir de la tabla de contingencia se calculan las frecuencias esperadas de
% cada par. La frecuencia esperada de mujeres fumadoras es el número de mujeres 40, por el número de personas fumadoras 38, dividido por el
% tamaño de la muestra 100, lo que da $15.2$. Etc. 
% 
% Una vez calculadas las frecuencias esperadas, para calcular el coeficiente chi-cuadrado vamos calculando los cuadrados de las diferencias
% entre las frecuencias reales y las esperadas, es decir, para las mujeres fumadoras sería $(12-15.2)^2$ y esto se divide por la frecuencia
% esperada $15.2$. Del mismo modo se calculan los sumandos para el resto de los pares y se realiza la suma, lo que nos da un coeficiente
% chi-cuadrado de $1.81$, que al ser un valor próximo a 0 indica que no hay gran relación entre el sexo y el hábito de fumar.
% }
% \end{frame}
% 
% 
% %---------------------------------------------------------------------slide----
% \begin{frame}
% \frametitle{Coeficiente de contingencia}
% El coeficiente $\chi^2$ depende del tamaño muestral, ya que al multiplicar por una constante las frecuencias de todas
% las casillas, su valor queda multiplicado por dicha constante, lo que podría llevarnos al equívoco de pensar que ha
% aumentado la relación, incluso cuando las proporciones se mantienen.
% En consecuencia el valor de $\chi^2$ no está acotado superiormente y resulta difícil de interpretar.
% 
% Para evitar estos problemas se suele utilizar el siguiente estadístico:
% \begin{definicion}[Coeficiente de contingencia]
% Dada una muestra de tamaño $n$ en la que se han medido dos atributos $X$ e $Y$, se define el \emph{coeficiente de
% contingencia} como 
% \[
% C = \sqrt{\frac{\chi^2}{\chi^2+n}}
% \]
% \end{definicion}
% 
% \note{El principal problema del coeficiente chi-cuadrado es que depende del tamaño muestral ya que se calcula a partir de las frecuencias
% absolutas, no de las relativas. De hecho, se puede comprobar que al multiplicar por una constante las frecuencias de todas
% las casillas de la tabla de contingencia, su valor queda multiplicado por dicha constante, lo que podría llevarnos al equívoco de pensar que
% ha aumentado la relación, incluso cuando las proporciones se mantienen. En consecuencia el valor de $\chi^2$ no está acotado superiormente y
% resulta difícil de interpretar.
% 
% Para evitar estos problemas suele utilizarse otro estadístico conocido como coeficiente de contingencia, que se denota $C$, y se calcula a
% partir del coeficiente chi-cuadrado haciendo la raíz del cociente entre el coeficiente-chi cuadrado entre el mismo coeficiente chi-cuadrado
% mas el tamaño de la muestra.
% }
% \end{frame}
% 
% 
% %---------------------------------------------------------------------slide----
% \begin{frame}
% \frametitle{Interpretación del coeficiente de contingencia}
% De la definición anterior se deduce que 
% \[
% 0\leq C\leq 1,
% \]
% de manera que cuando $C=0$ las variables son independientes, y crece a medida que aumenta la relación.
% 
% Aunque $C$ nunca puede llegar a valer 1, se puede demostrar que para tablas de contingencia con $k$ filas y $k$
% columnas, el valor máximo que puede alcanzar $C$ es $\sqrt{(k-1)/k}$.
% 
% \highlight{Ejemplo}} En el ejemplo anterior el coeficiente de contingencia vale
% \[
% C = \sqrt{\frac{1.81}{1.81+100}} = 0.13.
% \]
% Como se trata de una tabla de contingencia de $2\times 2$, el valor máximo que podría tomar el coeficiente de
% contingencia es $\sqrt{(2-1)/2}=\sqrt{1/2}=0.707$, y como $0.13$ está bastante lejos de este valor, se puede concluir
% que no existe demasiada relación entre el hábito de fumar y el sexo.
% 
% \note{De la definición del coeficiente de contingencia es fácil ver que, al ser el numerador siempre menor que el denominador, siempre
% estará entre 0 y 1, de manera que el coeficiente de contingencia será nulo cuando las variables sean independientes, y estará más próximo a
% 1 a medida que aumente la dependencia.
% 
% Aunque el coeficiente de contingencia nunca puede llegar a valer 1, se puede demostrar que para tablas de contingencia con $k$ filas y $k$
% columnas, el valor máximo que puede alcanzar es $\sqrt{(k-1)/k}$, de modo que se puede comparar con este valor.
% 
% En el ejemplo del sexo y el hábito de fumar, como el coeficiente chi-cuadrado valía $1.81$, el coeficiente de contingencia vale la
% raíz cuadrada de $1.81$ dividido entre $1.81$ más el tamaño de la muestra 100, lo que da $0.13$. Y como se trata de una tabla de
% contingencia de $2\times 2$, el valor máximo que podría tomar el coeficiente de contingencia es $\sqrt{(2-1)/2}=\sqrt{1/2}=0.707$, y como
% $0.13$ está bastante lejos de este valor, se puede concluir que no existe demasiada relación entre el hábito de fumar y el sexo.
% }
% \end{frame}
