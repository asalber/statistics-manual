\section{Continuous random variables}

\mode<presentation>{
%---------------------------------------------------------------------slide----
\begin{frame}
\frametitle{Continuous random variables}

\tableofcontents[sectionstyle=show/hide,hideothersubsections]
\end{frame}
}

% 
% \subsection{Variables aleatorias continuas}
% 
% %---------------------------------------------------------------------slide----
% \begin{frame}
% \frametitle{Variables aleatorias continuas}
% Las variables aleatorias continuas, a diferencia de las discretas, se caracterizan porque pueden tomar cualquier valor en un intervalo real.
% Es decir el conjunto de valores que pueden tomar no sólo es infinito, sino que además es no numerable.
% 
% Tal densidad de valores hace imposible el cálculo de las probabilidades de cada uno de ellos, y por tanto no podemos definir los modelos de
% distribución de probabilidad por medio de una función de probabilidad como en el caso discreto.
% 
% Por otro lado, la medida de una variable aleatoria continua suele estar limitada por las imprecisiones del proceso o instrumento de medida.
% Por ejemplo, cuando se dice que una estatura es $1.68$ m, no se está diciendo que es exactamente $1.68$ m, sino que la estatura está entre
% $1.675$ y $1.685$ m, ya que el instrumento de medida sólo es capaz de precisar hasta cm.
% 
% Así pues, en el caso de variables continuas, \alert{\emph{no tiene sentido medir probabilidades de valores aislados, sino que se medirán
% probabilidades de intervalos.}}
% 
% \note{
% Ya hemos visto cómo estudiar la distribución de probabilidad de variables discretas. Veamos ahora como estudiar las continuas. 
% 
% Las variables aleatorias continuas, a diferencia de las discretas, se caracterizan porque pueden tomar cualquier valor en un intervalo real.
% Es decir el conjunto de valores que pueden tomar no sólo es infinito, sino que además es no numerable.
% 
% Tal densidad de valores hace imposible el cálculo de las probabilidades de cada uno de ellos, y por tanto no podemos definir los modelos de
% distribución de probabilidad por medio de una función de probabilidad como en el caso discreto.
% 
% Por otro lado, la medida de una variable aleatoria continua suele estar limitada por las imprecisiones del proceso o instrumento de medida.
% Por ejemplo, conocer la estatura exacta de una persona es imposible, ya que no se dispone de un instrumento de medida de precisión infinta.
% Cuando se dice que una estatura es $1.68$ m, no se está diciendo que es exactamente $1.68$ m, sino que la estatura está entre $1.675$ y
% $1.685$ m, ya que el instrumento de medida sólo es capaz de precisar hasta cm.
% 
% Así pues, en el caso de variables continuas, \alert{\emph{no tiene sentido medir probabilidades de valores aislados, sino que se medirán
% probabilidades de intervalos.}}
% }
% \end{frame}
% 
% 
% %---------------------------------------------------------------------slide----
% \begin{frame}
% \frametitle{Función de densidad}
% Para conocer cómo se distribuye la probabilidad entre los valores de una variable aleatoria continua se utiliza la función de densidad.
% \begin{definition}[Función de densidad]
% La \emph{función de densidad} de una variable aleatoria continua $X$ es una función $f(x)$ que cumple las siguientes propiedades:
% \begin {itemize}
% \item Es no negativa: $f(x)\geq 0$ $\forall x\in \mathbb{R}$,
% \item El área acumulada entre la función y el eje de abscisas es 1, es decir,
% \[
% \int_{-\infty}^{\infty} f(x)\; dx = 1.
% \]
% \end{itemize}
% \end{definition}
% La probabilidad de que la variable tome un valor dentro un intervalo cualquiera $[a,b]$ es
% \[
% P(a\leq X\leq b) = \int_a^b f(x)\; dx
% \] 
% \begin{center}
% \alert{\emph{¡Ojo! $f(x)$ no es la probabilidad de que la variable tome el valor $x$.}}
% \end{center}
% 
% \note{
% Para conocer cómo se distribuye la probabilidad entre los valores de una variable aleatoria continua se utiliza una función equivalente a
% la función de probabilidad de las variables discretas, pero continua. Esta función se conoce como función de densidad y se caracteriza por
% \begin {itemize}
% \item Es no negativa: $f(x)\geq 0$ $\forall x\in \mathbb{R}$,
% \item El área acumulada entre la función y el eje de abscisas es 1, es decir,
% \[
% \int_{-\infty}^{\infty} f(x)\; dx = 1.
% \]
% \end{itemize}
% 
% La forma de calcular probabilidades a partir de la función de densidad es medir el área que queda por debajo de la función hasta el eje $X$
% y por tanto el area total que viene dada por la integral entre $-\infty$ y $\infty$ debe ser 1 que es la probabilidad total. 
% 
% De este modo, la probabilidad de que la variable tome un valor dentro un intervalo cualquiera $[a,b]$ es el área que queda por debajo de la
% función de densidad en los límites del intervalo y eso es 
% \[
% P(a\leq X\leq b) = \int_a^b f(x)\; dx
% \] 
% \begin{center}
% \alert{\emph{¡Ojo! a diferencia del caso discreto $f(x)$ no es la probabilidad de que la variable tome el valor $x$, ya que para variables
% continuas la probabilidad de un valor aislado es siempre nula.}}
% \end{center}
% }
% \end{frame}
% 
% 
% %---------------------------------------------------------------------slide----
% \begin{frame}
% \frametitle{Función de distribución}
% Al igual que para las variables discretas, también tiene sentido medir probabilidades acumuladas por debajo de un determinado valor.
% \begin{definition}[Función de distibución]
% La \emph{función de distribución} de una variable aleatoria continua $X$ es una función $F(x)$ que asocia a cada valor
% $a$ la probabilidad de que la variable tome un valor menor o igual que dicho valor.
% \[
% F(a) = P(X\leq a) = \int_{-\infty}^{a} f(x)\; dx.
% \] 
% \end{definition} 
% 
% \note{
% Al igual que para las variables discretas, también tiene sentido medir probabilidades acumuladas por debajo de un determinado valor mediante
% la función de distribución, que ahora se define, para un valor $a$ como 
% \[
% F(a) = P(X\leq a) = \int_{-\infty}^{a} f(x)\; dx.
% \] 
% }
% \end{frame}
% 
% 
% %---------------------------------------------------------------------slide----
% \begin{frame}
% \frametitle{Cálculo de probabilidades como áreas}
% La función de densidad nos permite calcular la probabilidad un intervalo $[a,b]$ como el área acumulada por debajo de la función en dicho
% intervalo.
% \begin{center}
% \scalebox{0.55}{\input{img/variables_aleatorias_continuas/calculo_probabilidad_funcion_densidad}}
% \end{center}
% \[
% P(a\leq X\leq b) = \int_a^b f(x)\; dx = F(b) -F(a)
% \]
% 
% \note{
% Como ya hemos dicho, la función de densidad nos permite calcular la probabilidad un intervalo $[a,b]$ como el área acumulada por debajo de
% la función en dicho intervalo.
% 
% Para ello tenemos dos posibilidades, a partir de la función de la función de densidad, calculando la integral definidad de esta entre $a$ y
% $b$ o, a partir de la función de distribución, midiendo el área acumulada por debajo de $b$ y restandole el área acumulada por debajo de $a$
% para quedarnos precisamente con el área que hay entre $a$ y $b$. Como es más fácil hacer restas que calcular integrales, trabajaremos la
% mayoría de las veces con funciones de distribución. 
% }
% \end{frame}
% 
% 
% %---------------------------------------------------------------------slide----
% \begin{frame}
% \frametitle{Cálculo de probabilidades como áreas}
% \framesubtitle{Ejemplo}
% Dada la siguiente función:
% \[
% f(x) = 
% \begin{cases}
% 0 & \mbox{si $x<0$}\\
% e^{-x} & \mbox{si $x\geq 0$},
% \end{cases}
% \]
% veamos que se trata de una función de densidad. Para ello hay que comprobar que es no negativa, lo cual es evidente al tratarse de una función exponencial, y que el área por debajo de ella es 1:
% \begin{align*}
% \int_{-\infty}^\infty f(x)\;dx &= \int_{-\infty}^0 f(x)\;dx +\int_0^\infty f(x)\;dx = \int_{-\infty}^0 0\;dx +\int_0^\infty e^{-x}\;dx =\\
% &= \left[-e^{-x}\right]_0^{\infty} = -e^{-\infty}+e^0 = 1.
% \end{align*}
% Ahora, a partir de ella, se puede calcular por ejemplo la probabilidad de que la variable tome un valor entre 0 y 2.
% \begin{align*}
% P(0\leq X\leq 2) &= \int_0^2 f(x)\;dx = \int_0^2 e^{-x}\;dx = \left[-e^{-x}\right]_0^2 = -e^{-2}+e^0 = 0.8646. 
% \end{align*}
% 
% \note{
% Veamos un ejemplo.
% }
% \end{frame}
% 
% 
% %---------------------------------------------------------------------slide----
% \begin{frame}
% \frametitle{Estadísticos poblacionales}
% El cálculo de los estadísticos poblacionales es similar al caso discreto, pero utilizando la función de densidad, en lugar de la función de
% probabilidad, y extendiendo la suma discreta a la integral en todo el recorrido de la variable.
% 
% Los más importantes son:
% \begin{itemize}
% \item \structure{Media o esperanza matemática}:
% \[
% \mu = E(X) = \int_{-\infty}^\infty x f(x)\; dx
% \]
% \item \structure{Varianza}:
% \[
% \sigma^2 = Var(X) = \int_{-\infty}^\infty x^2f(x)\; dx -\mu^2
% \]
% \item \structure{Desviación típica}:
% \[
% \sigma = +\sqrt{\sigma^2}
% \] 
% \end{itemize}
% 
% \note{
% El cálculo de los estadísticos poblacionales es similar al caso discreto, pero utilizando la función de densidad, en lugar de la función de
% probabilidad, y extendiendo la suma discreta a la integral en todo el recorrido de la variable.
% 
% Así la media se define como 
% \begin{itemize}
% \item \structure{Media o esperanza matemática}:
% \[
% \mu = E(X) = \int_{-\infty}^\infty x f(x)\; dx
% \]
% \item \structure{Varianza}:
% \[
% \sigma^2 = Var(X) = \int_{-\infty}^\infty x^2f(x)\; dx -\mu^2
% \]
% \item \structure{Desviación típica}:
% \[
% \sigma = +\sqrt{\sigma^2}
% \] 
% \end{itemize}
% }
% \end{frame}
% 
% 
% %---------------------------------------------------------------------slide----
% \begin{frame}
% \frametitle{Cálculo de los estadísticos poblacionales}
% \framesubtitle{Ejemplo}
% Sea la función de densidad del ejemplo anterior:
% \[
% f(x) = 
% \begin{cases}
% 0 & \mbox{si $x<0$}\\
% e^{-x} & \mbox{si $x\geq 0$}
% \end{cases}
% \]
% Su media es
% \begin{align*}
% \mu &= \int_{-\infty}^\infty xf(x)\;dx = \int_{-\infty}^0 xf(x)\;dx +\int_0^\infty xf(x)\;dx = \int_{-\infty}^0 0\;dx +\int_0^\infty xe^{-x}\;dx =\\
% &= \left[-e^{-x}(1+x)\right]_0^{\infty} = 1.
% \end{align*}
% y su varianza vale
% \begin{align*}
% \sigma^2 &= \int_{-\infty}^\infty x^2f(x)\;dx -\mu^2 = \int_{-\infty}^0 x^2f(x)\;dx +\int_0^\infty x^2f(x)\;dx -\mu^2 = \\
% &= \int_{-\infty}^0 0\;dx +\int_0^\infty x^2e^{-x}\;dx -\mu^2= \left[-e^{-x}(x^2+2x+2)\right]_0^{\infty} - 1^2= 2e^0-1 = 1.
% \end{align*}
% 
% \note{
% Siguiendo con el ejemplo anterior, la media vale:
% }
% \end{frame}
% 
% 
% 
% %---------------------------------------------------------------------slide----
% \begin{frame}
% \frametitle{Modelos de distribución continuos}
% Existen varios modelos de distribución de probabilidad que aparecen bastante a menudo en la naturaleza y también como consecuencia de los
% procesos de muestreo aleatorio simple.
% 
% A continuación veremos los más importantes:
% \begin{itemize}
% \item Distribución Uniforme continua.
% \item Distribución Normal.
% \item Distribución T de Student.
% \item Distribución Chi-cuadrado.
% \item Distribución F de Fisher-Snedecor.
% \end{itemize}
% 
% \note{
% Existen varios modelos de distribución de probabilidad que aparecen bastante a menudo en la naturaleza y también como consecuencia de los
% procesos de muestreo aleatorio simple.
% 
% A continuación veremos los más importantes:
% \begin{itemize}
% \item Distribución Uniforme continua.
% \item Distribución Normal.
% \item Distribución T de Student.
% \item Distribución Chi-cuadrado.
% \item Distribución F de Fisher-Snedecor.
% \end{itemize}
% }
% \end{frame}
% 
% 
% \subsection{Distribución Uniforme continua}
% 
% %---------------------------------------------------------------------slide----
% \begin{frame}
% \frametitle{Distribución Uniforme continua $U(a,b)$}
% Cuando todos los valores de una variable continua son equiprobables, se dice que la variable sigue un \emph{modelo de distribución uniforme
% continuo}.
% \begin{definition}[Distribución Uniforme continua]
% Una variable aleatoria continua $X$, cuyo recorrido es el intervalo $[a,b]$, sigue un modelo de distribución \emph{uniforme} $U(a,b)$, si todos los valores de la variable son equiprobables, y por tanto, su función de densidad es constante en todo el intervalo:
% \[
% f(x)= \frac{1}{b-a}\quad \forall x\in [a,b]
% \]
% \end{definition}
% 
% Su media y varianza valen
% \[
% \mu = \frac{a+b}{2}\qquad \sigma^2=\frac{(b-a)^2}{12}.
% \]
% 
% \note{
% Al igual que para variables discretas, cuando todos los valores de una variable continua son equiprobables, se dice que la variable sigue
% un \emph{modelo de distribución uniforme continuo}.
% 
% Si el recorrido de la variable es el intervalo $[a,b]$ entonces se dice que sigue un modelo de distribución \emph{uniforme} $U(a,b)$, y se
% cumple que su función de densidad es constante y vale:
% \[
% f(x)= \frac{1}{b-a}\quad \forall x\in [a,b]
% \]
% 
% Además, su media vale
% \[
% \mu = \frac{a+b}{2}.
% \]
% y su varianza
% \[
% \sigma^2=\frac{(b-a)^2}{12}.
% \]
% }
% \end{frame}
% 
% 
% %---------------------------------------------------------------------slide----
% \begin{frame}
% \frametitle{Función de densidad de la Uniforme continua $U(a,b)$}
% La generación aleatoria de un número real entre 0 y 1 sigue un modelo de distribución uniforme continuo $U(0,1)$. 
% \begin{center}
% \scalebox{0.7}{\input{img/variables_aleatorias_continuas/funcion_densidad_uniforme}}
% \end{center}
% 
% \note{
% Un ejemplo de variable aleatoria uniforme continua sería la que mide el resultado de generar aleatoriamente un número entre 0 y 1. Esta
% variable seguiría un modelo de distribución uniforme continuo $U(0,1)$, y la gráfica de su función de densidad es esta. Como se ve, la
% función es constante y vale 1, ya que debe cumplirse, al ser función de densidad, que el área total por debajo de ella debe ser 1, y como en
% realidad se trata del área de un rectángulo de base 1, pues la altura debe ser también 1.
% }
% \end{frame}
% 
% 
% %---------------------------------------------------------------------slide----
% \begin{frame}
% \frametitle{Función de distribución de la Uniforme continua $U(a,b)$}
% Como la función de densidad es constante, la función de distribución presenta un crecimiento lineal.
% \begin{center}
% \scalebox{0.7}{\input{img/variables_aleatorias_continuas/funcion_distribucion_uniforme}}
% \end{center}
% 
% \note{
% En esta otra gráfica tenemos la función de distribución $U(0,1)$.
% Como la función de densidad es constante, la función de distribución presenta un crecimiento lineal. Obsérvese que, como para cualquier
% función de distribución, la probabilidad acumulada al comienzo del recorrido es 0, y al final, vale 1, que es la probabilidad total. 
% }
% \end{frame}
% 
% 
% % ---------------------------------------------------------------------slide----
% \begin{frame}
% \frametitle{Cálculo de probabilidades con una Uniforme continua}
% \framesubtitle{Ejemplo de espera de un autobús}
% Supóngase que un autobús pasa por una parada cada 15 minutos. Si una persona puede llegar a la parada en cualquier instante, \emph{¿cuál es
% la probabilidad de que espere entre 5 y 10 minutos?}
% \begin{columns}
% \begin{column}{0.52\textwidth}
% En este caso, la variable $X$ que mide el tiempo de espera sigue un modelo de distribución uniforme continua $U(0,15)$ ya que cualquier
% valor entre los 0 y los 15 minutos es equipobrable.
% 
% Así pues, la probabilidad que nos piden es
% \begin{align*}
% P(5\leq X\leq 10) &= \int_{5}^{10} \frac{1}{15}\;dx = \left[\frac{x}{15}\right]_5^{10} = \\
% &= \frac{10}{15}-\frac{5}{15} =\frac{1}{3}.
% \end{align*}
% \end{column}
% \begin{column}{0.43\textwidth}
% \begin{center}
% \scalebox{0.5}{\input{img/variables_aleatorias_continuas/calculo_probabilidades_uniforme}}
% \end{center}
% \end{column}
% \end{columns}
% Además, el tiempo medio de espera será $\mu=\frac{0+15}{2}=7.5$ minutos.
% 
% \note{
% Como ejemplo, supóngase que un autobús pasa por una parada cada 15 minutos. Si una persona puede llegar a la parada en cualquier instante,
% \emph{¿cuál es la probabilidad de que espere entre 5 y 10 minutos?}
% 
% Está claro que lo mínimo que puede esperar una persona es 0 minutos, si llega justo cuando está el autobús, y lo máximo es 15 minutos, si
% llega justo cuando el autobús se marcha. Como además puede llegar en cualquier instante con equiprobabiliadad, la variable $X$ que mide el
% tiempo de espera sigue un modelo de distribución uniforme continua $U(0,15)$.
% 
% Así pues, la probabilidad que nos piden es
% \begin{align*}
% P(5\leq X\leq 10) &= \int_{5}^{10} \frac{1}{15}\;dx = \left[\frac{x}{15}\right]_5^{10} = \\
% &= \frac{10}{15}-\frac{5}{15} =\frac{1}{3}.
% \end{align*}
% 
% Aunque también podría calcularse fácilmente a partir de la gráfica de la función de densidad que es constante y vale $1/15$, midiendo el
% área que queda por debajo de esta función entre 5 y 10. Como se tráta del área de un rectánculo, basta multiplicar la base, que es 5, por la
% altura que es $1/15$ y de nuevo se obtiene $1/3$.
% }
% \end{frame}
% 
% 
% \subsection{Distribución Normal}
% 
% %---------------------------------------------------------------------slide----
% \begin{frame}
% \frametitle{Distribución Normal $N(\mu,\sigma)$}
% El modelo de distribución normal es, sin duda, el modelo de distribución continuo más importante, ya que es el que más a menudo se presenta
% en la naturaleza.
% \begin{definition}[Distribución Normal]
% Una variable aleatoria continua $X$ sigue un modelo de distribución \emph{normal} $N(\mu,\sigma)$ si su recorrido es $\mathbb{R}$ y su función de densidad vale
% \[
% f(x)= \frac{1}{\sigma\sqrt{2\pi}}e^{-\frac{(x-\mu)^2}{2\sigma^2}}.
% \]
% \end{definition}
% 
% La distribución normal depende de dos parámetros $\mu$ y $\sigma$ que son, precisamente, su media y desviación típica.
% 
% \note{
% El modelo de distribución normal es, sin duda, el modelo de distribución continuo más importante, ya que es el que más a menudo se presenta
% en la naturaleza, sobre todo en variables continuas biológicas.
% 
% \begin{definition}[Distribución Normal]
% Una variable aleatoria continua $X$ sigue un modelo de distribución \emph{normal} $N(\mu,\sigma)$ si su recorrido es $\mathbb{R}$ y su función de densidad vale
% \[
% f(x)= \frac{1}{\sigma\sqrt{2\pi}}e^{-\frac{(x-\mu)^2}{2\sigma^2}}.
% \]
% \end{definition}
% 
% La distribución normal depende de dos parámetros $\mu$ y $\sigma$ que son, precisamente, su media y desviación típica.
% }
% \end{frame}
% 
% 
% %---------------------------------------------------------------------slide----
% \begin{frame}
% \frametitle{Función de densidad de la Normal $N(\mu,\sigma)$}
% La gráfica de la función de densidad de la distribución normal tiene forma de una especie de campana, conocida como \emph{campana de Gauss}
% (en honor a su descubridor), y está centrada en la media $\mu$.
% \begin{center}
% \scalebox{0.65}{\input{img/variables_aleatorias_continuas/funcion_densidad_normal}}
% \end{center}
% 
% \note{
% La gráfica de la función de densidad de la distribución normal tiene forma de una especie de campana, conocida como \emph{campana de Gauss}
% (en honor a su descubridor), y está centrada en la media $\mu$.
% }
% \end{frame}
% 
% 
% %---------------------------------------------------------------------slide----
% \begin{frame}
% \frametitle{Función de densidad de la Normal $N(\mu,\sigma)$}
% La forma de la campana de Gauss depende de sus dos parámetros:
% \begin{itemize}
% \item La media $\mu$ determina dónde está centrada.
% \item La desviación típica $\sigma$ determina su anchura.
% \end{itemize} 
% \begin{center}
% \scalebox{0.5}{\input{img/variables_aleatorias_continuas/funcion_densidad_normales_distinta_media}}
% \quad
% \scalebox{0.5}{\input{img/variables_aleatorias_continuas/funcion_densidad_normales_distinta_varianza}}
% \end{center}
% 
% \note{
% La forma de la campana de Gauss depende de sus dos parámetros:
% \begin{itemize}
% \item Por un lado, ya hemos visto que la media $\mu$ determina dónde está centrada. En la gráfica de la izquierda podemos ver las funciones
% de densidad de dos normales, una con media 0 y desviación típica 1, y otra con media 2 y desviación típica 1. Como se ve, ambas gráficas son
% idénticas, salvo que la primera está centrada en el 0 y la segunda en el 2.
% \item Por otro lado, la desviación típica $\sigma$, al ser una medida de la dispersión de la población, determina su anchura. En la gráfica
% de la derecha podemos ver las funciones de densidad de dos normales con la misma media 0, y desviaciones típicas 1, y 2 respectivamente.
% Como se ve, ambas están centradas en el 0, pero la primera es más estrecha que la segunda al tener menos dispersión. 
% \end{itemize} 
% }
% \end{frame}
% 
% 
% %---------------------------------------------------------------------slide----
% \begin{frame}
% \frametitle{Función de distribución de la Normal $N(\mu,\sigma)$}
% Por su parte, la gráfica de la función de distribución tiene forma de S. 
% \begin{center}
% \scalebox{0.7}{\input{img/variables_aleatorias_continuas/funcion_distribucion_normal}}
% \end{center}
% 
% \note{
% Por su parte, la gráfica de la función de distribución de una normal siempre tiene forma de S. 
% }
% \end{frame}
% 
% 
% %---------------------------------------------------------------------slide----
% \begin{frame}
% \frametitle{Propiedades de la distribución Normal}
% \begin{itemize}
% \item La función de densidad es simétrica respecto a la media y por tanto, su coeficiente de asimetría es $g_1=0$.
% \item También es mesocúrtica, y por tanto, su coeficiente de apuntamiento vale $g_2=0$.
% \item La media, la mediana y la moda coinciden
% \[
% \mu = Me = Mo.
% \]
% \item Tiende asintóticamente a 0 cuando $x$ tiende a $\pm \infty$.
% \end{itemize}
% 
% \note{
% La distribución normal tiene propiedades muy interesantes:
% \begin{itemize}
% \item La función de densidad es simétrica respecto a la media y por tanto, su coeficiente de asimetría es $g_1=0$.
% \item También es mesocúrtica, y por tanto, su coeficiente de apuntamiento vale $g_2=0$. Recordemos que el apuntamiento de cualquier variable
% se compara precisamente con el apuntamiento de la distribución normal, ya que al ser esta la distribución más común, se toma como
% referencia.
% \item De nuevo, al ser simétrica con respecto a la media, hasta la media tendremos acumulada la mitad de la probabilidad, y por tanto la
% media coincide con la mediana, y también con la moda, ya que sobre la media se alcanza el máximo de la función de densidad.
% \[
% \mu = Me = Mo.
% \]
% \item Tiende asintóticamente a 0 cuando $x$ tiende a $\pm \infty$.
% \end{itemize}
% }
% \end{frame}
% 
% 
% %---------------------------------------------------------------------slide----
% \begin{frame}
% \frametitle{Propiedades de la distribución Normal}
% \begin{itemize}
% \item Se cumple que
% \begin{align*}
% & P(\mu-\sigma \leq X \leq \mu+\sigma) = 0.68,\\
% & P(\mu-2\sigma \leq X \leq \mu+2\sigma) = 0.95,\\
% & P(\mu-3\sigma \leq X \leq \mu+3\sigma) = 0.99.
% \end{align*}
% \end{itemize}
% \begin{center}
% \scalebox{0.6}{\input{img/variables_aleatorias_continuas/propiedades_normal}}
% \end{center}
% 
% \note{
% Y también se cumple que
% \begin{align*}
% & P(\mu-\sigma \leq X \leq \mu+\sigma) = 0.68,\\
% & P(\mu-2\sigma \leq X \leq \mu+2\sigma) = 0.95,\\
% & P(\mu-3\sigma \leq X \leq \mu+3\sigma) = 0.99.
% \end{align*}
% }
% es decir, casi la totalidad de los individuos de la población presentarán valores entre la media menos tres veces la desviación típica, y la
% media mas tres veces la desviación típica.
% \end{frame}
% 
% 
% %---------------------------------------------------------------------slide----
% \begin{frame}
% \frametitle{Propiedades de la distribución Normal}
% \framesubtitle{Ejemplo}
% En un estudio se ha comprobado que el nivel de colesterol total en mujeres
% sanas de entre 40 y 50 años sigue una distribución normal de media de 210 mg/dl y desviación
% típica 20 mg/dl. 
% \emph{¿Qué quiere decir esto?}
% 
% Atendiendo a las propiedades de la campana de Gauss, se tiene que 
% \begin{itemize}
% \item El 68\% de las mujeres sanas tendrán el colesterol entre $210\pm 20$ mg/dl, es decir, entre 190 y 230 mg/dl.
% \item El 95\% de las mujeres sanas tendrán el colesterol entre $210\pm 2\cdot 20$ mg/dl, es decir, entre 170 y 250
% mg/dl.
% \item El 99\% de las mujeres sanas tendrán el colesterol entre $210\pm 3\cdot 20$ mg/dl, es decir, entre 150 y 270
% mg/dl.
% \end{itemize}
% 
% En la analítica sanguínea suele utilizarse el intervalo $\mu\pm 2\sigma$ para detectar posibles patologías. En el caso
% del coresterol, dicho intervalo es $[170\text{ mg/dl}, 250\text{ mg/dl}]$. Cuando una persona tiene el colesterol fuera
% de estos límites, se tiende a pensar que tiene alguna patología, aunque ciertamente podría estar sana, pero la
% probabilidad de que eso ocurra es sólo de un 5\%.
% 
% \note{
% Veamos una aplicación de estas propiedades. 
% 
% En un estudio se ha comprobado que el nivel de colesterol total en mujeres sanas de entre 40 y 50 años sigue una distribución normal de
% media de 210 mg/dl y desviación típica 20 mg/dl. 
% \emph{¿Qué quiere decir esto?}
% 
% Atendiendo a las propiedades de la campana de Gauss, se tiene que 
% \begin{itemize}
% \item El 68\% de las mujeres sanas tendrán el colesterol entre $210\pm 20$ mg/dl, es decir, entre 190 y 230 mg/dl.
% \item El 95\% de las mujeres sanas tendrán el colesterol entre $210\pm 2\cdot 20$ mg/dl, es decir, entre 170 y 250
% mg/dl.
% \item El 99\% de las mujeres sanas tendrán el colesterol entre $210\pm 3\cdot 20$ mg/dl, es decir, entre 150 y 270
% mg/dl.
% \end{itemize}
% 
% En la analítica sanguínea suele utilizarse el intervalo $\mu\pm 2\sigma$ para detectar posibles patologías. En el caso
% del coresterol, dicho intervalo es $[170\text{ mg/dl}, 250\text{ mg/dl}]$. Cuando una persona tiene el colesterol fuera
% de estos límites, se tiende a pensar que tiene alguna patología, aunque ciertamente podría estar sana, pero la
% probabilidad de que eso ocurra es sólo de un 5\%.
% }
% \end{frame}
% 
% 
% %---------------------------------------------------------------------slide----
% \begin{frame}
% \frametitle{El teorema central del límite}
% El comportamiento anterior lo presentan muchas variables continuas físicas y biológicas.
% 
% Si se piensa por ejemplo en la distribución de las estaturas, se verá que la mayor parte de los individuos presentan estaturas en torno a la
% media, tanto por arriba, como por debajo, pero que a medida que van alejándose de la media, cada vez hay menos individuos con dichas
% estaturas.
% 
% La justificación de que la distribución normal aparezca de manera tan frecuente en la naturaleza la encontramos en el
% \structure{\textbf{teorema central del límite}}, que veremos más adelante, y que establece que si una variable aleatoria $X$ proviene de un
% experimento aleatorio cuyos resultados son debidos a un conjunto muy grande de causas independientes que actúan sumando sus efectos,
% entonces $X$ sigue una distribución aproximadamente normal.
% 
% \note{
% Se ha observado que la mayor parte de las variables continuas físicas y biológicas presentan una distribucón con forma de campana de Gauss.
% 
% Si se piensa por ejemplo en la distribución de las estaturas, se verá que la mayor parte de los individuos presentan estaturas en torno a la
% media, tanto por arriba, como por debajo, pero que a medida que van alejándose de la media, cada vez hay menos individuos con dichas
% estaturas.
% 
% La justificación de que la distribución normal aparezca de manera tan frecuente en la naturaleza la encontramos en el
% \structure{\textbf{teorema central del límite}}, que veremos más detenidamente en el siguiente tema, y que establece que si una variable
% aleatoria $X$ proviene de un experimento aleatorio cuyos resultados son debidos a un conjunto muy grande de causas independientes que actúan
% sumando sus efectos, entonces $X$ sigue una distribución aproximadamente normal.
% 
% Si pensamos en cualquier variable biológica, como por ejemplo la tensión arterial, rápidamente podremos identificar multitud de factores que
% influyen en ella, como por ejemplo la edad, el sexo, la dieta, el ejercicio físico, si se fuma o no, etc. El valor de la tensión arterial en
% un individuo es el resultado de todos estos factores que suman sus efectos de manera independiente, y por ello, el el colesterol acaba
% teniendo una distribución normal.
% }
% \end{frame}
% 
% 
% %---------------------------------------------------------------------slide----
% \begin{frame}
% \frametitle{La distribución Normal estándar $N(0,1)$}
% De todas las distribuciones normales, la más importante es la que tiene media $\mu=0$ y desviación típica $\sigma=1$,
% que se conoce como \structure{\textbf{normal estándar}} y se designa por $Z$.
% \begin{center}
% \scalebox{0.65}{\input{img/variables_aleatorias_continuas/funcion_densidad_normal_estandar}}
% \end{center}
% 
% \note{
% De todas las distribuciones normales, la más importante es la que tiene media $\mu=0$ y desviación típica $\sigma=1$,
% que se conoce como \structure{\textbf{normal estándar}} y se designa por $Z$.
% 
% Su función de densidad será una campana de Gauss centrada en el 0, en la que el 99\% de los individuos estarán entre -3 y 3. 
% }
% \end{frame}
% 
% 
% %---------------------------------------------------------------------slide----
% \begin{frame}
% \frametitle{Cálculo de probabilidades con la Normal estándar}
% \framesubtitle{Manejo de la tabla de la función de distribución}
% Para evitar tener que calcular probabilidades integrando la función de densidad de la normal estándar se suele utilizar su función de
% distribución.
% 
% Habitualmente se suele manejar una tabla con los valores de la función de distribución tabulados cada centésima. 
% \begin{columns}
% \begin{column}{0.47\textwidth}
% \begin{center}
% Ejemplo $P(Z\leq 0.52)$
% 
% \scalebox{0.8}{\input{img/variables_aleatorias_continuas/tabla_normal}}
% 
% $0.52 \rightarrow $ fila $0.5$ + columna $0.02$
% \end{center}
% \end{column}
% \begin{column}{0.53\textwidth}
% \scalebox{0.6}{\input{img/variables_aleatorias_continuas/calculo_probabilidades_normal_estandar}}
% \end{column}
% \end{columns}
% 
% \note{
% Para evitar tener que calcular probabilidades integrando la función de densidad de la normal estándar se suele utilizar su función de
% distribución.
% 
% Habitualmente se suele manejar una tabla con los valores de la función de distribución tabulados cada centésima. En esta tabla, la primera
% columna contiene las décimas y la primera fila las centésimas, de forma que si por ejemplo queremos medir la probabilidad acumulada hasta el
% $0.52$, tenemos que ir hasta la casilla correspondiente a la fila $0.5$  (5 décimas) y hasta la columna $0.02$ (2 centésimas), donde aparece
% la probabilidad acumulada $0.6985$ y esta sería el área que queda por debajo de la campana de Gauss de la normal estándar entre $-\infty$ y
% $0.52$.
% }
% \end{frame}
% 
% 
% %---------------------------------------------------------------------slide----
% \begin{frame}
% \frametitle{Cálculo de probabilidades con la Normal estándar}
% \framesubtitle{Probabilidades acumuladas por encima de un valor}
% Cuando tengamos que calcular probabilidades acumuladas por encima de un determinado valor, podemos hacerlo por medio de la probabilidad del
% suceso contrario.
% 
% Por ejemplo
% \[
% P(Z>0.52) =1-P(Z\leq 0.52) = 1-F(0.52) = 1 - 0.6985 = 0.3015.
% \]
% \begin{center}
% \scalebox{0.55}{\input{img/variables_aleatorias_continuas/calculo_probabilidades_derecha_normal_estandar}}
% \end{center}
% 
% \note{
% Cuando tengamos que calcular probabilidades acumuladas por encima de un determinado valor, podemos hacerlo por medio de la probabilidad del
% suceso contrario, ya que la función de distribución siempre mide probabilidades acumuladas por debajo de cada valor. 
% 
% Así, la probabilidad acumulada por encima de $0.52$, será 1 menos la probabilidad acumulada por debajo de dicho valor, que como hemos visto
% en la tabla, valía $0.6985$, lo que nos da $0.3015$. Esto es lógico, ya que si el área total por debajo de la campana de Gauss es 1, el área
% acumulada por encima de $0.52$ será 1 menos el área que quede por debajo de este valor. 
% }
% \end{frame}
% 
% 
% %---------------------------------------------------------------------slide----
% \begin{frame}
% \frametitle{Tipificación}
% Ya se ha visto cómo calcular probabilidades con una distribución normal estándar, pero \emph{¿qué hacer cuando la
% distribución normal no es la estándar?}
% 
% Afortunadamente, siempre se puede transformar una variable normal para convertirla en una normal estándar. 
% \begin{teorema}[Tipificación]
% Si $X$ es una variable normal de media $\mu$ y desviación típica $\sigma$, entonces la variable resultante de restarle a $X$ su media y
% dividir por su desviación típica, sigue un modelo de distribución normal estándar:
% \[
% X\sim N(\mu,\sigma) \Rightarrow Z=\frac{X-\mu}{\sigma}\sim N(0,1).
% \]
% Esta transformación lineal se conoce como \emph{transformación de tipificación} y la variable resultante $Z$ se conoce como \emph{normal
% tipificada}.
% \end{teorema} 
% 
% Así pues, para calcular probabilidades de una variable normal que no sea la normal estándar, se aplica primero la transformación de
% tipificación y después se puede utilizar la función de distribución de la normal estándar.
% 
% \note{
% Ya se ha visto cómo calcular probabilidades con una distribución normal estándar, pero \emph{¿qué hacer cuando la
% distribución normal no es la estándar?}
% 
% Afortunadamente, siempre se puede transformar una variable normal para convertirla en una normal estándar tipificándola.
% \begin{teorema}[Tipificación]
% Si $X$ es una variable normal de media $\mu$ y desviación típica $\sigma$, entonces la variable resultante de restarle a $X$ su media y
% dividir por su desviación típica, sigue un modelo de distribución normal estándar:
% \[
% X\sim N(\mu,\sigma) \Rightarrow Z=\frac{X-\mu}{\sigma}\sim N(0,1).
% \]
% Esta transformación lineal se conoce como \emph{transformación de tipificación} y la variable resultante $Z$ se conoce como \emph{normal
% tipificada}.
% \end{teorema} 
% 
% Así pues, para calcular probabilidades de una variable normal que no sea la normal estándar, tendremos que aplicar la transformación de
% tipificación primero y después se puede utilizar la tabla de la función de distribución de la normal estándar.
% }
% \end{frame}
% 
% 
% %---------------------------------------------------------------------slide----
% \begin{frame}
% \frametitle{Cálculo de probabilidades tipificando}
% \framesubtitle{Ejemplo}
% Supóngase que la nota de un examen sigue un modelo de distribución de probabilidad normal $N(\mu=6,\sigma=1.5)$. \emph{¿Qué porcentaje de
% suspensos habrá en la población?}
% 
% Para responder a esta pregunta necesitamos calcular la probabilidad $P(X<5)$. Como $X$ no es la normal estándar, se le aplica la
% transformación de tipificación $Z=\displaymath \frac{X-\mu}{\sigma} = \frac{X-6}{1.5}$:
% \[
% P(X<5) = P\left(\frac{X-6}{1.5}<\frac{5-6}{1.5}\right) = P(Z<-0.67).
% \]
% Después se mira en la tabla de la función de distribución de la normal estándar:
% \[
% P(Z<-0.67) = F(-0.67) = 0.2514.
% \]
% 
% Así pues, habrán suspendido el $25.14\%$ de los alumnos. 
% 
% \note{
% Veamos un ejemplo. 
% 
% Supóngase que la nota de un examen sigue un modelo de distribución de probabilidad normal $N(\mu=6,\sigma=1.5)$. \emph{¿Qué porcentaje de
% suspensos habrá en la población?}
% 
% Para responder a esta pregunta necesitamos calcular la probabilidad de que la nota sea inferior a 5. Como $X$ no es la normal estándar, se
% le aplica la transformación de tipificación, que es $Z=\displaymath \frac{X-\mu}{\sigma} = \frac{X-6}{1.5}$. Por tanto, 
% \[
% P(X<5) = P\left(\frac{X-6}{1.5}<\frac{5-6}{1.5}\right) = P(Z<-0.67).
% \]
% Después sólo queda mirar en tabla de la función de distribución de la normal estándar y observar que la probabilidad acumulada hasta el
% $-0.67$ es $0.2514$.
% 
% Así pues, habrán suspendido el $25.14\%$ de los alumnos. 
% }
% \end{frame}
% 
% 
% \subsection{Distribución Chi-cuadrado}
% 
% %---------------------------------------------------------------------slide----
% \begin{frame}
% \frametitle{Distribución chi-cuadrado $\chi^2(n)$}
% \begin{definition}[Distribución chi-cuadrado $\chi^2(n)$]
% Si  $Z_1,\ldots,Z_n$ son $n$ variables aleatorias normales estándar independientes, entonces la suma de sus cuadrados sigue un modelo de
% distribución \emph{chi-cuadrado de $n$ grados de libertad}:
% \[
% \chi^2(n) = Z_1^2+\cdots +Z_n^2.
% \]
% \end{definition}
% Su recorrido es  $\mathbb{R}^+$ y su media y varianza valen
% \[
% \mu = n, \qquad \sigma^2 = 2n.
% \]
% Como se verá más adelante, la distribución chi-cuadrado juega un papel importante en la estimación de la varianza poblacional y en el
% estudio de la relación entre variables cualitativas.
% 
% \note{
% Las distribuciones que veremos a continuación son distribuciones que siguen algunos de los estadísticos observados en las muestras
% aleatorias. Su principal utilidad se verá en el siguiente tema.
% 
% El primer modelo de distribución es el modelo chi-cuadrado.
% 
% Si  $Z_1,\ldots,Z_n$ son $n$ variables aleatorias normales estándar independientes, entonces
% la suma de sus cuadrados sigue un modelo de distribución \emph{chi-cuadrado de $n$ grados de libertad}:
% \[
% \chi^2(n) = Z_1^2+\cdots +Z_n^2.
% \]
% 
% Se cumple que su recorrido es  $\mathbb{R}^+$, su media vale $\mu=n$ y su varianza vale $\sigma^2 = 2n$.
% 
% Como se verá más adelante, la distribución chi-cuadrado juega un papel importante en la estimación de la varianza poblacional y en el
% estudio de la relación entre variables cualitativas.
% }
% \end{frame}
% 
% 
% %---------------------------------------------------------------------slide----
% \begin{frame}
% \frametitle{Función de densidad de la distribución chi-cuadrado}
% 
% \begin{center}
% \scalebox{0.8}{\input{img/variables_aleatorias_continuas/funcion_densidad_chi_cuadrado}}
% \end{center}
% 
% \note{
% Aquí tenemos las gráfica de la función de densidad de varias distribuciones chi-cuadrado con distintos grados de libertad. Como se puede
% apreciar, esta distribución, por ser la suma de cuadrados, no presenta valores negativos, y a diferencia de la normal, siempre es asimétrica
% hacia la izquierda.
% }
% \end{frame}
% 
% 
% %---------------------------------------------------------------------slide----
% \begin{frame}
% \frametitle{Propiedades de la distribución chi-cuadrado $\chi^2(n)$}
% \begin{itemize}
% \item No toma valores negativos.
% \item Si $X\sim \chi^2(n)$ e $Y\sim \chi^2(m)$, entonces
% \[
% X+Y \sim \chi^2(n+m).
% \]
% \item Al aumentar el número de grados de libertad, se aproxima asintóticamente a una normal.
% \end{itemize}
% 
% \note{
% Algunas propiedades que tiene la distribución chi-cuadrado son:
% \begin{itemize}
% \item No toma valores negativos.
% \item Si $X\sim \chi^2(n)$ e $Y\sim \chi^2(m)$, entonces
% \[
% X+Y \sim \chi^2(n+m).
% \]
% \item Al aumentar el número de grados de libertad, se aproxima asintóticamente a una normal.
% \end{itemize}
% }
% \end{frame}
% 
% 
% \subsection{Distribución T de Student}
% 
% %---------------------------------------------------------------------slide----
% \begin{frame}
% \frametitle{Distribución T de Student $T(n)$}
% \begin{definition}[Distribución T de Student $T(n)$]
% Si  $Z\sim N(0,1)$ es una variable aleatoria normal estándar y $X\sim \chi^2(n)$ es una variable aleatoria chi-cuadrado de $n$ grados de
% libertad, ambas independientes, entonces la variable
% \[
% T = \frac{Z}{\sqrt{X/n}},
% \]
% sigue un modelo de distribución \emph{T de Student de $n$ grados de libertad}.
% \end{definition}
% 
% Su recorrido es $\mathbb{R}$ y su media y varianza valen
% \[
% \mu = 0, \qquad \sigma^2 = \frac{n}{n-2} \mbox{ si $n>2$}.
% \]
% 
% Como se verá más adelante, la distribución T de Student juega un papel importante en la estimación la media poblacional.
% 
% \note{
% \begin{definition}[Distribución T de Student $T(n)$]
% Si  $Z\sim N(0,1)$ es una variable aleatoria normal estándar y $X\sim \chi^2(n)$ es una variable aleatoria chi-cuadrado de $n$ grados de
% libertad, ambas independientes, entonces la variable
% \[
% T = \frac{Z}{\sqrt{X/n}},
% \]
% sigue un modelo de distribución \emph{T de Student de $n$ grados de libertad}.
% \end{definition}
% 
% Se cumple que su recorrido es $\mathbb{R}$ y su media vale $\mu=0$ y su varianza vale
% \[
% \sigma^2 = \frac{n}{n-2} \mbox{ si $n>2$}.
% \]
% 
% Como se verá más adelante, la distribución T de Student juega un papel importante en la estimación la media poblacional.
% }
% \end{frame}
% 
% 
% %---------------------------------------------------------------------slide----
% \begin{frame}
% \frametitle{Función de densidad de la distribución T de Student}
% 
% \begin{center}
% \scalebox{0.8}{\input{img/variables_aleatorias_continuas/funcion_densidad_t_student}}
% \end{center}
% 
% \note{
% Aquí tenemos las gráfica de la función de densidad de varias distribuciones T de student con distintos grados de libertad. Como se puede
% apreciar, esta distribución está centrada en el 0 y es muy parecida a la normal estándar, aunque un poco más platicúrtica. 
% }
% \end{frame}
% 
% 
% %---------------------------------------------------------------------slide----
% \begin{frame}
% \frametitle{Propiedades de la distribución T de Student $T(n)$}
% \begin{itemize}
% \item Es simétrica con respecto a su media $\mu=0$.
% \item Es muy similar a la normal estándar, pero algo más platicúrtica. Además, a medida que aumentan los grados de libertad, la gráfica de la distribución tiende hacia la de la normal estándar, hasta llegar a ser prácticamente iguales para $n\geq 30$.
% \[
% T(n)\stackrel{n\rightarrow \infty}{\approx}N(0,1).
% \]
% \end{itemize}
% 
% \note{
% Algunas propiedades que tiene la distribución T de student son:
% \begin{itemize}
% \item Es simétrica con respecto a su media $\mu=0$.
% \item Como ya hemos dicho, es muy similar a la normal estándar, pero algo más platicúrtica. Además, a medida que aumentan los grados de
% libertad, la gráfica de la distribución tiende hacia la de la normal estándar, hasta llegar a ser prácticamente iguales para $n\geq 30$.
% \[
% T(n)\stackrel{n\rightarrow \infty}{\approx}N(0,1).
% \]
% \end{itemize}
% }
% \end{frame}
% 
% 
% \subsection{Distribución F de Fisher-Snedecor}
% 
% %---------------------------------------------------------------------slide----
% \begin{frame}
% \frametitle{Distribución F de Fisher-Snedecor $F(m,n)$}
% \begin{definition}[Distribución F de Fisher-Snedecor $F(m,n)$]
% Si  $X\sim \chi^2(m)$ es una variable aleatoria chi-cuadrado de $m$ grados de libertad e $Y\sim \chi^2(n)$ es otra variable aleatoria
% chi-cuadrado de $n$ grados de libertad, ambas independientes, entonces la variable
% \[
% F = \frac{X/m}{Y/n},
% \]
% sigue un modelo de distribución \emph{F de Fisher-Snedecor de $m$ y $n$ grados de libertad}.
% \end{definition}
% 
% Su recorrido es $\mathbb{R}^+$ y su media y varianza valen
% \[
% \mu = \frac{n}{n-2}, \qquad \sigma^2 =\frac{2n^2(m+n−2)}{m(n-2)^2(n-4)}\mbox{ si $n>4$}.
% \]
% 
% Como se verá más adelante, la distribución F de Fisher-Snedecor juega un papel importante en la comparación de varianzas poblacionales y en
% el análisis de la varianza.
% 
% \note{
% \begin{definition}[Distribución F de Fisher-Snedecor $F(m,n)$]
% Si  $X\sim \chi^2(m)$ es una variable aleatoria chi-cuadrado de $m$ grados de libertad e $Y\sim \chi^2(n)$ es otra variable aleatoria
% chi-cuadrado de $n$ grados de libertad, ambas independientes, entonces la variable
% \[
% F = \frac{X/m}{Y/n},
% \]
% sigue un modelo de distribución \emph{F de Fisher-Snedecor de $m$ y $n$ grados de libertad}.
% \end{definition}
% 
% Se cumple que su recorrido es $\mathbb{R}^+$ y su media vale
% \[
% \mu = \frac{n}{n-2},
% \]
% y su varianza 
% \[
% \sigma^2 =\frac{2n^2(m+n−2)}{m(n-2)^2(n-4)}\mbox{ si $n>4$}.
% \]
% 
% Como se verá más adelante, la distribución F de Fisher-Snedecor juega un papel importante en la comparación de varianzas poblacionales y en
% el análisis de la varianza.
% }
% \end{frame}
% 
% 
% %---------------------------------------------------------------------slide----
% \begin{frame}
% \frametitle{Función de densidad de la distribución F de Fisher-Snedecor $F(m,n)$}
% 
% \begin{center}
% \scalebox{0.75}{\input{img/variables_aleatorias_continuas/funcion_densidad_f_fisher}}
% \end{center}
% 
% \note{
% Aquí tenemos las gráfica de la función de densidad de varias distribuciones F de Fisher con distintos grados de libertad. Como se puede
% apreciar, al igual que la distribución chi-cuadrado, no presenta valores negativos, y siempre es asimétrica hacia la izquierda.
% }
% \end{frame}
% 
% 
% %---------------------------------------------------------------------slide----
% \begin{frame}
% \frametitle{Propiedades de la distribución F de Fisher-Snedecor $F(m,n)$}
% \begin{itemize}
% \item No está definida para valores negativos.
% \item De la definición se deduce que
% \[
% F(m,n) =\frac{1}{F(n,m)}
% \]
% de manera que si llamamos $f(m,n)_p$ al valor que cumple que $P(F(m,n)≤f(m,n)_p)=p$, entonces se cumple
% \[                                          
% f(m,n)_p =\frac{1}{f(n,m)_{1−p}}
% \]
% Esto resulta muy útil para utilizar las tablas de su función de distribución.
% \end{itemize}
% 
% \note{
% Algunas propiedades que tiene la distribución F de Fisher son:
% \begin{itemize}
% \item No está definida para valores negativos.
% \item De la definición se deduce que
% \[
% F(m,n) =\frac{1}{F(n,m)}
% \]
% de manera que si llamamos $f(m,n)_p$ al valor que cumple que $P(F(m,n)≤f(m,n)_p)=p$, entonces se cumple
% \[                                          
% f(m,n)_p =\frac{1}{f(n,m)_{1−p}}
% \]
% Esto resulta muy útil para utilizar las tablas de su función de distribución.
% \end{itemize}
% }
% \end{frame}